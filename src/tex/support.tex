% -*- mode: Noweb; noweb-code-mode: tex-mode -*-% ===> this file was generated automatically by noweave --- better not edit it
\documentclass[twoside]{article}
\usepackage[hypertex]{hyperref}
\usepackage{url}
\usepackage{noweb}
\pagestyle{noweb}
\noweboptions{longchunks,smallcode}
\title{{\TeX} and {\LaTeX} support for {\tt noweb}}
\author{Norman Ramsey}
\newcommand{\stylehook}{\marginpar{\raggedright\sl Style hook}}
\begin{document}
\maketitle
\tableofcontents
\nwfilename{support.nw}\nwbegindocs{1}\nwdocspar
This document describes the {\TeX} code that supports {\tt noweave}
and {\tt noweb}.  Those interested in customizing their output should
focus on Section~\ref{section:sty}.  
Hooks you can easily use (apart from those provided by
{\Tt{}{\nwbackslash}noweboptions\nwendquote}) are indicated by marginal notes.
This file contains both plain {\TeX} and {\LaTeX} support:
\nwenddocs{}\nwbegincode{2}\sublabel{NW4Nr7fb-38jgpJ-1}\nwmargintag{{\nwtagstyle{}\subpageref{NW4Nr7fb-38jgpJ-1}}}\moddef{nwmac.tex~{\nwtagstyle{}\subpageref{NW4Nr7fb-38jgpJ-1}}}\endmoddef\nwstartdeflinemarkup\nwprevnextdefs{\relax}{NW4Nr7fb-38jgpJ-2}\nwenddeflinemarkup
% nwmac.tex -- plain TeX support for noweb
% DON'T read or edit this file!  Use ...noweb-source/tex/support.nw instead.
\nwalsodefined{\\{NW4Nr7fb-38jgpJ-2}\\{NW4Nr7fb-38jgpJ-3}\\{NW4Nr7fb-38jgpJ-4}\\{NW4Nr7fb-38jgpJ-5}\\{NW4Nr7fb-38jgpJ-6}\\{NW4Nr7fb-38jgpJ-7}\\{NW4Nr7fb-38jgpJ-8}\\{NW4Nr7fb-38jgpJ-9}\\{NW4Nr7fb-38jgpJ-A}\\{NW4Nr7fb-38jgpJ-B}\\{NW4Nr7fb-38jgpJ-C}\\{NW4Nr7fb-38jgpJ-D}\\{NW4Nr7fb-38jgpJ-E}\\{NW4Nr7fb-38jgpJ-F}\\{NW4Nr7fb-38jgpJ-G}\\{NW4Nr7fb-38jgpJ-H}}\nwnotused{nwmac.tex}\nwendcode{}\nwbegincode{3}\sublabel{NW4Nr7fb-3olr1Q-1}\nwmargintag{{\nwtagstyle{}\subpageref{NW4Nr7fb-3olr1Q-1}}}\moddef{noweb.sty~{\nwtagstyle{}\subpageref{NW4Nr7fb-3olr1Q-1}}}\endmoddef\nwstartdeflinemarkup\nwprevnextdefs{\relax}{NW4Nr7fb-3olr1Q-2}\nwenddeflinemarkup
% noweb.sty -- LaTeX support for noweb
% DON'T read or edit this file!  Use ...noweb-source/tex/support.nw instead.
\nwalsodefined{\\{NW4Nr7fb-3olr1Q-2}\\{NW4Nr7fb-3olr1Q-3}\\{NW4Nr7fb-3olr1Q-4}\\{NW4Nr7fb-3olr1Q-5}\\{NW4Nr7fb-3olr1Q-6}\\{NW4Nr7fb-3olr1Q-7}\\{NW4Nr7fb-3olr1Q-8}\\{NW4Nr7fb-3olr1Q-9}\\{NW4Nr7fb-3olr1Q-A}\\{NW4Nr7fb-3olr1Q-B}\\{NW4Nr7fb-3olr1Q-C}\\{NW4Nr7fb-3olr1Q-D}\\{NW4Nr7fb-3olr1Q-E}\\{NW4Nr7fb-3olr1Q-F}\\{NW4Nr7fb-3olr1Q-G}\\{NW4Nr7fb-3olr1Q-H}\\{NW4Nr7fb-3olr1Q-I}\\{NW4Nr7fb-3olr1Q-J}\\{NW4Nr7fb-3olr1Q-K}\\{NW4Nr7fb-3olr1Q-L}\\{NW4Nr7fb-3olr1Q-M}\\{NW4Nr7fb-3olr1Q-N}\\{NW4Nr7fb-3olr1Q-O}\\{NW4Nr7fb-3olr1Q-P}\\{NW4Nr7fb-3olr1Q-Q}\\{NW4Nr7fb-3olr1Q-R}\\{NW4Nr7fb-3olr1Q-S}\\{NW4Nr7fb-3olr1Q-T}\\{NW4Nr7fb-3olr1Q-U}\\{NW4Nr7fb-3olr1Q-V}\\{NW4Nr7fb-3olr1Q-W}\\{NW4Nr7fb-3olr1Q-X}\\{NW4Nr7fb-3olr1Q-Y}\\{NW4Nr7fb-3olr1Q-Z}\\{NW4Nr7fb-3olr1Q-a}\\{NW4Nr7fb-3olr1Q-b}\\{NW4Nr7fb-3olr1Q-c}\\{NW4Nr7fb-3olr1Q-d}\\{NW4Nr7fb-3olr1Q-e}\\{NW4Nr7fb-3olr1Q-f}\\{NW4Nr7fb-3olr1Q-g}\\{NW4Nr7fb-3olr1Q-h}\\{NW4Nr7fb-3olr1Q-i}\\{NW4Nr7fb-3olr1Q-j}\\{NW4Nr7fb-3olr1Q-k}\\{NW4Nr7fb-3olr1Q-l}\\{NW4Nr7fb-3olr1Q-m}\\{NW4Nr7fb-3olr1Q-n}\\{NW4Nr7fb-3olr1Q-o}\\{NW4Nr7fb-3olr1Q-p}\\{NW4Nr7fb-3olr1Q-q}\\{NW4Nr7fb-3olr1Q-r}\\{NW4Nr7fb-3olr1Q-s}\\{NW4Nr7fb-3olr1Q-t}\\{NW4Nr7fb-3olr1Q-u}\\{NW4Nr7fb-3olr1Q-v}\\{NW4Nr7fb-3olr1Q-w}\\{NW4Nr7fb-3olr1Q-x}\\{NW4Nr7fb-3olr1Q-y}\\{NW4Nr7fb-3olr1Q-z}\\{NW4Nr7fb-3olr1Q-10}\\{NW4Nr7fb-3olr1Q-11}\\{NW4Nr7fb-3olr1Q-12}\\{NW4Nr7fb-3olr1Q-13}\\{NW4Nr7fb-3olr1Q-14}\\{NW4Nr7fb-3olr1Q-15}\\{NW4Nr7fb-3olr1Q-16}\\{NW4Nr7fb-3olr1Q-17}\\{NW4Nr7fb-3olr1Q-18}\\{NW4Nr7fb-3olr1Q-19}\\{NW4Nr7fb-3olr1Q-1A}\\{NW4Nr7fb-3olr1Q-1B}\\{NW4Nr7fb-3olr1Q-1C}}\nwnotused{noweb.sty}\nwendcode{}\nwbegindocs{4}\nwdocspar

\section{Basic {\TeX} support for {\tt noweb}}
This basic code is used for both {\TeX} and {\LaTeX}.
The first step is to define {\Tt{}{\nwbackslash}codehsize\nwendquote}, which is the width in
which code is set, and {\Tt{}{\nwbackslash}codemargin\nwendquote}, which is the amount by which
it is indented.\stylehook
\nwenddocs{}\nwbegincode{5}\sublabel{NW4Nr7fb-eb66b-1}\nwmargintag{{\nwtagstyle{}\subpageref{NW4Nr7fb-eb66b-1}}}\moddef{kernel~{\nwtagstyle{}\subpageref{NW4Nr7fb-eb66b-1}}}\endmoddef\nwstartdeflinemarkup\nwusesondefline{\\{NW4Nr7fb-3olr1Q-5}\\{NW4Nr7fb-38jgpJ-D}}\nwprevnextdefs{\relax}{NW4Nr7fb-eb66b-2}\nwenddeflinemarkup
% make \\hsize in code sufficient for 88 columns
\\setbox0=\\hbox\{\\tt m\}
\\newdimen\\codehsize
\\codehsize=91\\wd0 % 88 columns wasn't enough; I don't know why
\\newdimen\\codemargin
\\codemargin=0pt
\nwalsodefined{\\{NW4Nr7fb-eb66b-2}\\{NW4Nr7fb-eb66b-3}\\{NW4Nr7fb-eb66b-4}\\{NW4Nr7fb-eb66b-5}\\{NW4Nr7fb-eb66b-6}\\{NW4Nr7fb-eb66b-7}\\{NW4Nr7fb-eb66b-8}\\{NW4Nr7fb-eb66b-9}\\{NW4Nr7fb-eb66b-A}\\{NW4Nr7fb-eb66b-B}}\nwused{\\{NW4Nr7fb-3olr1Q-5}\\{NW4Nr7fb-38jgpJ-D}}\nwendcode{}\nwbegindocs{6}\nwdocspar
{\Tt{}{\nwbackslash}defspace\nwendquote} is the space we would like on the right of navigational info
that appears on definition lines, so that it lines up with the text above
and below.
\nwenddocs{}\nwbegincode{7}\sublabel{NW4Nr7fb-eb66b-2}\nwmargintag{{\nwtagstyle{}\subpageref{NW4Nr7fb-eb66b-2}}}\moddef{kernel~{\nwtagstyle{}\subpageref{NW4Nr7fb-eb66b-1}}}\plusendmoddef\nwstartdeflinemarkup\nwusesondefline{\\{NW4Nr7fb-3olr1Q-5}\\{NW4Nr7fb-38jgpJ-D}}\nwprevnextdefs{NW4Nr7fb-eb66b-1}{NW4Nr7fb-eb66b-3}\nwenddeflinemarkup
\\newdimen\\nwdefspace
\\nwdefspace=\\codehsize
% need to use \\textwidth in \{\\LaTeX\} to handle styles with
% non-standard margins (David Bruce).  Don't know why we sometimes
% wanted \\hsize.  27 August 1997.
%% \\advance\\nwdefspace by -\\hsize\\relax
\\ifx\\textwidth\\undefined
  \\advance\\nwdefspace by -\\hsize\\relax
\\else
  \\advance\\nwdefspace by -\\textwidth\\relax
\\fi
\nwused{\\{NW4Nr7fb-3olr1Q-5}\\{NW4Nr7fb-38jgpJ-D}}\nwendcode{}\nwbegindocs{8}\nwdocspar
Most code is set in an environment in which {\Tt{}{\nwbackslash}setupcode\nwendquote} has been
executed.
In this environment, only {\Tt{}{\nwbackslash}\nwendquote}, {\Tt{}{\nwlbrace}\nwendquote}, and {\Tt{}{\nwrbrace}\nwendquote} have their usual
categories; every other character represents itself.
Appropriate {\Tt{}{\nwbackslash}chardef\nwendquote}s ensure that the special characters can be
escaped with a backslash.
\nwenddocs{}\nwbegincode{9}\sublabel{NW4Nr7fb-eb66b-3}\nwmargintag{{\nwtagstyle{}\subpageref{NW4Nr7fb-eb66b-3}}}\moddef{kernel~{\nwtagstyle{}\subpageref{NW4Nr7fb-eb66b-1}}}\plusendmoddef\nwstartdeflinemarkup\nwusesondefline{\\{NW4Nr7fb-3olr1Q-5}\\{NW4Nr7fb-38jgpJ-D}}\nwprevnextdefs{NW4Nr7fb-eb66b-2}{NW4Nr7fb-eb66b-4}\nwenddeflinemarkup
\\chardef\\other=12
\\def\\setupcode\{%
  \\chardef\\\\=`\\\\
  \\chardef\\\{=`\\\{
  \\chardef\\\}=`\\\}
  \\catcode`\\$=\\other
  \\catcode`\\&=\\other
  \\catcode`\\#=\\other
  \\catcode`\\%=\\other
  \\catcode`\\~=\\other
  \\catcode`\\_=\\other
  \\catcode`\\^=\\other
  \\catcode`\\"=\\other    % fixes problem with german.sty
  \\obeyspaces\\Tt
\}
\nwused{\\{NW4Nr7fb-3olr1Q-5}\\{NW4Nr7fb-38jgpJ-D}}\nwendcode{}\nwbegindocs{10}\nwdocspar
{\Tt{}{\nwbackslash}nwendquote\nwendquote} is called after quoted code.
It resets the spacefactor
\nwenddocs{}\nwbegincode{11}\sublabel{NW4Nr7fb-eb66b-4}\nwmargintag{{\nwtagstyle{}\subpageref{NW4Nr7fb-eb66b-4}}}\moddef{kernel~{\nwtagstyle{}\subpageref{NW4Nr7fb-eb66b-1}}}\plusendmoddef\nwstartdeflinemarkup\nwusesondefline{\\{NW4Nr7fb-3olr1Q-5}\\{NW4Nr7fb-38jgpJ-D}}\nwprevnextdefs{NW4Nr7fb-eb66b-3}{NW4Nr7fb-eb66b-5}\nwenddeflinemarkup
\\def\\nwendquote\{\\relax\\ifhmode\\spacefactor=1000 \\fi\}
\nwused{\\{NW4Nr7fb-3olr1Q-5}\\{NW4Nr7fb-38jgpJ-D}}\nwendcode{}\nwbegindocs{12}\nwdocspar
{\Tt{}{\nwbackslash}eatline\nwendquote} is used to consume newlines that should be ignored,
for example, the newlines at the end of {\Tt{}@\ {\%}def\ \nwendquote}{\em identifiers} lines.
I can't remember what {\Tt{}{\nwbackslash}startline\nwendquote} or {\Tt{}{\nwbackslash}newlines\nwendquote} are for; I don't
think {\tt noweave} ever emits them.
\nwenddocs{}\nwbegincode{13}\sublabel{NW4Nr7fb-eb66b-5}\nwmargintag{{\nwtagstyle{}\subpageref{NW4Nr7fb-eb66b-5}}}\moddef{kernel~{\nwtagstyle{}\subpageref{NW4Nr7fb-eb66b-1}}}\plusendmoddef\nwstartdeflinemarkup\nwusesondefline{\\{NW4Nr7fb-3olr1Q-5}\\{NW4Nr7fb-38jgpJ-D}}\nwprevnextdefs{NW4Nr7fb-eb66b-4}{NW4Nr7fb-eb66b-6}\nwenddeflinemarkup
\{\\catcode`\\^^M=\\active % make CR an active character
  \\gdef\\newlines\{\\catcode`\\^^M=\\active % make CR an active character
         \\def^^M\{\\par\\startline\}\}%
  \\gdef\\eatline#1^^M\{\\relax\}%
\}
%%% DON'T   \\gdef^^M\{\\par\\startline\}\}% in case ^^M appears in a \\write
\\def\\startline\{\\noindent\\hskip\\parindent\\ignorespaces\}
\\def\\nwnewline\{\\ifvmode\\else\\hfil\\break\\leavevmode\\hbox\{\}\\fi\}
\nwused{\\{NW4Nr7fb-3olr1Q-5}\\{NW4Nr7fb-38jgpJ-D}}\nwendcode{}\nwbegindocs{14}\nwdocspar

Within a code environment, it may be necessary to restore the 
category codes in order to set a module (chunk) name.
This hack doesn't properly restore {\Tt{}"\nwendquote} for use in {\tt german.sty}.
\nwenddocs{}\nwbegincode{15}\sublabel{NW4Nr7fb-eb66b-6}\nwmargintag{{\nwtagstyle{}\subpageref{NW4Nr7fb-eb66b-6}}}\moddef{kernel~{\nwtagstyle{}\subpageref{NW4Nr7fb-eb66b-1}}}\plusendmoddef\nwstartdeflinemarkup\nwusesondefline{\\{NW4Nr7fb-3olr1Q-5}\\{NW4Nr7fb-38jgpJ-D}}\nwprevnextdefs{NW4Nr7fb-eb66b-5}{NW4Nr7fb-eb66b-7}\nwenddeflinemarkup
\\def\\setupmodname\{%
  \\catcode`\\$=3
  \\catcode`\\&=4
  \\catcode`\\#=6
  \\catcode`\\%=14
  \\catcode`\\~=13
  \\catcode`\\_=8
  \\catcode`\\^=7
  \\catcode`\\ =10
  \\catcode`\\^^M=5
  \\let\\\{\\lbrace
  \\let\\\}\\rbrace
  % bad news --- don't know what catcode to give "
  \\Rm\}
\nwused{\\{NW4Nr7fb-3olr1Q-5}\\{NW4Nr7fb-38jgpJ-D}}\nwendcode{}\nwbegindocs{16}\sublabel{ref:fred}
\nwenddocs{}\nwbegindocs{17}\nwdocspar
Setting up the space code has to be done differently for {\TeX} and
{\LaTeX}, so as not to screw up {\LaTeX}'s \texttt{verbatim} package.
(Fix from Rafael Laboissiere.)
\nwenddocs{}\nwbegincode{18}\sublabel{NW4Nr7fb-38jgpJ-2}\nwmargintag{{\nwtagstyle{}\subpageref{NW4Nr7fb-38jgpJ-2}}}\moddef{nwmac.tex~{\nwtagstyle{}\subpageref{NW4Nr7fb-38jgpJ-1}}}\plusendmoddef\nwstartdeflinemarkup\nwprevnextdefs{NW4Nr7fb-38jgpJ-1}{NW4Nr7fb-38jgpJ-3}\nwenddeflinemarkup
\{\\obeyspaces\\global\\let =\\ \} % from texbook, p 381
\nwendcode{}\nwbegincode{19}\sublabel{NW4Nr7fb-3olr1Q-2}\nwmargintag{{\nwtagstyle{}\subpageref{NW4Nr7fb-3olr1Q-2}}}\moddef{noweb.sty~{\nwtagstyle{}\subpageref{NW4Nr7fb-3olr1Q-1}}}\plusendmoddef\nwstartdeflinemarkup\nwprevnextdefs{NW4Nr7fb-3olr1Q-1}{NW4Nr7fb-3olr1Q-3}\nwenddeflinemarkup
\{\\obeyspaces\\AtBeginDocument\{\\global\\let =\\ \}\} % from texbook, p 381
\nwendcode{}\nwbegindocs{20}\nwdocspar

{\tt noweave} brackets uses of chunk names with {\Tt{}{\nwbackslash}LA\nwendquote} and {\Tt{}{\nwbackslash}RA\nwendquote}, which
handle the angle brackets, font, and environment.

As it stands, chunk names can be broken across lines (or pages).  This
could result in unnecessary page breaks in code
(c.f.~p.~\pageref{sec:pagebreaking}).  {\Tt{}{\nwbackslash}let{\nwbackslash}{\nwbackslash}maybehbox={\nwbackslash}mbox\nwendquote} to
\stylehook
avoid breaking them
(or to make them work in math mode); this is done in code chunks, but could be done
in general.
\nwenddocs{}\nwbegincode{21}\sublabel{NW4Nr7fb-eb66b-7}\nwmargintag{{\nwtagstyle{}\subpageref{NW4Nr7fb-eb66b-7}}}\moddef{kernel~{\nwtagstyle{}\subpageref{NW4Nr7fb-eb66b-1}}}\plusendmoddef\nwstartdeflinemarkup\nwusesondefline{\\{NW4Nr7fb-3olr1Q-5}\\{NW4Nr7fb-38jgpJ-D}}\nwprevnextdefs{NW4Nr7fb-eb66b-6}{NW4Nr7fb-eb66b-8}\nwenddeflinemarkup
\\def\\LA\{\\begingroup\\maybehbox\\bgroup\\setupmodname\\It$\\langle$\}
\\def\\RA\{\\/$\\rangle$\\egroup\\endgroup\}
\\def\\code\{\\leavevmode\\begingroup\\setupcode\\newlines\}
\\def\\edoc\{\\endgroup\}
\\let\\maybehbox\\relax
\nwused{\\{NW4Nr7fb-3olr1Q-5}\\{NW4Nr7fb-38jgpJ-D}}\nwendcode{}\nwbegindocs{22}\nwdocspar

{\Tt{}{\nwbackslash}equivbox\nwendquote} and {\Tt{}{\nwbackslash}plusequivbox\nwendquote} are used to set the
``\unhcopy\equivbox''
and ``\unhcopy\plusequivbox'' that open a chunk definition or its
continuation.
{\tt noweave} brackets definitions of chunk names with {\Tt{}{\nwbackslash}moddef\nwendquote} and
either {\Tt{}{\nwbackslash}endmoddef\nwendquote} or {\Tt{}{\nwbackslash}plusendmoddef\nwendquote}.
\nwenddocs{}\nwbegincode{23}\sublabel{NW4Nr7fb-eb66b-8}\nwmargintag{{\nwtagstyle{}\subpageref{NW4Nr7fb-eb66b-8}}}\moddef{kernel~{\nwtagstyle{}\subpageref{NW4Nr7fb-eb66b-1}}}\plusendmoddef\nwstartdeflinemarkup\nwusesondefline{\\{NW4Nr7fb-3olr1Q-5}\\{NW4Nr7fb-38jgpJ-D}}\nwprevnextdefs{NW4Nr7fb-eb66b-7}{NW4Nr7fb-eb66b-9}\nwenddeflinemarkup
\\newbox\\equivbox
\\setbox\\equivbox=\\hbox\{$\\equiv$\}
\\newbox\\plusequivbox
\\setbox\\plusequivbox=\\hbox\{$\\mathord\{+\}\\mathord\{\\equiv\}$\}
% \\moddef can't have an argument because there might be \\code...\\edoc
\\def\\moddef\{\\leavevmode\\kern-\\codemargin\\LA\}
\\def\\endmoddef\{\\RA\\ifmmode\\equiv\\else\\unhcopy\\equivbox\\fi
               \\nobreak\\hfill\\nobreak\}
\\def\\plusendmoddef\{\\RA\\ifmmode\\mathord\{+\}\\mathord\{\\equiv\}\\else\\unhcopy\\plusequivbox\\fi
               \\nobreak\\hfill\\nobreak\}
\nwused{\\{NW4Nr7fb-3olr1Q-5}\\{NW4Nr7fb-38jgpJ-D}}\nwendcode{}\nwbegindocs{24}\nwdocspar
Within a code environment, margin tags might be used to mark sub-page
numbers in the margins, separated by {\Tt{}{\nwbackslash}nwmarginglue\nwendquote}.\stylehook
The interaction with {\Tt{}{\nwbackslash}moddef\nwendquote} involves tricky kerning.
The tag itself is displayed using {\Tt{}{\nwbackslash}nwthemargintag\nwendquote}
\nwenddocs{}\nwbegincode{25}\sublabel{NW4Nr7fb-3olr1Q-3}\nwmargintag{{\nwtagstyle{}\subpageref{NW4Nr7fb-3olr1Q-3}}}\moddef{noweb.sty~{\nwtagstyle{}\subpageref{NW4Nr7fb-3olr1Q-1}}}\plusendmoddef\nwstartdeflinemarkup\nwprevnextdefs{NW4Nr7fb-3olr1Q-2}{NW4Nr7fb-3olr1Q-4}\nwenddeflinemarkup
\\def\\nwopt@nomargintag\{\\let\\nwmargintag=\\@gobble\}
\\def\\nwopt@margintag\{%
  \\def\\nwmargintag##1\{\\leavevmode\\llap\{##1\\kern\\nwmarginglue\\kern\\codemargin\}\}\}
\\def\\nwopt@margintag\{%
  \\def\\nwmargintag##1\{\\leavevmode\\kern-\\codemargin\\nwthemargintag\{##1\}\\kern\\codemargin\}\}
\\def\\nwthemargintag#1\{\\llap\{#1\\kern\\nwmarginglue\}\}
\\nwopt@margintag
\\newdimen\\nwmarginglue
\\nwmarginglue=0.3in
\nwendcode{}\nwbegindocs{26}\nwdocspar
\iffalse
\nwenddocs{}\nwbegincode{27}\sublabel{NW4Nr7fb-3RhSlV-1}\nwmargintag{{\nwtagstyle{}\subpageref{NW4Nr7fb-3RhSlV-1}}}\moddef{man page: \code{}{\nwbackslash}noweboptions\edoc{}~{\nwtagstyle{}\subpageref{NW4Nr7fb-3RhSlV-1}}}\endmoddef\nwstartdeflinemarkup\nwprevnextdefs{\relax}{NW4Nr7fb-3RhSlV-2}\nwenddeflinemarkup
.TP
.B margintag
Put the sub-page number (tag) of each code-chunk definition in the
left margin. (Default)
.TP
.B nomargintag
Don't use margin tags.
\nwalsodefined{\\{NW4Nr7fb-3RhSlV-2}\\{NW4Nr7fb-3RhSlV-3}\\{NW4Nr7fb-3RhSlV-4}\\{NW4Nr7fb-3RhSlV-5}\\{NW4Nr7fb-3RhSlV-6}\\{NW4Nr7fb-3RhSlV-7}\\{NW4Nr7fb-3RhSlV-8}\\{NW4Nr7fb-3RhSlV-9}\\{NW4Nr7fb-3RhSlV-A}\\{NW4Nr7fb-3RhSlV-B}\\{NW4Nr7fb-3RhSlV-C}}\nwnotused{man page: [[\noweboptions]]}\nwendcode{}\nwbegindocs{28}\fi
{\Tt{}{\nwbackslash}nwtagstyle\nwendquote} determines the style in which tags are displayed.\stylehook
\nwenddocs{}\nwbegincode{29}\sublabel{NW4Nr7fb-3olr1Q-4}\nwmargintag{{\nwtagstyle{}\subpageref{NW4Nr7fb-3olr1Q-4}}}\moddef{noweb.sty~{\nwtagstyle{}\subpageref{NW4Nr7fb-3olr1Q-1}}}\plusendmoddef\nwstartdeflinemarkup\nwprevnextdefs{NW4Nr7fb-3olr1Q-3}{NW4Nr7fb-3olr1Q-5}\nwenddeflinemarkup
\\def\\nwtagstyle\{\\footnotesize\\Rm\}
\nwendcode{}\nwbegindocs{30}\nwdocspar
\nwenddocs{}\nwbegincode{31}\sublabel{NW4Nr7fb-eb66b-9}\nwmargintag{{\nwtagstyle{}\subpageref{NW4Nr7fb-eb66b-9}}}\moddef{kernel~{\nwtagstyle{}\subpageref{NW4Nr7fb-eb66b-1}}}\plusendmoddef\nwstartdeflinemarkup\nwusesondefline{\\{NW4Nr7fb-3olr1Q-5}\\{NW4Nr7fb-38jgpJ-D}}\nwprevnextdefs{NW4Nr7fb-eb66b-8}{NW4Nr7fb-eb66b-A}\nwenddeflinemarkup
\\def\\chunklist\{%
\\errhelp\{I changed \\chunklist to \\nowebchunks.  
I'll try to avoid such incompatible changes in the future.\}%
\\errmessage\{Use \\string\\nowebchunks\\space instead of \\string\\chunklist\}\}
\\def\\nowebchunks\{\\message\{<Warning: You need noweave -x to use \\string\\nowebchunks>\}\}
\\def\\nowebindex\{\\message\{<Warning: You need noweave -index to use \\string\\nowebindex>\}\}
\nwused{\\{NW4Nr7fb-3olr1Q-5}\\{NW4Nr7fb-38jgpJ-D}}\nwendcode{}\nwbegindocs{32}We have to be careful with font-changing in the presence of
different font-selection schemes.  In the \LaTeX{} New Font Selection
Scheme something like {\Tt{}{\nwbackslash}it{\nwbackslash}tt\nwendquote} will attempt to use an italic
typewriter font.  Thus we define new commands like {\Tt{}{\nwbackslash}Tt\nwendquote} which will
work with both the Plain and old and new \LaTeX{} schemes.  (Note that
NFSS will be standard in the next version of \LaTeX.)  A problem with
these definitions arises with NFSS: in math mode the won't work
unless the {\tt oldlfont} backwards-compatibility option is in effect.
For the moment, you can get round this by using {\Tt{}{\nwbackslash}mbox\nwendquote}.

If you wanted code set in a different font, you could re-define
{\Tt{}{\nwbackslash}Tt\nwendquote}.\stylehook{}  [\LaTeX2e actually behaves like OFSS, but the
extra {\Tt{}{\nwbackslash}reset@font\nwendquote} does no harm.]
\nwenddocs{}\nwbegincode{33}\sublabel{NW4Nr7fb-eb66b-A}\nwmargintag{{\nwtagstyle{}\subpageref{NW4Nr7fb-eb66b-A}}}\moddef{kernel~{\nwtagstyle{}\subpageref{NW4Nr7fb-eb66b-1}}}\plusendmoddef\nwstartdeflinemarkup\nwusesondefline{\\{NW4Nr7fb-3olr1Q-5}\\{NW4Nr7fb-38jgpJ-D}}\nwprevnextdefs{NW4Nr7fb-eb66b-9}{NW4Nr7fb-eb66b-B}\nwenddeflinemarkup
% here is support for the new-style (capitalized) font-changing commands
% thanks to Dave Love
\\ifx\\documentstyle\\undefined
  \\let\\Rm=\\rm \\let\\It=\\it \\let\\Tt=\\tt       % plain
\\else\\ifx\\selectfont\\undefined
  \\let\\Rm=\\rm \\let\\It=\\it \\let\\Tt=\\tt       % LaTeX OFSS
\\else                                       % LaTeX NFSS
  \\def\\Rm\{\\reset@font\\rm\}
  \\def\\It\{\\reset@font\\it\}
  \\def\\Tt\{\\reset@font\\tt\}
  \\def\\Bf\{\\reset@font\\bf\}
\\fi\\fi
\\ifx\\reset@font\\undefined \\let\\reset@font=\\relax \\fi
\nwused{\\{NW4Nr7fb-3olr1Q-5}\\{NW4Nr7fb-38jgpJ-D}}\nwendcode{}\nwbegindocs{34}\nwdocspar
\clearpage
\section{The {\tt noweb} document-style option for {\LaTeX}}
\label{section:sty}
{\LaTeX} support begins with the kernel shown above.
\nwenddocs{}\nwbegincode{35}\sublabel{NW4Nr7fb-3olr1Q-5}\nwmargintag{{\nwtagstyle{}\subpageref{NW4Nr7fb-3olr1Q-5}}}\moddef{noweb.sty~{\nwtagstyle{}\subpageref{NW4Nr7fb-3olr1Q-1}}}\plusendmoddef\nwstartdeflinemarkup\nwprevnextdefs{NW4Nr7fb-3olr1Q-4}{NW4Nr7fb-3olr1Q-6}\nwenddeflinemarkup
\LA{}kernel~{\nwtagstyle{}\subpageref{NW4Nr7fb-eb66b-1}}\RA{}
\nwendcode{}\nwbegindocs{36}\nwdocspar
\subsection{Support for noweb options}
\nwenddocs{}\nwbegincode{37}\sublabel{NW4Nr7fb-3olr1Q-6}\nwmargintag{{\nwtagstyle{}\subpageref{NW4Nr7fb-3olr1Q-6}}}\moddef{noweb.sty~{\nwtagstyle{}\subpageref{NW4Nr7fb-3olr1Q-1}}}\plusendmoddef\nwstartdeflinemarkup\nwprevnextdefs{NW4Nr7fb-3olr1Q-5}{NW4Nr7fb-3olr1Q-7}\nwenddeflinemarkup
\\def\\noweboptions#1\{%
  \\def\\@nwoptionlist\{#1\}%
  \\@for\\@nwoption:=\\@nwoptionlist\\do\{%
    \\@ifundefined\{nwopt@\\@nwoption\}\{%
        \\@latexerr\{There is no such noweb option as '\\@nwoption'\}\\@eha\}\{%
        \\csname nwopt@\\@nwoption\\endcsname\}\}\}
\nwendcode{}\nwbegindocs{38}\nwdocspar
\subsection{Adjusting placement of code on the page}
{\LaTeX} requires a larger {\Tt{}{\nwbackslash}codehsize\nwendquote} because 
code is indented by {\Tt{}{\nwbackslash}codemargin\nwendquote}.\stylehook
\nwenddocs{}\nwbegincode{39}\sublabel{NW4Nr7fb-3olr1Q-7}\nwmargintag{{\nwtagstyle{}\subpageref{NW4Nr7fb-3olr1Q-7}}}\moddef{noweb.sty~{\nwtagstyle{}\subpageref{NW4Nr7fb-3olr1Q-1}}}\plusendmoddef\nwstartdeflinemarkup\nwprevnextdefs{NW4Nr7fb-3olr1Q-6}{NW4Nr7fb-3olr1Q-8}\nwenddeflinemarkup
\\codemargin=10pt
\\advance\\codehsize by \\codemargin       % make room for indentation of code
\\advance\\nwdefspace by \\codemargin      % and fix adjustment for def/use
\\def\\setcodemargin#1\{%
  \\advance\\codehsize by -\\codemargin       % make room for indentation of code
  \\advance\\nwdefspace by -\\codemargin   % and fix adjustment for def/use
  \\codemargin=#1
  \\advance\\codehsize by \\codemargin       % make room for indentation of code
  \\advance\\nwdefspace by \\codemargin    % and fix adjustment for
                                        % def/use
\}
\nwendcode{}\nwbegindocs{40}\nwdocspar
{\Tt{}{\nwbackslash}noweboptions{\nwlbrace}shift{\nwrbrace}\nwendquote} is used to shift the whole page left to make room for
the wide code lines.
It may be emitted by {\tt noweave -shift}, or it might be given by a user.
\nwenddocs{}\nwbegincode{41}\sublabel{NW4Nr7fb-3olr1Q-8}\nwmargintag{{\nwtagstyle{}\subpageref{NW4Nr7fb-3olr1Q-8}}}\moddef{noweb.sty~{\nwtagstyle{}\subpageref{NW4Nr7fb-3olr1Q-1}}}\plusendmoddef\nwstartdeflinemarkup\nwprevnextdefs{NW4Nr7fb-3olr1Q-7}{NW4Nr7fb-3olr1Q-9}\nwenddeflinemarkup
\\def\\nwopt@shift\{%
  \\dimen@=-0.8in
  \\if@twoside                 % Values for two-sided printing:
     \\advance\\evensidemargin by \\dimen@
  \\else                       % Values for one-sided printing:
     \\advance\\evensidemargin by \\dimen@
     \\advance\\oddsidemargin by \\dimen@
  \\fi
%  \\advance \\marginparwidth -\\dimen@
\}
\\let\\nwopt@noshift\\@empty
\nwendcode{}\nwbegindocs{42}\nwdocspar

\iffalse
\nwenddocs{}\nwbegincode{43}\sublabel{NW4Nr7fb-3RhSlV-2}\nwmargintag{{\nwtagstyle{}\subpageref{NW4Nr7fb-3RhSlV-2}}}\moddef{man page: \code{}{\nwbackslash}noweboptions\edoc{}~{\nwtagstyle{}\subpageref{NW4Nr7fb-3RhSlV-1}}}\plusendmoddef\nwstartdeflinemarkup\nwprevnextdefs{NW4Nr7fb-3RhSlV-1}{NW4Nr7fb-3RhSlV-3}\nwenddeflinemarkup
.TP
.B shift
Shift text to the left so that long code lines won't extend
off the right-hand side of the page.
\nwendcode{}\nwbegindocs{44}\fi

\subsection{Page-breaking strategy}\label{sec:pagebreaking}

We want to insert penalties aiming for:
\begin{enumerate}
\item 
No page breaks in the middle of a code chunk unless necessary to avoid
an overfull vbox;
\item 
Documentation immediately preceding a code chunk should appear on
the same page as that code chunk unless doing so would violate rule 1.
\end{enumerate}
{\Tt{}{\nwbackslash}filbreak\nwendquote} is useful for this sort of thing (see {\em The \TeX
  book\/}) and is used to encourage breaks at the right places between
chunks.  Appropriate penalties are inserted elsewhere, between code
lines in particular.

\subsection{Environments for setting code}

{\Tt{}{\nwbackslash}nwbegincode\nwendquote} and {\Tt{}{\nwbackslash}nwendcode\nwendquote} are used by {\tt noweave} to bracket
code chunks.
The {\Tt{}webcode\nwendquote} environment is intended for users who want to paste
{\tt noweave} output into papers.

The definition of {\Tt{}{\nwbackslash}nwbegincode\nwendquote} is based on the verbatim
implementation in {\tt verbatim.sty}, which will, presumably be in the
next version of \LaTeX\@.  One thing it does differently, apart from
the catcode changes is setting {\Tt{}{\nwbackslash}linewidth\nwendquote}; this will avoid some
overfull hboxen when the code lines are too long, but the lines won't
be broken anyhow (even within chunk names because of the
{\Tt{}{\nwbackslash}maybehbox\nwendquote} definition).
\nwenddocs{}\nwbegincode{45}\sublabel{NW4Nr7fb-3olr1Q-9}\nwmargintag{{\nwtagstyle{}\subpageref{NW4Nr7fb-3olr1Q-9}}}\moddef{noweb.sty~{\nwtagstyle{}\subpageref{NW4Nr7fb-3olr1Q-1}}}\plusendmoddef\nwstartdeflinemarkup\nwprevnextdefs{NW4Nr7fb-3olr1Q-8}{NW4Nr7fb-3olr1Q-A}\nwenddeflinemarkup
\\def\\nwbegincode#1\{%
  \\begingroup
  \LA{}\code{}{\nwbackslash}nwbegincode\edoc{} separation and penalties~{\nwtagstyle{}\subpageref{NW4Nr7fb-23V9Kn-1}}\RA{}
  \\@begincode \}
\\def\\nwendcode\{\\endtrivlist \\endgroup \\filbreak\} % keeps code on 1 page

\\newenvironment\{webcode\}\{%
  \\@begincode
\}\{%
  \\endtrivlist\}
\nwendcode{}\nwbegindocs{46}This is just common code between {\Tt{}{\nwbackslash}nwbegincode\nwendquote} and {\Tt{}webcode\nwendquote}.
\nwenddocs{}\nwbegincode{47}\sublabel{NW4Nr7fb-3olr1Q-A}\nwmargintag{{\nwtagstyle{}\subpageref{NW4Nr7fb-3olr1Q-A}}}\moddef{noweb.sty~{\nwtagstyle{}\subpageref{NW4Nr7fb-3olr1Q-1}}}\plusendmoddef\nwstartdeflinemarkup\nwprevnextdefs{NW4Nr7fb-3olr1Q-9}{NW4Nr7fb-3olr1Q-B}\nwenddeflinemarkup
\\def\\@begincode\{%
  \LA{}\code{}{\nwbackslash}trivlist\edoc{} clich\'e (\`a la {\Tt verbatim})~{\nwtagstyle{}\subpageref{NW4Nr7fb-2VyA2L-1}}\RA{}
  \\linewidth\\codehsize
  \LA{}\code{}{\nwbackslash}obeylines\edoc{} setup~{\nwtagstyle{}\subpageref{NW4Nr7fb-2oTcJP-1}}\RA{}
  \LA{}zap ligatures, fix spaces~{\nwtagstyle{}\subpageref{NW4Nr7fb-1YqcEl-1}}\RA{}
  \\nowebsize \\setupcode
  \\let\\maybehbox\\mbox \}
\nwendcode{}\nwbegindocs{48}\iffalse
\nwenddocs{}\nwbegincode{49}\sublabel{NW4Nr7fb-3RhSlV-3}\nwmargintag{{\nwtagstyle{}\subpageref{NW4Nr7fb-3RhSlV-3}}}\moddef{man page: \code{}{\nwbackslash}noweboptions\edoc{}~{\nwtagstyle{}\subpageref{NW4Nr7fb-3RhSlV-1}}}\plusendmoddef\nwstartdeflinemarkup\nwprevnextdefs{NW4Nr7fb-3RhSlV-2}{NW4Nr7fb-3RhSlV-4}\nwenddeflinemarkup
.TP
.B smallcode
Set code in 
.I LaTeX
.B "\\\\\\\\small"
font instead of 
.B "\\\\\\\\normalsize."
Similar options exist for all the 
.I LaTeX
size-changing commands.
\nwendcode{}\nwbegindocs{50}\fi
\nwenddocs{}\nwbegindocs{51}{\Tt{}{\nwbackslash}nowebsize\nwendquote} governs the size at which code is set; users who want
to minimize code can {\Tt{}{\nwbackslash}let{\nwbackslash}nowebsize={\nwbackslash}small\nwendquote}.  
Slitex users should try
\begin{quote}
{\Tt{}{\nwbackslash}def{\nwbackslash}nowebsize{\nwlbrace}{\nwbackslash}normalsize{\nwbackslash}baselineskip=20pt\ {\nwbackslash}parskip=5pt\ {\nwrbrace}\nwendquote}
\end{quote}
to avoid code lines that are too far apart.
{\Tt{}{\nwbackslash}nwcodetopsep\nwendquote} is
the glue placed before code chunks.\stylehook
\nwenddocs{}\nwbegincode{52}\sublabel{NW4Nr7fb-3olr1Q-B}\nwmargintag{{\nwtagstyle{}\subpageref{NW4Nr7fb-3olr1Q-B}}}\moddef{noweb.sty~{\nwtagstyle{}\subpageref{NW4Nr7fb-3olr1Q-1}}}\plusendmoddef\nwstartdeflinemarkup\nwprevnextdefs{NW4Nr7fb-3olr1Q-A}{NW4Nr7fb-3olr1Q-C}\nwenddeflinemarkup
  \\newskip\\nwcodetopsep \\nwcodetopsep = 3pt plus 1.2pt minus 1pt
  \\let\\nowebsize=\\normalsize
  \\def\\nwopt@tinycode\{\\let\\nowebsize=\\tiny\}
  \\def\\nwopt@footnotesizecode\{\\let\\nowebsize=\\footnotesize\}
  \\def\\nwopt@scriptsizecode\{\\let\\nowebsize=\\scriptsize\}
  \\def\\nwopt@smallcode\{\\let\\nowebsize=\\small\}
  \\def\\nwopt@normalsizecode\{\\let\\nowebsize=\\normalsize\}
  \\def\\nwopt@largecode\{\\let\\nowebsize=\\large\}
  \\def\\nwopt@Largecode\{\\let\\nowebsize=\\Large\}
  \\def\\nwopt@LARGEcode\{\\let\\nowebsize=\\LARGE\}
  \\def\\nwopt@hugecode\{\\let\\nowebsize=\\huge\}
  \\def\\nwopt@Hugecode\{\\let\\nowebsize=\\Huge\}
\nwendcode{}\nwbegindocs{53}Maybe the penalties ought to be parameters\dots
\nwenddocs{}\nwbegincode{54}\sublabel{NW4Nr7fb-23V9Kn-1}\nwmargintag{{\nwtagstyle{}\subpageref{NW4Nr7fb-23V9Kn-1}}}\moddef{\code{}{\nwbackslash}nwbegincode\edoc{} separation and penalties~{\nwtagstyle{}\subpageref{NW4Nr7fb-23V9Kn-1}}}\endmoddef\nwstartdeflinemarkup\nwusesondefline{\\{NW4Nr7fb-3olr1Q-9}}\nwenddeflinemarkup
  \\topsep \\nwcodetopsep
  \\@beginparpenalty \\@highpenalty
  \\@endparpenalty -\\@highpenalty
\nwused{\\{NW4Nr7fb-3olr1Q-9}}\nwendcode{}\nwbegindocs{55}\nwdocspar
The {\Tt{}{\nwbackslash}trivlist\nwendquote} clich\'e isn't quite a clich\'e because we adjust
{\Tt{}{\nwbackslash}leftskip\nwendquote} for indentation by {\Tt{}{\nwbackslash}codemargin\nwendquote} and adjust
{\Tt{}{\nwbackslash}rightskip\nwendquote} to allow lines up to {\Tt{}{\nwbackslash}codehsize\nwendquote} long without
overfull boxen
($\mbox{{\Tt{}{\nwbackslash}codehsize\nwendquote}}=\mbox{{\Tt{}{\nwbackslash}hsize\nwendquote}}+\mbox{{\Tt{}{\nwbackslash}rightskip\nwendquote}}$).
Note that {\Tt{}{\nwbackslash}hsize\nwendquote} isn't altered.
\nwenddocs{}\nwbegincode{56}\sublabel{NW4Nr7fb-2VyA2L-1}\nwmargintag{{\nwtagstyle{}\subpageref{NW4Nr7fb-2VyA2L-1}}}\moddef{\code{}{\nwbackslash}trivlist\edoc{} clich\'e (\`a la {\Tt verbatim})~{\nwtagstyle{}\subpageref{NW4Nr7fb-2VyA2L-1}}}\endmoddef\nwstartdeflinemarkup\nwusesondefline{\\{NW4Nr7fb-3olr1Q-A}}\nwenddeflinemarkup
  \\trivlist \\item[]%
  \\leftskip\\@totalleftmargin \\advance\\leftskip\\codemargin
  \\rightskip\\hsize \\advance\\rightskip -\\codehsize
  \\parskip\\z@ \\parindent\\z@ \\parfillskip\\@flushglue
\nwused{\\{NW4Nr7fb-3olr1Q-A}}\nwendcode{}\nwbegindocs{57}The penalty inserted between verbatim lines would normally be
{\Tt{}{\nwbackslash}interlinepenalty\nwendquote}, but we want to prohibit breaks there.
\nwenddocs{}\nwbegindocs{58}\nwdocspar
Note the bug lurking somewhere in this code, as reported by Steven Ooms:
\begin{quote}
I have some lay-out
problems in the documentation chunks.  When using the (La)TeX commands
{\Tt{}{\nwbackslash}hline\nwendquote} or {\Tt{}{\nwbackslash}vtop\nwendquote} the right margin is always extended far beyond the page
margin after the first code chunk has been typeset.  I'm still looking for
the exact cause of it, but to me it seems that LaTeX supposes for those
commands that the line width for the documentation chunk is as large as that
for code chunks, which isn't true in reality.
\end{quote}
\nwenddocs{}\nwbegindocs{59}\nwdocspar
\nwenddocs{}\nwbegincode{60}\sublabel{NW4Nr7fb-2oTcJP-1}\nwmargintag{{\nwtagstyle{}\subpageref{NW4Nr7fb-2oTcJP-1}}}\moddef{\code{}{\nwbackslash}obeylines\edoc{} setup~{\nwtagstyle{}\subpageref{NW4Nr7fb-2oTcJP-1}}}\endmoddef\nwstartdeflinemarkup\nwusesondefline{\\{NW4Nr7fb-3olr1Q-A}}\nwenddeflinemarkup
  \\@@par
  \\def\\par\{\\leavevmode\\null \\@@par \\penalty\\nwcodepenalty\}%
  \\obeylines
\nwused{\\{NW4Nr7fb-3olr1Q-A}}\nwendcode{}\nwbegindocs{61}{\Tt{}{\nwbackslash}nwcodepenalty\nwendquote} is the penalty for breaking between lines in a
code chunk.  If you set it to 10000, code will never be broken across
pages.\stylehook{}  I guess this should be settable in {\Tt{}{\nwbackslash}noweboptions\nwendquote}.
\nwenddocs{}\nwbegincode{62}\sublabel{NW4Nr7fb-3olr1Q-C}\nwmargintag{{\nwtagstyle{}\subpageref{NW4Nr7fb-3olr1Q-C}}}\moddef{noweb.sty~{\nwtagstyle{}\subpageref{NW4Nr7fb-3olr1Q-1}}}\plusendmoddef\nwstartdeflinemarkup\nwprevnextdefs{NW4Nr7fb-3olr1Q-B}{NW4Nr7fb-3olr1Q-D}\nwenddeflinemarkup
\\newcount\\nwcodepenalty  \\nwcodepenalty=\\@highpenalty
\nwendcode{}\nwbegindocs{63}\nwdocspar
The cursing chunk accounts for the addition of a mess of characters
to those reset by {\Tt{}{\nwbackslash}@noligs\nwendquote} in \LaTeX2e.
\nwenddocs{}\nwbegincode{64}\sublabel{NW4Nr7fb-1YqcEl-1}\nwmargintag{{\nwtagstyle{}\subpageref{NW4Nr7fb-1YqcEl-1}}}\moddef{zap ligatures, fix spaces~{\nwtagstyle{}\subpageref{NW4Nr7fb-1YqcEl-1}}}\endmoddef\nwstartdeflinemarkup\nwusesondefline{\\{NW4Nr7fb-3olr1Q-A}}\nwenddeflinemarkup
  \\@noligs \LA{}make all those damn active characters ``other''~{\nwtagstyle{}\subpageref{NW4Nr7fb-2GjufW-1}}\RA{}
  \\setupcode \\frenchspacing \\@vobeyspaces
\nwused{\\{NW4Nr7fb-3olr1Q-A}}\nwendcode{}\nwbegindocs{65}\nwdocspar
We can't make {\Tt{}`\nwendquote} ``other,'' because then we'll get ligatures.
(Why Don put these ligatures in the {\Tt{}{\nwbackslash}tt\nwendquote} font I wish I knew.)
But we'll step on all the others.
\nwenddocs{}\nwbegincode{66}\sublabel{NW4Nr7fb-2GjufW-1}\nwmargintag{{\nwtagstyle{}\subpageref{NW4Nr7fb-2GjufW-1}}}\moddef{make all those damn active characters ``other''~{\nwtagstyle{}\subpageref{NW4Nr7fb-2GjufW-1}}}\endmoddef\nwstartdeflinemarkup\nwusesondefline{\\{NW4Nr7fb-1YqcEl-1}}\nwenddeflinemarkup
  \\ifx\\verbatim@nolig@list\\undefined\\else
    \\let\\do=\\nw@makeother \\verbatim@nolig@list \\do@noligs\\`
  \\fi
\nwused{\\{NW4Nr7fb-1YqcEl-1}}\nwendcode{}\nwbegincode{67}\sublabel{NW4Nr7fb-3olr1Q-D}\nwmargintag{{\nwtagstyle{}\subpageref{NW4Nr7fb-3olr1Q-D}}}\moddef{noweb.sty~{\nwtagstyle{}\subpageref{NW4Nr7fb-3olr1Q-1}}}\plusendmoddef\nwstartdeflinemarkup\nwprevnextdefs{NW4Nr7fb-3olr1Q-C}{NW4Nr7fb-3olr1Q-E}\nwenddeflinemarkup
\\def\\nw@makeother#1\{\\catcode`#1=12 \}
\nwendcode{}\nwbegindocs{68}\nwdocspar
{\tt noweave} uses {\Tt{}{\nwbackslash}nwbegindocs{\nwlbrace}nnn{\nwrbrace}\nwendquote} and {\Tt{}{\nwbackslash}nwenddocs\nwendquote} to bracket
documentation chunks.
If a documentation chunk does not continue the current paragraph, 
{\tt noweave} inserts {\Tt{}{\nwbackslash}nwdocspar\nwendquote}, which uses
{\Tt{}{\nwbackslash}filbreak\nwendquote} in an attempt to keep the documentation chunk on the
same page as the code chunk that follows it.  (The code chunk will
have another {\Tt{}{\nwbackslash}filbreak\nwendquote} after it---see {\Tt{}{\nwbackslash}nwbegincode\nwendquote}.)
{\Tt{}{\nwbackslash}nwbegindocs\nwendquote} doesn't start a
new paragraph if the previous chunk didn't end one, i.e.\ didn't enter
vmode; if it does start a new one, it's only indented by the use of
{\Tt{}{\nwbackslash}nwdocspar\nwendquote}.
\nwenddocs{}\nwbegincode{69}\sublabel{NW4Nr7fb-3olr1Q-E}\nwmargintag{{\nwtagstyle{}\subpageref{NW4Nr7fb-3olr1Q-E}}}\moddef{noweb.sty~{\nwtagstyle{}\subpageref{NW4Nr7fb-3olr1Q-1}}}\plusendmoddef\nwstartdeflinemarkup\nwprevnextdefs{NW4Nr7fb-3olr1Q-D}{NW4Nr7fb-3olr1Q-F}\nwenddeflinemarkup
\\def\\nwbegindocs#1\{\\ifvmode\\noindent\\fi\}
\\let\\nwenddocs=\\relax
\\let\\nwdocspar=\\filbreak
\nwendcode{}\nwbegindocs{70}\nwdocspar
Some people don't like it that noweb leaves so much white space.
The secret, undocument style option {\Tt{}breakcode\nwendquote} will break up code
in order to use less white space.
The parameter {\Tt{}{\nwbackslash}nwbreakcodespace\nwendquote} controls how much white space to
leave.
\nwenddocs{}\nwbegincode{71}\sublabel{NW4Nr7fb-3olr1Q-F}\nwmargintag{{\nwtagstyle{}\subpageref{NW4Nr7fb-3olr1Q-F}}}\moddef{noweb.sty~{\nwtagstyle{}\subpageref{NW4Nr7fb-3olr1Q-1}}}\plusendmoddef\nwstartdeflinemarkup\nwprevnextdefs{NW4Nr7fb-3olr1Q-E}{NW4Nr7fb-3olr1Q-G}\nwenddeflinemarkup
\\def\\@nwsemifilbreak#1\{\\vskip0pt plus#1\\penalty-200\\vskip0pt plus -#1\}
\\newdimen\\nwbreakcodespace
\\nwbreakcodespace=0.2in  % by default, leave no more than this on a page
\\def\\nwopt@breakcode\{%
  \\def\\nwdocspar\{\\@nwsemifilbreak\{0.2in\}\}%
  \\def\\nwendcode\{\\endtrivlist\\endgroup\} % ditches filbreak
\}
\nwendcode{}\nwbegindocs{72}\nwdocspar
The page-breaking strategy implies ragged bottom pages, so we should
turn it on in general (this is relevant for the {\tt report} style):
\nwenddocs{}\nwbegincode{73}\sublabel{NW4Nr7fb-3olr1Q-G}\nwmargintag{{\nwtagstyle{}\subpageref{NW4Nr7fb-3olr1Q-G}}}\moddef{noweb.sty~{\nwtagstyle{}\subpageref{NW4Nr7fb-3olr1Q-1}}}\plusendmoddef\nwstartdeflinemarkup\nwprevnextdefs{NW4Nr7fb-3olr1Q-F}{NW4Nr7fb-3olr1Q-H}\nwenddeflinemarkup
\\raggedbottom
\nwendcode{}\nwbegincode{74}\sublabel{NW4Nr7fb-38jgpJ-3}\nwmargintag{{\nwtagstyle{}\subpageref{NW4Nr7fb-38jgpJ-3}}}\moddef{nwmac.tex~{\nwtagstyle{}\subpageref{NW4Nr7fb-38jgpJ-1}}}\plusendmoddef\nwstartdeflinemarkup\nwprevnextdefs{NW4Nr7fb-38jgpJ-2}{NW4Nr7fb-38jgpJ-4}\nwenddeflinemarkup
\\def\\nwdocspar\{\\par\\semifilbreak\}
\nwendcode{}\nwbegindocs{75}\nwdocspar

{\tt noweave} doesn't bracket quoted code with {\Tt{}{\nwbackslash}code\nwendquote} and {\Tt{}{\nwbackslash}edoc\nwendquote} any more.
It probably should do something nifty, just to make {\TeX} hackers happy, but it doesn't.
\nwenddocs{}\nwbegincode{76}\sublabel{NW4Nr7fb-3olr1Q-H}\nwmargintag{{\nwtagstyle{}\subpageref{NW4Nr7fb-3olr1Q-H}}}\moddef{noweb.sty~{\nwtagstyle{}\subpageref{NW4Nr7fb-3olr1Q-1}}}\plusendmoddef\nwstartdeflinemarkup\nwprevnextdefs{NW4Nr7fb-3olr1Q-G}{NW4Nr7fb-3olr1Q-I}\nwenddeflinemarkup
\\def\\code\{\\leavevmode\\begingroup\\setupcode\\@vobeyspaces\\obeylines\}
\\let\\edoc=\\endgroup
\nwendcode{}\nwbegindocs{77}\nwdocspar
\subsection{The {\tt noweb} page style}
Headers contain file name, date, and page number.
{\tt noweave} emits {\Tt{}{\nwbackslash}nwfilename{\nwlbrace}\nwendquote}{\em name}{\Tt{}{\nwrbrace}\nwendquote} for each new file.
In the {\tt noweb} page style, new files cause page breaks;
otherwise they are ignored.
\nwenddocs{}\nwbegincode{78}\sublabel{NW4Nr7fb-3olr1Q-I}\nwmargintag{{\nwtagstyle{}\subpageref{NW4Nr7fb-3olr1Q-I}}}\moddef{noweb.sty~{\nwtagstyle{}\subpageref{NW4Nr7fb-3olr1Q-1}}}\plusendmoddef\nwstartdeflinemarkup\nwprevnextdefs{NW4Nr7fb-3olr1Q-H}{NW4Nr7fb-3olr1Q-J}\nwenddeflinemarkup
\\newdimen\\@original@textwidth
\\def\\ps@noweb\{%
  \\@original@textwidth=\\textwidth
  \\let\\@mkboth\\@gobbletwo
  \\def\\@oddfoot\{\}\\def\\@evenfoot\{\}%       No feet.
  \\if@twoside         % If two-sided printing.
    \\def\\@evenhead\{\\hbox to \\@original@textwidth\{%
           \\Rm \\thepage\\qquad\{\\Tt\\leftmark\}\\hfil\\today\}\}%        Left heading.
    \\def\\@oddhead\{\\hbox to \\@original@textwidth\{%
           \\Rm \\today\\hfil\{\\Tt\\leftmark\}\\qquad\\thepage\}\}% Right heading.
  \\else               % If one-sided printing.
    \\def\\@oddhead\{\\hbox to \\@original@textwidth\{%
           \\Rm \\today\\hfil\{\\Tt\\leftmark\}\\qquad\\thepage\}\}% Right heading.
    \\let\\@evenhead\\@oddhead
  \\fi
  \\let\\chaptermark\\@gobble
  \\let\\sectionmark\\@gobble
  \\let\\subsectionmark\\@gobble
  \\let\\subsubsectionmark\\@gobble
  \\let\\paragraphmark\\@gobble
  \\let\\subparagraphmark\\@gobble
  \\def\\nwfilename\{\\begingroup\\let\\do\\@makeother\\dospecials
                \\catcode`\\\{=1 \\catcode`\\\}=2 \\nw@filename\}
  \\def\\nw@filename##1\{\\endgroup\\markboth\{##1\}\{##1\}\\let\\nw@filename=\\nw@laterfilename\}%
\}
\\def\\nw@laterfilename#1\{\\endgroup\\clearpage \\markboth\{#1\}\{#1\}\}
\\let\\nwfilename=\\@gobble
\nwendcode{}\nwbegindocs{79}\nwdocspar
\subsection{Chunk cross-reference}
{\Tt{}{\nwbackslash}nwalsodefined\nwendquote}, {\Tt{}{\nwbackslash}nwused\nwendquote}, and {\Tt{}{\nwbackslash}nwnotused\nwendquote} are emitted by the {\tt
noweb} cross-referencers. (What arguments?)
If unused chunks are output chunks, a filter can slip in
{\Tt{}{\nwbackslash}let{\nwbackslash}nwnotused{\nwbackslash}nwoutput\nwendquote}.
The style uses {\Tt{}{\nwbackslash}nwcodecomment\nwendquote} for all annotations that follow code
chunks.
Fiddling with it can change the appearance of the output.
Note that {\Tt{}{\nwbackslash}nwcodecomment\nwendquote} is used after {\Tt{}{\nwbackslash}nwbegincode\nwendquote}, with
{\Tt{}{\nwbackslash}obeylines\nwendquote} in efect.  Since linebreaking can occur here, we need
to change the {\Tt{}{\nwbackslash}interlinepenalty\nwendquote}.
A little vertical space ({\Tt{}{\nwbackslash}nwcodecommentsep\nwendquote}\stylehook) appears before the first
comment.

We firkled with {\Tt{}{\nwbackslash}rightskip\nwendquote} in {\Tt{}{\nwbackslash}nwbegincode\nwendquote} above; now we want to
reset it so that paragraphs are the normal width ({\Tt{}{\nwbackslash}textwidth\nwendquote},
possibly less {\Tt{}{\nwbackslash}codemargin\nwendquote}) and set ragged right.  This is done as
usuall by making {\Tt{}{\nwbackslash}rightskip\nwendquote} naturally zero but stretchable.
\nwenddocs{}\nwbegincode{80}\sublabel{NW4Nr7fb-3olr1Q-J}\nwmargintag{{\nwtagstyle{}\subpageref{NW4Nr7fb-3olr1Q-J}}}\moddef{noweb.sty~{\nwtagstyle{}\subpageref{NW4Nr7fb-3olr1Q-1}}}\plusendmoddef\nwstartdeflinemarkup\nwprevnextdefs{NW4Nr7fb-3olr1Q-I}{NW4Nr7fb-3olr1Q-K}\nwenddeflinemarkup
\\def\\nwcodecomment#1\{\\@@par\\penalty\\nwcodepenalty
    \LA{}add \code{}{\nwbackslash}nwcodecommentsep\edoc{} if this is the first \code{}{\nwbackslash}nwcodecomment\edoc{}~{\nwtagstyle{}\subpageref{NW4Nr7fb-7q7UY-1}}\RA{}%
    \\hspace\{-\\codemargin\}\{%
        \\rightskip=0pt plus1in
        \\interlinepenalty\\nwcodepenalty
        \\let\\\\\\relax\\footnotesize\\Rm #1\\@@par\\penalty\\nwcodepenalty\}\}
\nwendcode{}\nwbegindocs{81}\nwdocspar
This stuff is used at the end of a chunk.
\nwenddocs{}\nwbegincode{82}\sublabel{NW4Nr7fb-3olr1Q-K}\nwmargintag{{\nwtagstyle{}\subpageref{NW4Nr7fb-3olr1Q-K}}}\moddef{noweb.sty~{\nwtagstyle{}\subpageref{NW4Nr7fb-3olr1Q-1}}}\plusendmoddef\nwstartdeflinemarkup\nwprevnextdefs{NW4Nr7fb-3olr1Q-J}{NW4Nr7fb-3olr1Q-L}\nwenddeflinemarkup
\\def\\@nwalsodefined#1\{\\nwcodecomment\{\\@nwlangdepdef\\ \\nwpageprep\\ \\@pagesl\{#1\}.\}\}
\\def\\@nwused#1\{\\nwcodecomment\{\\@nwlangdepcud\\ \\nwpageprep\\ \\@pagesl\{#1\}.\}\}
\\def\\@nwnotused#1\{\\nwcodecomment\{\\@nwlangdeprtc.\}\}
\\def\\nwoutput#1\{\\nwcodecomment\{\\@nwlangdepcwf\\ \{\\Tt \\@stripstar#1*\\stripped\}.\}\}
\\def\\@stripstar#1*#2\\stripped\{#1\}
\nwendcode{}\nwbegincode{83}\sublabel{NW4Nr7fb-7q7UY-1}\nwmargintag{{\nwtagstyle{}\subpageref{NW4Nr7fb-7q7UY-1}}}\moddef{add \code{}{\nwbackslash}nwcodecommentsep\edoc{} if this is the first \code{}{\nwbackslash}nwcodecomment\edoc{}~{\nwtagstyle{}\subpageref{NW4Nr7fb-7q7UY-1}}}\endmoddef\nwstartdeflinemarkup\nwusesondefline{\\{NW4Nr7fb-3olr1Q-J}}\nwenddeflinemarkup
\\if@firstnwcodecomment
  \\vskip\\nwcodecommentsep\\penalty\\nwcodepenalty\\@firstnwcodecommentfalse
\\fi
\nwused{\\{NW4Nr7fb-3olr1Q-J}}\nwendcode{}\nwbegindocs{84}\nwdocspar
This stuff on the definition line.
Note the hooks\stylehook{} for pointer styles.
\nwenddocs{}\nwbegincode{85}\sublabel{NW4Nr7fb-3olr1Q-L}\nwmargintag{{\nwtagstyle{}\subpageref{NW4Nr7fb-3olr1Q-L}}}\moddef{noweb.sty~{\nwtagstyle{}\subpageref{NW4Nr7fb-3olr1Q-1}}}\plusendmoddef\nwstartdeflinemarkup\nwprevnextdefs{NW4Nr7fb-3olr1Q-K}{NW4Nr7fb-3olr1Q-M}\nwenddeflinemarkup
\\newcommand\{\\nwprevdefptr\}[1]\{%
  \\mbox\{$\\mathord\{\\triangleleft\}\\,\\mathord\{\\mbox\{\\subpageref\{#1\}\}\}$\}\}
\\newcommand\{\\nwnextdefptr\}[1]\{%
  \\mbox\{$\\mathord\{\\mbox\{\\subpageref\{#1\}\}\}\\,\\mathord\{\\triangleright\}$\}\}

\\newcommand\{\\@nwprevnextdefs\}[2]\{%
  \{\\nwtagstyle
  \\ifx\\relax#1\\else ~~\\nwprevdefptr\{#1\}\\fi
  \\ifx\\relax#2\\else ~~\\nwnextdefptr\{#2\}\\fi\}\}
\\newcommand\{\\@nwusesondefline\}[1]\{\{\\nwtagstyle~~(\\@pagenumsl\{#1\})\}\}
\\newcommand\{\\@nwstartdeflinemarkup\}\{\\nobreak\\hskip 1.5em plus 1fill\\nobreak\}
\\newcommand\{\\@nwenddeflinemarkup\}\{\\nobreak\\hskip \\nwdefspace minus\\nwdefspace\\nobreak\}
\nwendcode{}\nwbegincode{86}\sublabel{NW4Nr7fb-38jgpJ-4}\nwmargintag{{\nwtagstyle{}\subpageref{NW4Nr7fb-38jgpJ-4}}}\moddef{nwmac.tex~{\nwtagstyle{}\subpageref{NW4Nr7fb-38jgpJ-1}}}\plusendmoddef\nwstartdeflinemarkup\nwprevnextdefs{NW4Nr7fb-38jgpJ-3}{NW4Nr7fb-38jgpJ-5}\nwenddeflinemarkup
\\def\\nwstartdeflinemarkup\{\\nobreak\\hskip 1.5em plus 1fill\\nobreak\}
\\def\\nwenddeflinemarkup\{\\nobreak\\hskip \\nwdefspace minus\\nwdefspace\\nobreak\}
\nwendcode{}\nwbegindocs{87}\nwdocspar
And here are the options we use to choose one, the other, or neither.
\nwenddocs{}\nwbegincode{88}\sublabel{NW4Nr7fb-3olr1Q-M}\nwmargintag{{\nwtagstyle{}\subpageref{NW4Nr7fb-3olr1Q-M}}}\moddef{noweb.sty~{\nwtagstyle{}\subpageref{NW4Nr7fb-3olr1Q-1}}}\plusendmoddef\nwstartdeflinemarkup\nwprevnextdefs{NW4Nr7fb-3olr1Q-L}{NW4Nr7fb-3olr1Q-N}\nwenddeflinemarkup
\\def\\nwopt@longxref\{%
  \\let\\nwalsodefined\\@nwalsodefined
  \\let\\nwused\\@nwused
  \\let\\nwnotused\\@nwnotused
  \\let\\nwprevnextdefs\\@gobbletwo
  \\let\\nwusesondefline\\@gobble
  \\let\\nwstartdeflinemarkup\\relax
  \\let\\nwenddeflinemarkup\\relax
\}
\\def\\nwopt@shortxref\{%
  \\let\\nwalsodefined\\@gobble
  \\let\\nwused\\@gobble
  \\let\\nwnotused\\@gobble
  \\let\\nwprevnextdefs\\@nwprevnextdefs
  \\let\\nwusesondefline\\@nwusesondefline
  \\let\\nwstartdeflinemarkup\\@nwstartdeflinemarkup
  \\let\\nwenddeflinemarkup\\@nwenddeflinemarkup
\}
\\def\\nwopt@noxref\{%
  \\let\\nwalsodefined\\@gobble
  \\let\\nwused\\@gobble
  \\let\\nwnotused\\@gobble
  \\let\\nwprevnextdefs\\@gobbletwo
  \\let\\nwusesondefline\\@gobble
  \\let\\nwstartdeflinemarkup\\relax
  \\let\\nwenddeflinemarkup\\relax
\}
\\nwopt@shortxref % to hell with backward compatibility!
\nwendcode{}\nwbegindocs{89}\iffalse
\nwenddocs{}\nwbegincode{90}\sublabel{NW4Nr7fb-3RhSlV-4}\nwmargintag{{\nwtagstyle{}\subpageref{NW4Nr7fb-3RhSlV-4}}}\moddef{man page: \code{}{\nwbackslash}noweboptions\edoc{}~{\nwtagstyle{}\subpageref{NW4Nr7fb-3RhSlV-1}}}\plusendmoddef\nwstartdeflinemarkup\nwprevnextdefs{NW4Nr7fb-3RhSlV-3}{NW4Nr7fb-3RhSlV-5}\nwenddeflinemarkup
.TP
.B longxref, shortxref, noxref
Choose a style for chunk cross-reference.
Long style uses  small paragraphs after each chunk, as in Knuth.
Short style uses symbols on the definition line, as in Hanson.
.B noxref
provides no chunk cross-reference. 
Defaults to 
.B shortxref.
\nwendcode{}\nwbegindocs{91}\fi
\nwenddocs{}\nwbegincode{92}\sublabel{NW4Nr7fb-3olr1Q-N}\nwmargintag{{\nwtagstyle{}\subpageref{NW4Nr7fb-3olr1Q-N}}}\moddef{noweb.sty~{\nwtagstyle{}\subpageref{NW4Nr7fb-3olr1Q-1}}}\plusendmoddef\nwstartdeflinemarkup\nwprevnextdefs{NW4Nr7fb-3olr1Q-M}{NW4Nr7fb-3olr1Q-O}\nwenddeflinemarkup
\\newskip\\nwcodecommentsep \\nwcodecommentsep=3pt plus 1pt minus 1pt
\\newif\\if@firstnwcodecomment\\@firstnwcodecommenttrue
\nwendcode{}\nwbegindocs{93}\nwdocspar
\subsection{Page ranges}
The goal is to combine sub-page numbers in a way that makes sense.
Multiple sub-pages of one page become that page, and individual
pages are combined into ranges.
(A range may be only one page.)
\nwenddocs{}\nwbegincode{94}\sublabel{NW4Nr7fb-3olr1Q-O}\nwmargintag{{\nwtagstyle{}\subpageref{NW4Nr7fb-3olr1Q-O}}}\moddef{noweb.sty~{\nwtagstyle{}\subpageref{NW4Nr7fb-3olr1Q-1}}}\plusendmoddef\nwstartdeflinemarkup\nwprevnextdefs{NW4Nr7fb-3olr1Q-N}{NW4Nr7fb-3olr1Q-P}\nwenddeflinemarkup
\\newcount\\@nwlopage\\newcount\\@nwhipage  % range lo..hi-1
\\newcount\\@nwlosub              % subpage of lo
\\newcount\\@nwhisub              % subpage of hi
\\def\\@nwfirstpage#1#2#3\{% subpage page xref-tag
  \\@nwlopage=#2 \\@nwlosub=#1
  \\def\\@nwloxreftag\{#3\}%
  \\advance\\@nwpagecount by \\@ne
  \LA{}$\mbox{\code{}{\nwbackslash}@nwhipage\edoc{}} := \mbox{\code{}{\nwbackslash}@nwlopage\edoc{}}+1$~{\nwtagstyle{}\subpageref{NW4Nr7fb-XOf88-1}}\RA{}\}
\\def\\@nwnextpage#1#2#3\{% subpage page xref-tag
  \\ifnum\\@nwhipage=#2 
    \\advance\\@nwhipage by \\@ne 
    \\advance\\@nwpagecount by \\@ne
    \\@nwhisub=#1 
    \\def\\@nwhixreftag\{#3\}\\else
  \\ifnum#2<\\@nwlopage \LA{}new range starting with \code{}{\#}2\edoc{}~{\nwtagstyle{}\subpageref{NW4Nr7fb-rhQSd-1}}\RA{}\\else
  \\ifnum#2>\\@nwhipage \LA{}new range starting with \code{}{\#}2\edoc{}~{\nwtagstyle{}\subpageref{NW4Nr7fb-rhQSd-1}}\RA{}\\else
    \\@nwlosub=0 \\@nwhisub=0
  \\fi\\fi\\fi
  \}
\nwendcode{}\nwbegincode{95}\sublabel{NW4Nr7fb-rhQSd-1}\nwmargintag{{\nwtagstyle{}\subpageref{NW4Nr7fb-rhQSd-1}}}\moddef{new range starting with \code{}{\#}2\edoc{}~{\nwtagstyle{}\subpageref{NW4Nr7fb-rhQSd-1}}}\endmoddef\nwstartdeflinemarkup\nwusesondefline{\\{NW4Nr7fb-3olr1Q-O}}\nwenddeflinemarkup
\LA{}add range to range list~{\nwtagstyle{}\subpageref{NW4Nr7fb-1VTyPN-1}}\RA{}\\@nwfirstpage\{#1\}\{#2\}\{#3\}
\nwused{\\{NW4Nr7fb-3olr1Q-O}}\nwendcode{}\nwbegincode{96}\sublabel{NW4Nr7fb-XOf88-1}\nwmargintag{{\nwtagstyle{}\subpageref{NW4Nr7fb-XOf88-1}}}\moddef{$\mbox{\code{}{\nwbackslash}@nwhipage\edoc{}} := \mbox{\code{}{\nwbackslash}@nwlopage\edoc{}}+1$~{\nwtagstyle{}\subpageref{NW4Nr7fb-XOf88-1}}}\endmoddef\nwstartdeflinemarkup\nwusesondefline{\\{NW4Nr7fb-3olr1Q-O}}\nwenddeflinemarkup
\\@nwhipage=\\@nwlopage\\advance\\@nwhipage by \\@ne 
\nwused{\\{NW4Nr7fb-3olr1Q-O}}\nwendcode{}\nwbegincode{97}\sublabel{NW4Nr7fb-1VTyPN-1}\nwmargintag{{\nwtagstyle{}\subpageref{NW4Nr7fb-1VTyPN-1}}}\moddef{add range to range list~{\nwtagstyle{}\subpageref{NW4Nr7fb-1VTyPN-1}}}\endmoddef\nwstartdeflinemarkup\nwusesondefline{\\{NW4Nr7fb-rhQSd-1}\\{NW4Nr7fb-3olr1Q-R}\\{NW4Nr7fb-3olr1Q-S}}\nwenddeflinemarkup
\LA{}set \code{}{\nwbackslash}@tempa\edoc{} to page range(s), marked with \code{}{\nwbackslash}{\nwbackslash}\edoc{}~{\nwtagstyle{}\subpageref{NW4Nr7fb-PvOgT-1}}\RA{}%
\\edef\\@tempa\{\\noexpand\\nwix@cons\\noexpand\\nw@pages\{\\@tempa\}\}\\@tempa
\nwused{\\{NW4Nr7fb-rhQSd-1}\\{NW4Nr7fb-3olr1Q-R}\\{NW4Nr7fb-3olr1Q-S}}\nwendcode{}\nwbegincode{98}\sublabel{NW4Nr7fb-PvOgT-1}\nwmargintag{{\nwtagstyle{}\subpageref{NW4Nr7fb-PvOgT-1}}}\moddef{set \code{}{\nwbackslash}@tempa\edoc{} to page range(s), marked with \code{}{\nwbackslash}{\nwbackslash}\edoc{}~{\nwtagstyle{}\subpageref{NW4Nr7fb-PvOgT-1}}}\endmoddef\nwstartdeflinemarkup\nwusesondefline{\\{NW4Nr7fb-1VTyPN-1}}\nwenddeflinemarkup
\\advance\\@nwhipage by \\m@ne
\\ifnum\\@nwhipage=\\@nwlopage
     \\edef\\@tempa\{\\noexpand\\noexpand\\noexpand\\\\%
                       \{\{\\nwthepagenum\{\\number\\@nwlosub\}\{\\number\\@nwlopage\}\}%
                        \{\\@nwloxreftag\}\}\}%
\\else
  \\count@=\\@nwhipage \\advance\\count@ by \\m@ne
  \\ifnum\\count@=\\@nwlopage % consecutive pages
      \\edef\\@tempa\{\\noexpand\\noexpand\\noexpand\\\\%
                       \{\{\\nwthepagenum\{\\number\\@nwlosub\}\{\\number\\@nwlopage\}\}%
                        \{\\@nwloxreftag\}\}%
                    \\noexpand\\noexpand\\noexpand\\\\%
                       \{\{\\nwthepagenum\{\\number\\@nwhisub\}\{\\number\\@nwhipage\}\}
                        \{\\@nwhixreftag\}\}\}%
  \\else \LA{}use rules from Chicago style manual~{\nwtagstyle{}\subpageref{NW4Nr7fb-8i24d-1}}\RA{}%
  \\fi
\\fi
\nwused{\\{NW4Nr7fb-1VTyPN-1}}\nwendcode{}\nwbegincode{99}\sublabel{NW4Nr7fb-8i24d-1}\nwmargintag{{\nwtagstyle{}\subpageref{NW4Nr7fb-8i24d-1}}}\moddef{use rules from Chicago style manual~{\nwtagstyle{}\subpageref{NW4Nr7fb-8i24d-1}}}\endmoddef\nwstartdeflinemarkup\nwusesondefline{\\{NW4Nr7fb-PvOgT-1}}\nwenddeflinemarkup
\\ifnum\\@nwlopage<110 \LA{}normal range~{\nwtagstyle{}\subpageref{NW4Nr7fb-3rTQ1n-1}}\RA{}\\else
  \\count@=\\@nwlopage \\divide\\count@ by 100 \\multiply\\count@ by 100
  \\ifnum\\count@=\\@nwlopage \LA{}normal range~{\nwtagstyle{}\subpageref{NW4Nr7fb-3rTQ1n-1}}\RA{}\\else
    \\count@=\\@nwlopage \\divide\\count@ by 100
    \\@nwpagetemp=\\@nwhipage \\divide\\@nwpagetemp by 100
    \\ifnum\\count@=\\@nwpagetemp %  lo--least 2 digits of hi
      \\multiply\\@nwpagetemp by 100
      \\advance \\@nwhipage by -\\@nwpagetemp
      \LA{}normal range~{\nwtagstyle{}\subpageref{NW4Nr7fb-3rTQ1n-1}}\RA{}%
    \\else \LA{}normal range~{\nwtagstyle{}\subpageref{NW4Nr7fb-3rTQ1n-1}}\RA{}%
    \\fi
  \\fi
\\fi
\nwused{\\{NW4Nr7fb-PvOgT-1}}\nwendcode{}\nwbegincode{100}\sublabel{NW4Nr7fb-3rTQ1n-1}\nwmargintag{{\nwtagstyle{}\subpageref{NW4Nr7fb-3rTQ1n-1}}}\moddef{normal range~{\nwtagstyle{}\subpageref{NW4Nr7fb-3rTQ1n-1}}}\endmoddef\nwstartdeflinemarkup\nwusesondefline{\\{NW4Nr7fb-8i24d-1}}\nwenddeflinemarkup
\\edef\\@tempa\{\\noexpand\\noexpand\\noexpand\\\\\{\{\\number\\@nwlopage--\\number\\@nwhipage\}\{\}\}\}
\nwused{\\{NW4Nr7fb-8i24d-1}}\nwendcode{}\nwbegincode{101}\sublabel{NW4Nr7fb-3olr1Q-P}\nwmargintag{{\nwtagstyle{}\subpageref{NW4Nr7fb-3olr1Q-P}}}\moddef{noweb.sty~{\nwtagstyle{}\subpageref{NW4Nr7fb-3olr1Q-1}}}\plusendmoddef\nwstartdeflinemarkup\nwprevnextdefs{NW4Nr7fb-3olr1Q-O}{NW4Nr7fb-3olr1Q-Q}\nwenddeflinemarkup
\\newcount\\@nwpagetemp
\nwendcode{}\nwbegindocs{102}\nwdocspar
The sequence {\Tt{}{\nwbackslash}@pagesl\nwendquote} makes a range of pages from a list of labels.
{\Tt{}{\nwbackslash}subpages\nwendquote} works from a list of {\Tt{}{\nwlbrace}{\nwlbrace}subpage{\nwrbrace}{\nwlbrace}page{\nwrbrace}{\nwrbrace}\nwendquote}.
\nwenddocs{}\nwbegincode{103}\sublabel{NW4Nr7fb-3olr1Q-Q}\nwmargintag{{\nwtagstyle{}\subpageref{NW4Nr7fb-3olr1Q-Q}}}\moddef{noweb.sty~{\nwtagstyle{}\subpageref{NW4Nr7fb-3olr1Q-1}}}\plusendmoddef\nwstartdeflinemarkup\nwprevnextdefs{NW4Nr7fb-3olr1Q-P}{NW4Nr7fb-3olr1Q-R}\nwenddeflinemarkup
\\newcount\\@nwpagecount
\\def\\@nwfirstpagel#1\{% label
  \\@ifundefined\{r@#1\}\{\LA{}warn of undefined reference to \code{}{\#}1\edoc{} and add page ??~{\nwtagstyle{}\subpageref{NW4Nr7fb-1dYqBd-1}}\RA{}\}\{%
    \\edef\\@tempa\{\\noexpand\\@nwfirstpage\\subpagepair\{#1\}\{#1\}\}\\@tempa\}\}
\\def\\@nwnextpagel#1\{% label
  \\@ifundefined\{r@#1\}\{\LA{}warn of undefined reference to \code{}{\#}1\edoc{} and add page ??~{\nwtagstyle{}\subpageref{NW4Nr7fb-1dYqBd-1}}\RA{}\}\{%
    \\edef\\@tempa\{\\noexpand\\@nwnextpage\\subpagepair\{#1\}\{#1\}\}\\@tempa\}\}
\nwendcode{}\nwbegincode{104}\sublabel{NW4Nr7fb-3olr1Q-R}\nwmargintag{{\nwtagstyle{}\subpageref{NW4Nr7fb-3olr1Q-R}}}\moddef{noweb.sty~{\nwtagstyle{}\subpageref{NW4Nr7fb-3olr1Q-1}}}\plusendmoddef\nwstartdeflinemarkup\nwprevnextdefs{NW4Nr7fb-3olr1Q-Q}{NW4Nr7fb-3olr1Q-S}\nwenddeflinemarkup
\\def\\@pagesl#1\{%  list of labels
  \\gdef\\nw@pages\{\}\\@nwpagecount=0
  \\def\\\\##1\{\\@nwfirstpagel\{##1\}\\let\\\\=\\@nwnextpagel\}#1%
  \LA{}add range to range list~{\nwtagstyle{}\subpageref{NW4Nr7fb-1VTyPN-1}}\RA{}\\def\\\\##1\{\\@nwhyperpagenum##1\}%
  \\ifnum\\@nwpagecount=1 \\nwpageword \\else \\nwpagesword\\fi~\\commafy\{\\nw@pages\}\}
\\def\\@nwhyperpagenum#1#2\{\\nwhyperreference\{#2\}\{#1\}\}

\\def\\@pagenumsl#1\{%  list of labels -- doesn't include word `pages', commas, or `and'
  \\gdef\\nw@pages\{\}\\@nwpagecount=0
  \\def\\\\##1\{\\@nwfirstpagel\{##1\}\\let\\\\=\\@nwnextpagel\}#1%
  \LA{}add range to range list~{\nwtagstyle{}\subpageref{NW4Nr7fb-1VTyPN-1}}\RA{}%
  \\def\\\\##1\{\\@nwhyperpagenum##1\\let\\\\=\\@nwpagenumslrest\}\\nw@pages\}
\\def\\@nwpagenumslrest#1\{~\\@nwhyperpagenum#1\}
\nwendcode{}\nwbegincode{105}\sublabel{NW4Nr7fb-3olr1Q-S}\nwmargintag{{\nwtagstyle{}\subpageref{NW4Nr7fb-3olr1Q-S}}}\moddef{noweb.sty~{\nwtagstyle{}\subpageref{NW4Nr7fb-3olr1Q-1}}}\plusendmoddef\nwstartdeflinemarkup\nwprevnextdefs{NW4Nr7fb-3olr1Q-R}{NW4Nr7fb-3olr1Q-T}\nwenddeflinemarkup
\\def\\subpages#1\{% list of \{\{subpage\}\{page\}\}
  \\gdef\\nw@pages\{\}\\@nwpagecount=0
  \\def\\\\##1\{\\edef\\@tempa\{\\noexpand\\@nwfirstpage##1\{\}\}\\@tempa
            \\def\\\\####1\{\\edef\\@tempa\{\\noexpand\\@nwnextpage####1\}\\@tempa\}\}#1%
  \LA{}add range to range list~{\nwtagstyle{}\subpageref{NW4Nr7fb-1VTyPN-1}}\RA{}\\def\\\\##1\{\\@firstoftwo##1\}%
  \\ifnum\\@nwpagecount=1 \\nwpageword \\else \\nwpagesword\\fi~\\commafy\{\\nw@pages\}\}
\\def\\@nwaddrange\{\LA{}add range to range list~{\nwtagstyle{}\subpageref{NW4Nr7fb-1VTyPN-1}}\RA{}\}
\nwendcode{}\nwbegindocs{106}\nwdocspar
{\Tt{}{\nwbackslash}nwpageword\nwendquote}, {\Tt{}{\nwbackslash}nwpagesword\nwendquote}, and {\Tt{}{\nwbackslash}nwpageprep\nwendquote} let you change
the wording of the cross-reference  information.
\nwenddocs{}\nwbegincode{107}\sublabel{NW4Nr7fb-3olr1Q-T}\nwmargintag{{\nwtagstyle{}\subpageref{NW4Nr7fb-3olr1Q-T}}}\moddef{noweb.sty~{\nwtagstyle{}\subpageref{NW4Nr7fb-3olr1Q-1}}}\plusendmoddef\nwstartdeflinemarkup\nwprevnextdefs{NW4Nr7fb-3olr1Q-S}{NW4Nr7fb-3olr1Q-U}\nwenddeflinemarkup
\\def\\nwpageword\{\\@nwlangdepchk\}  % chunk, was page
\\def\\nwpagesword\{\\@nwlangdepchks\}  % chunk, was page
\\def\\nwpageprep\{\\@nwlangdepin\}     % in, was on
\nwendcode{}\nwbegincode{108}\sublabel{NW4Nr7fb-1dYqBd-1}\nwmargintag{{\nwtagstyle{}\subpageref{NW4Nr7fb-1dYqBd-1}}}\moddef{warn of undefined reference to \code{}{\#}1\edoc{} and add page ??~{\nwtagstyle{}\subpageref{NW4Nr7fb-1dYqBd-1}}}\endmoddef\nwstartdeflinemarkup\nwusesondefline{\\{NW4Nr7fb-3olr1Q-Q}}\nwenddeflinemarkup
\LA{}warn of undefined reference to \code{}{\#}1\edoc{}~{\nwtagstyle{}\subpageref{NW4Nr7fb-4CAbV-1}}\RA{}%
\\nwix@cons\\nw@pages\{\\\\\{\\bf ??\}\}
\nwused{\\{NW4Nr7fb-3olr1Q-Q}}\nwendcode{}\nwbegindocs{109}\nwdocspar
\subsection{Sub-page references}

This is the wonderful code that Dave Love provided to make page
references like 7a, 7b, and so on.

This code
provides a mechanism for defining `page
sub-references' using {\Tt{}{\nwbackslash}sublabel{\nwlbrace}foo{\nwrbrace}\nwendquote} referenced with
{\Tt{}{\nwbackslash}subpageref{\nwlbrace}foo{\nwrbrace}\nwendquote}.  Sub-references will be numbered like these real
examples: \subpageref{ref:foo}, \subpageref{ref:bar},
\subpageref{ref:baz}\sublabel{ref:foo}\sublabel{ref:bar}\sublabel{ref:baz}
etc.\ unless there is only one on the page, in which case the letter
will be dropped like this: \subpageref{ref:fred}.

To be able to use {\Tt{}{\nwbackslash}subpageref\nwendquote} we must define the label with
{\Tt{}{\nwbackslash}sublabel\nwendquote}, used like label.  (Using
{\Tt{}{\nwbackslash}ref\nwendquote} with a label defined by {\Tt{}{\nwbackslash}sublabel\nwendquote} will
produce the sub-reference number, by the way, and {\Tt{}{\nwbackslash}pageref\nwendquote}
works as expected.)  Note that
{\Tt{}{\nwbackslash}subpageref\nwendquote} is robust and {\Tt{}{\nwbackslash}ref\nwendquote} and {\Tt{}{\nwbackslash}pageref\nwendquote} are redefined to be
robust also, as they will be in future \LaTeX{} releases.
Incidentally, these expand to the relevant text plus {\Tt{}{\nwbackslash}null\nwendquote}---you
might want to strip this off, e.g.\ for sorting lists.


There are various ways we could attack this task (which is made
non-trivial by the well-known asynchrony of (La)\TeX's output
routine), but
they all must depend on hacks in the {\Tt{}.aux\nwendquote} file or a similar one.
Joachim Schrod's {\Tt{}fnpag.sty\nwendquote} does the same sort of thing differently
to this \LaTeX-specific approach.  See {\Tt{}latex.tex\nwendquote} for enlightenment
on the cross-referencing mechanism and the \LaTeX{} internals used
below.  [DL: The internals change in \LaTeX2e compared with
\LaTeX~2.09.  The code here still works, though.]
\nwenddocs{}\nwbegindocs{110}\nwdocspar
The new-style {\LaTeX} page-reference macros all work the same way: 
if the thing is undefined, barf.  Otherwise, do the specified thing.
We need to handle the fact that the expansion of the label may be two
items or five items, depending on whether hypertext is used.
Since we're only ever interested in the first two items, we use a
hack---the ``do the specified thing'' must be defined as
\mbox{{\Tt{}{\nwbackslash}def{\nwbackslash}dome{\#}1{\#}2{\#}3{\nwbackslash}{\nwbackslash}{\nwlbrace}...{\nwrbrace}\nwendquote}} where {\Tt{}...\nwendquote} uses only the first two parameters.
\nwenddocs{}\nwbegincode{111}\sublabel{NW4Nr7fb-3olr1Q-U}\nwmargintag{{\nwtagstyle{}\subpageref{NW4Nr7fb-3olr1Q-U}}}\moddef{noweb.sty~{\nwtagstyle{}\subpageref{NW4Nr7fb-3olr1Q-1}}}\plusendmoddef\nwstartdeflinemarkup\nwprevnextdefs{NW4Nr7fb-3olr1Q-T}{NW4Nr7fb-3olr1Q-V}\nwenddeflinemarkup
\\newcommand\\nw@genericref[2]\{% what to do, name of ref
  \\expandafter\\nw@g@nericref\\csname r@#2\\endcsname#1\{#2\}\}
\\newcommand\\nw@g@nericref[3]\{% control sequence, what to do, name
  \\ifx#1\\relax
    \\ref\{#3\}% trigger the standard `undefined ref' mechanisms
  \\else
    \\expandafter#2#1.\\\\%
  \\fi\}
\nwendcode{}\nwbegindocs{112}Much of what we want can be done by pulling out the first, second,
or first and second elements of a ref.
\nwenddocs{}\nwbegincode{113}\sublabel{NW4Nr7fb-3olr1Q-V}\nwmargintag{{\nwtagstyle{}\subpageref{NW4Nr7fb-3olr1Q-V}}}\moddef{noweb.sty~{\nwtagstyle{}\subpageref{NW4Nr7fb-3olr1Q-1}}}\plusendmoddef\nwstartdeflinemarkup\nwprevnextdefs{NW4Nr7fb-3olr1Q-U}{NW4Nr7fb-3olr1Q-W}\nwenddeflinemarkup
\\def\\nw@selectone#1#2#3\\\\\{#1\}
\\def\\nw@selecttwo#1#2#3\\\\\{#2\}
\\def\\nw@selectonetwo#1#2#3\\\\\{\{#1\}\{#2\}\}
\nwendcode{}\nwbegindocs{114}\nwdocspar
The {\Tt{}{\nwbackslash}subpageref\nwendquote} macro first does a normal {\Tt{}{\nwbackslash}pageref\nwendquote}.  If the
reference is actually defined, it then goes on to check whether the
control sequence {\Tt{}2on\nwendquote}\LA{}{page referenced}\RA{} is defined and sets the
{\Tt{}{\nwbackslash}ref\nwendquote} value to get {\Tt{}a\nwendquote} etc.\ if so.  The magic, of course, is in
defining the {\Tt{}2on\nwendquote} bit appropriately.
{\Tt{}{\nwbackslash}subpageref\nwendquote} also tries to include the right hyperstuff for xhdvi.
\nwenddocs{}\nwbegincode{115}\sublabel{NW4Nr7fb-3olr1Q-W}\nwmargintag{{\nwtagstyle{}\subpageref{NW4Nr7fb-3olr1Q-W}}}\moddef{noweb.sty~{\nwtagstyle{}\subpageref{NW4Nr7fb-3olr1Q-1}}}\plusendmoddef\nwstartdeflinemarkup\nwprevnextdefs{NW4Nr7fb-3olr1Q-V}{NW4Nr7fb-3olr1Q-X}\nwenddeflinemarkup
\\newcommand\{\\subpageref\}[1]\{%
  \\nwhyperreference\{#1\}\{\\nw@genericref\\@subpageref\{#1\}\}\}
\\def\\@subpageref#1#2#3\\\\\{%
  \\@ifundefined\{2on#2\}\{#2\}\{\\nwthepagenum\{#1\}\{#2\}\}\}
\nwendcode{}\nwbegindocs{116}\nwdocspar
{\Tt{}{\nwbackslash}subpagepair\nwendquote} produces a {\Tt{}{\nwlbrace}subpage{\nwrbrace}{\nwlbrace}page{\nwrbrace}\nwendquote} pair.
\nwenddocs{}\nwbegincode{117}\sublabel{NW4Nr7fb-3olr1Q-X}\nwmargintag{{\nwtagstyle{}\subpageref{NW4Nr7fb-3olr1Q-X}}}\moddef{noweb.sty~{\nwtagstyle{}\subpageref{NW4Nr7fb-3olr1Q-1}}}\plusendmoddef\nwstartdeflinemarkup\nwprevnextdefs{NW4Nr7fb-3olr1Q-W}{NW4Nr7fb-3olr1Q-Y}\nwenddeflinemarkup
\\newcommand\{\\subpagepair\}[1]\{%  % produces \{subpage\}\{page\}
  \\@ifundefined\{r@#1\}%
    \{\{0\}\{0\}\}%
    \{\\nw@genericref\\@subpagepair\{#1\}\}\}
\\def\\@subpagepair#1#2#3\\\\\{%
  \\@ifundefined\{2on#2\}\{\{0\}\{#2\}\}\{\{#1\}\{#2\}\}\}
\nwendcode{}\nwbegindocs{118}\nwdocspar
{\Tt{}{\nwbackslash}sublabel\nwendquote} is like the {\Tt{}{\nwbackslash}label\nwendquote} command, except that it writes
{\Tt{}{\nwbackslash}newsublabel\nwendquote} onto the {\Tt{}.aux\nwendquote} file rather than {\Tt{}{\nwbackslash}newlabel\nwendquote}.
For hyperreferencing, all labels must be hypertext 
anchors, for which we use {\Tt{}{\nwbackslash}nwblindhyperanchor\nwendquote}.
\nwenddocs{}\nwbegincode{119}\sublabel{NW4Nr7fb-3olr1Q-Y}\nwmargintag{{\nwtagstyle{}\subpageref{NW4Nr7fb-3olr1Q-Y}}}\moddef{noweb.sty~{\nwtagstyle{}\subpageref{NW4Nr7fb-3olr1Q-1}}}\plusendmoddef\nwstartdeflinemarkup\nwprevnextdefs{NW4Nr7fb-3olr1Q-X}{NW4Nr7fb-3olr1Q-Z}\nwenddeflinemarkup
\\newcommand\{\\sublabel\}[1]\{%
  \\leavevmode % needed to make \\@bsphack work
  \\@bsphack
  \\nwblindhyperanchor\{#1\}%
  \\if@filesw \{\\let\\thepage\\relax
   \\def\\protect\{\\noexpand\\noexpand\\noexpand\}%
   \\edef\\@tempa\{\\write\\@auxout\{\\string
      \\newsublabel\{#1\}\{\{\}\{\\thepage\}\}\}\}%
   \\expandafter\}\\@tempa
   \\if@nobreak \\ifvmode\\nobreak\\fi\\fi\\fi\\@esphack\}
\nwendcode{}\nwbegindocs{120}\nwdocspar
{\Tt{}{\nwbackslash}nosublabel\nwendquote} creates a label with a sub-page part of~0.
\nwenddocs{}\nwbegincode{121}\sublabel{NW4Nr7fb-3olr1Q-Z}\nwmargintag{{\nwtagstyle{}\subpageref{NW4Nr7fb-3olr1Q-Z}}}\moddef{noweb.sty~{\nwtagstyle{}\subpageref{NW4Nr7fb-3olr1Q-1}}}\plusendmoddef\nwstartdeflinemarkup\nwprevnextdefs{NW4Nr7fb-3olr1Q-Y}{NW4Nr7fb-3olr1Q-a}\nwenddeflinemarkup
\\newcommand\{\\nosublabel\}[1]\{%
  \\@bsphack\\if@filesw \{\\let\\thepage\\relax
   \\def\\protect\{\\noexpand\\noexpand\\noexpand\}%
   \\edef\\@tempa\{\\write\\@auxout\{\\string
      \\newlabel\{#1\}\{\{0\}\{\\thepage\}\}\}\}%
   \\expandafter\}\\@tempa
   \\if@nobreak \\ifvmode\\nobreak\\fi\\fi\\fi\\@esphack\}
\nwendcode{}\nwbegindocs{122}\nwdocspar
{\Tt{}{\nwbackslash}newsublabel\nwendquote} is the macro that does the important work.  It is called with the
same sort of arguments as {\Tt{}{\nwbackslash}newlabel\nwendquote}: the first argument is the
label name and the second is  {\Tt{}{\nwlbrace}\LA{}ref value~{\nwtagstyle{}\subpageref{nw@notdef}}\RA{}{\nwrbrace}{\nwlbrace}\LA{}page number~{\nwtagstyle{}\subpageref{nw@notdef}}\RA{}{\nwrbrace}\nwendquote}.
(Note that the only definition here which needs to be
global is the one which is, and that {\Tt{}{\nwbackslash}global\nwendquote} is redefined by
{\Tt{}{\nwbackslash}enddocument\nwendquote}, which will bite you if you use it\dots)
\nwenddocs{}\nwbegincode{123}\sublabel{NW4Nr7fb-3olr1Q-a}\nwmargintag{{\nwtagstyle{}\subpageref{NW4Nr7fb-3olr1Q-a}}}\moddef{noweb.sty~{\nwtagstyle{}\subpageref{NW4Nr7fb-3olr1Q-1}}}\plusendmoddef\nwstartdeflinemarkup\nwprevnextdefs{NW4Nr7fb-3olr1Q-Z}{NW4Nr7fb-3olr1Q-b}\nwenddeflinemarkup
\LA{}definition of \code{}{\nwbackslash}newsublabel\edoc{}~{\nwtagstyle{}\subpageref{NW4Nr7fb-48i3K7-1}}\RA{}
\nwendcode{}\nwbegindocs{124}\nwdocspar
Before we create a {\Tt{}{\nwbackslash}newsublabel\nwendquote} for the first time, we set the
proper trailers.
\nwenddocs{}\nwbegincode{125}\sublabel{NW4Nr7fb-48i3K7-1}\nwmargintag{{\nwtagstyle{}\subpageref{NW4Nr7fb-48i3K7-1}}}\moddef{definition of \code{}{\nwbackslash}newsublabel\edoc{}~{\nwtagstyle{}\subpageref{NW4Nr7fb-48i3K7-1}}}\endmoddef\nwstartdeflinemarkup\nwusesondefline{\\{NW4Nr7fb-3olr1Q-a}}\nwprevnextdefs{\relax}{NW4Nr7fb-48i3K7-2}\nwenddeflinemarkup
\\newcommand\\newsublabel\{%
  \\nw@settrailers
  \\global\\let\\newsublabel\\@newsublabel
  \\@newsublabel\}
\nwalsodefined{\\{NW4Nr7fb-48i3K7-2}\\{NW4Nr7fb-48i3K7-3}\\{NW4Nr7fb-48i3K7-4}\\{NW4Nr7fb-48i3K7-5}}\nwused{\\{NW4Nr7fb-3olr1Q-a}}\nwendcode{}\nwbegindocs{126}First we extract the page number into {\Tt{}{\nwbackslash}this@page\nwendquote}.
\nwenddocs{}\nwbegincode{127}\sublabel{NW4Nr7fb-48i3K7-2}\nwmargintag{{\nwtagstyle{}\subpageref{NW4Nr7fb-48i3K7-2}}}\moddef{definition of \code{}{\nwbackslash}newsublabel\edoc{}~{\nwtagstyle{}\subpageref{NW4Nr7fb-48i3K7-1}}}\plusendmoddef\nwstartdeflinemarkup\nwusesondefline{\\{NW4Nr7fb-3olr1Q-a}}\nwprevnextdefs{NW4Nr7fb-48i3K7-1}{NW4Nr7fb-48i3K7-3}\nwenddeflinemarkup
\\newcommand\{\\@newsublabel\}[2]\{%
  \\edef\\this@page\{\\@cdr#2\\@nil\}%
\nwused{\\{NW4Nr7fb-3olr1Q-a}}\nwendcode{}\nwbegindocs{128}\nwdocspar
Then we see whether it's changed from the value of {\Tt{}{\nwbackslash}last@page\nwendquote}
which was stashed away by the last {\Tt{}{\nwbackslash}newsublabel\nwendquote} (or is {\Tt{}{\nwbackslash}relax\nwendquote} if
this is the first one).  If the page has changed, we reset the
counter {\Tt{}{\nwbackslash}sub@page\nwendquote} telling us how many sub-labels there have been
on the page.
\nwenddocs{}\nwbegincode{129}\sublabel{NW4Nr7fb-48i3K7-3}\nwmargintag{{\nwtagstyle{}\subpageref{NW4Nr7fb-48i3K7-3}}}\moddef{definition of \code{}{\nwbackslash}newsublabel\edoc{}~{\nwtagstyle{}\subpageref{NW4Nr7fb-48i3K7-1}}}\plusendmoddef\nwstartdeflinemarkup\nwusesondefline{\\{NW4Nr7fb-3olr1Q-a}}\nwprevnextdefs{NW4Nr7fb-48i3K7-2}{NW4Nr7fb-48i3K7-4}\nwenddeflinemarkup
  \\ifx\\this@page\\last@page\\else
    \\sub@page=\\z@
  \\fi
  \\edef\\last@page\{\\this@page\}
  \\advance\\sub@page by \\@ne
\nwused{\\{NW4Nr7fb-3olr1Q-a}}\nwendcode{}\nwbegindocs{130}\nwdocspar
If we've had at least two on the page, we define the 
{\Tt{}2on\nwendquote}\LA{}{page no.}\RA{} macro to indicate the fact.
\nwenddocs{}\nwbegincode{131}\sublabel{NW4Nr7fb-48i3K7-4}\nwmargintag{{\nwtagstyle{}\subpageref{NW4Nr7fb-48i3K7-4}}}\moddef{definition of \code{}{\nwbackslash}newsublabel\edoc{}~{\nwtagstyle{}\subpageref{NW4Nr7fb-48i3K7-1}}}\plusendmoddef\nwstartdeflinemarkup\nwusesondefline{\\{NW4Nr7fb-3olr1Q-a}}\nwprevnextdefs{NW4Nr7fb-48i3K7-3}{NW4Nr7fb-48i3K7-5}\nwenddeflinemarkup
  \\ifnum\\sub@page=\\tw@
    \\global\\@namedef\{2on\\this@page\}\{\}%
  \\fi
\nwused{\\{NW4Nr7fb-3olr1Q-a}}\nwendcode{}\nwbegindocs{132}\nwdocspar
\nextchunklabel{cl1}\nextchunklabel{cl2}
Then we write a normal {\Tt{}{\nwbackslash}newlabel\nwendquote} with the sub-reference as the
normal reference value in the second argument.
Unfortunately, if we want hypertext support, the second argument of
{\Tt{}{\nwbackslash}newlabel\nwendquote} gets complicated.
It is either
\begin{itemize}
\item
{\Tt{}{\nwlbrace}\LA{}ref value~{\nwtagstyle{}\subpageref{nw@notdef}}\RA{}{\nwrbrace}{\nwlbrace}\LA{}page number~{\nwtagstyle{}\subpageref{nw@notdef}}\RA{}{\nwrbrace}\nwendquote}, when normal {\LaTeX} is
running, or
\item
{\Tt{}{\nwlbrace}\LA{}ref value~{\nwtagstyle{}\subpageref{nw@notdef}}\RA{}{\nwrbrace}{\nwlbrace}\LA{}page number~{\nwtagstyle{}\subpageref{nw@notdef}}\RA{}{\nwrbrace}{\nwlbrace}\LA{}text~{\nwtagstyle{}\subpageref{nw@notdef}}\RA{}{\nwrbrace}{\nwrbrace}{\nwlbrace}\LA{}hyper category~{\nwtagstyle{}\subpageref{nw@notdef}}\RA{}{\nwrbrace}{\nwlbrace}\LA{}URL~{\nwtagstyle{}\subpageref{nw@notdef}}\RA{}{\nwrbrace}\nwendquote},
when the \texttt{hyperref} package is running.
(We actually detect this by looking for the \texttt{nameref} package,
because that's the one that changes the use of labels.)
\end{itemize}
We unify these two things by producing
{\Tt{}{\nwlbrace}\LA{}ref value~{\nwtagstyle{}\subpageref{nw@notdef}}\RA{}{\nwrbrace}{\nwlbrace}\LA{}page number~{\nwtagstyle{}\subpageref{nw@notdef}}\RA{}{\nwrbrace}{\nwbackslash}nw@labeltrailers\nwendquote}

We may have pending labels in support of {\Tt{}{\nwbackslash}nextchunklabel\nwendquote}, as defined in
chunk~\subpageref{chunklabel}. 
Because we want to define all of the ``pending sublabels'' in exactly
the same way, we do something a bit odd---we make the current label a
pending label as well.
\nwenddocs{}\nwbegincode{133}\sublabel{NW4Nr7fb-48i3K7-5}\nwmargintag{{\nwtagstyle{}\subpageref{NW4Nr7fb-48i3K7-5}}}\moddef{definition of \code{}{\nwbackslash}newsublabel\edoc{}~{\nwtagstyle{}\subpageref{NW4Nr7fb-48i3K7-1}}}\plusendmoddef\nwstartdeflinemarkup\nwusesondefline{\\{NW4Nr7fb-3olr1Q-a}}\nwprevnextdefs{NW4Nr7fb-48i3K7-4}{\relax}\nwenddeflinemarkup
  \\pendingsublabel\{#1\}%
  \\edef\\@tempa##1\{\\noexpand\\newlabel\{##1\}%
    \{\{\\number\\sub@page\}\{\\this@page\}\\nw@labeltrailers\}\}%
  \\pending@sublabels
  \\def\\pending@sublabels\{\}\}
\nwused{\\{NW4Nr7fb-3olr1Q-a}}\nwendcode{}\nwbegindocs{134}\nwdocspar
We can't use {\Tt{}{\nwbackslash}@ifpackageloaded\nwendquote} to see if \texttt{nameref} is
loaded, because that is restricted to the preamble, and
{\Tt{}{\nwbackslash}newsublabel\nwendquote} goes into the {\Tt{}.aux\nwendquote} file, which is executed after
the whole document is processed.
We therefore test for {\Tt{}{\nwbackslash}@secondoffive\nwendquote}.
This is lame, but what else can we do?
\nwenddocs{}\nwbegincode{135}\sublabel{NW4Nr7fb-3olr1Q-b}\nwmargintag{{\nwtagstyle{}\subpageref{NW4Nr7fb-3olr1Q-b}}}\moddef{noweb.sty~{\nwtagstyle{}\subpageref{NW4Nr7fb-3olr1Q-1}}}\plusendmoddef\nwstartdeflinemarkup\nwprevnextdefs{NW4Nr7fb-3olr1Q-a}{NW4Nr7fb-3olr1Q-c}\nwenddeflinemarkup
\\newcommand\\nw@settrailers\{% -- won't work on first run
  \\@ifpackageloaded\{nameref\}%
     \{\\gdef\\nw@labeltrailers\{\{\}\{\}\{\}\}\}%
     \{\\gdef\\nw@labeltrailers\{\}\}\}
\\renewcommand\\nw@settrailers\{% 
  \\@ifundefined\{@secondoffive\}%
     \{\\gdef\\nw@labeltrailers\{\}\}%
     \{\\gdef\\nw@labeltrailers\{\{\}\{\}\{\}\}\}\}
\nwendcode{}\nwbegindocs{136}\nwdocspar
Now we keep track of those pending guys.\nextchunklabel{chunklabel}
The goal here is to save them up until they're all equivalent to the
label on the next chunk.
We have to control expansion so chunks like \subpageref{cl1}
(\subpageref{cl2}) can be labelled twice.
\nwenddocs{}\nwbegincode{137}\sublabel{NW4Nr7fb-3olr1Q-c}\nwmargintag{{\nwtagstyle{}\subpageref{NW4Nr7fb-3olr1Q-c}}}\moddef{noweb.sty~{\nwtagstyle{}\subpageref{NW4Nr7fb-3olr1Q-1}}}\plusendmoddef\nwstartdeflinemarkup\nwprevnextdefs{NW4Nr7fb-3olr1Q-b}{NW4Nr7fb-3olr1Q-d}\nwenddeflinemarkup
\\newcommand\{\\nextchunklabel\}[1]\{%
  \\nwblindhyperanchor\{#1\}%   % looks slightly bogus --- nr
  \\@bsphack\\if@filesw \{\\let\\thepage\\relax
      \\edef\\@tempa\{\\write\\@auxout\{\\string\\pendingsublabel\{#1\}\}\}%
      \\expandafter\}\\@tempa
   \\if@nobreak \\ifvmode\\nobreak\\fi\\fi\\fi\\@esphack\}
\\newcommand\\pendingsublabel[1]\{%
  \\def\\@tempa\{\\noexpand\\@tempa\}%
  \\edef\\pending@sublabels\{\\noexpand\\@tempa\{#1\}\\pending@sublabels\}\}
\\def\\pending@sublabels\{\}
\nwendcode{}\nwbegincode{138}\sublabel{NW4Nr7fb-4PphKm-1}\nwmargintag{{\nwtagstyle{}\subpageref{NW4Nr7fb-4PphKm-1}}}\moddef{man page: noweb style control sequences~{\nwtagstyle{}\subpageref{NW4Nr7fb-4PphKm-1}}}\endmoddef\nwstartdeflinemarkup\nwprevnextdefs{\relax}{NW4Nr7fb-4PphKm-2}\nwenddeflinemarkup
.PP \\" .TP will not work with the backslashes on the next line. Period.
\\fB\\\\nextchunklabel\{l\}\\fP
.RS
Associates label \\fBl\\fP
with the sub-page reference of the next code chunk.
Can be used in for concise chunk cross-reference with, e.g.,
\\fBchunk~\\\\subpageref\{l\}\\fP.
.RE
\nwalsodefined{\\{NW4Nr7fb-4PphKm-2}\\{NW4Nr7fb-4PphKm-3}}\nwnotused{man page: noweb style control sequences}\nwendcode{}\nwbegindocs{139}\nwdocspar
We need to define these.
\nwenddocs{}\nwbegincode{140}\sublabel{NW4Nr7fb-3olr1Q-d}\nwmargintag{{\nwtagstyle{}\subpageref{NW4Nr7fb-3olr1Q-d}}}\moddef{noweb.sty~{\nwtagstyle{}\subpageref{NW4Nr7fb-3olr1Q-1}}}\plusendmoddef\nwstartdeflinemarkup\nwprevnextdefs{NW4Nr7fb-3olr1Q-c}{NW4Nr7fb-3olr1Q-e}\nwenddeflinemarkup
\\def\\last@page\{\\relax\}
\\newcount\\sub@page
\nwendcode{}\nwbegindocs{141}\nwdocspar
We no longer use Rainer's new expandable definitions of {\Tt{}{\nwbackslash}ref\nwendquote} and
{\Tt{}{\nwbackslash}pageref\nwendquote} to minimise the risk of nasty surprises; enough time has
elapsed that this should no longer be necessary.
\nwenddocs{}\nwbegincode{142}\sublabel{NW4Nr7fb-1YYI5t-1}\nwmargintag{{\nwtagstyle{}\subpageref{NW4Nr7fb-1YYI5t-1}}}\moddef{old noweb.sty~{\nwtagstyle{}\subpageref{NW4Nr7fb-1YYI5t-1}}}\endmoddef\nwstartdeflinemarkup\nwenddeflinemarkup
% RmS 92/08/14: made \\ref and \\pageref robust
\\def\\ref#1\{\\@ifundefined\{r@#1\}\{\{\\bf ??\}\LA{}warn of undefined reference to \code{}{\#}1\edoc{}~{\nwtagstyle{}\subpageref{NW4Nr7fb-4CAbV-1}}\RA{}\}%
    \{\\expandafter\\expandafter\\expandafter
     \\@car\\csname r@#1\\endcsname\\@nil\\null\}\}
\\def\\pageref#1\{\\@ifundefined\{r@#1\}\{\{\\bf ??\}\LA{}warn of undefined reference to \code{}{\#}1\edoc{}~{\nwtagstyle{}\subpageref{NW4Nr7fb-4CAbV-1}}\RA{}\}%
     \{\\expandafter\\expandafter\\expandafter
      \\@cdr\\csname r@#1\\endcsname\\@nil\\null\}\}
\\def\\@refpair#1\{\\@ifundefined\{r@#1\}\{\{0\}\{0\}\LA{}warn of undefined reference to \code{}{\#}1\edoc{}~{\nwtagstyle{}\subpageref{NW4Nr7fb-4CAbV-1}}\RA{}\}%
    \{\\@nameuse\{r@#1\}\}\}
\nwnotused{old noweb.sty}\nwendcode{}\nwbegincode{143}\sublabel{NW4Nr7fb-4CAbV-1}\nwmargintag{{\nwtagstyle{}\subpageref{NW4Nr7fb-4CAbV-1}}}\moddef{warn of undefined reference to \code{}{\#}1\edoc{}~{\nwtagstyle{}\subpageref{NW4Nr7fb-4CAbV-1}}}\endmoddef\nwstartdeflinemarkup\nwusesondefline{\\{NW4Nr7fb-1dYqBd-1}\\{NW4Nr7fb-1YYI5t-1}}\nwenddeflinemarkup
\\@warning\{Reference `#1' on page \\thepage \\space undefined\}
\nwused{\\{NW4Nr7fb-1dYqBd-1}\\{NW4Nr7fb-1YYI5t-1}}\nwendcode{}\nwbegindocs{144}\nwdocspar

Here a a couple of hooks for formatting sub-page numbers,
which can be alphabetic, numeric, or omitted.\stylehook
\nwenddocs{}\nwbegincode{145}\sublabel{NW4Nr7fb-3olr1Q-e}\nwmargintag{{\nwtagstyle{}\subpageref{NW4Nr7fb-3olr1Q-e}}}\moddef{noweb.sty~{\nwtagstyle{}\subpageref{NW4Nr7fb-3olr1Q-1}}}\plusendmoddef\nwstartdeflinemarkup\nwprevnextdefs{NW4Nr7fb-3olr1Q-d}{NW4Nr7fb-3olr1Q-f}\nwenddeflinemarkup
\\def\\@alphasubpagenum#1#2\{#2\\ifnum#1=0 \\else\\@alph\{#1\}\\fi\}
\\def\\@nosubpagenum#1#2\{#2\}
\\def\\@numsubpagenum#1#2\{#2\\ifnum#1=0 \\else.\\@arabic\{#1\}\\fi\}
\\def\\nwopt@nosubpage\{\\let\\nwthepagenum=\\@nosubpagenum\\nwopt@nomargintag\}
\\def\\nwopt@numsubpage\{\\let\\nwthepagenum=\\@numsubpagenum\}
\\def\\nwopt@alphasubpage\{\\let\\nwthepagenum=\\@alphasubpagenum\}
\\nwopt@alphasubpage
\nwendcode{}\nwbegindocs{146}\nwdocspar
In rare cases, there may be more than 26 chunks on a page.
In such a case, we need a sub-page numbering scheme that can go beyond
``a to z.''
The scheme I have chosen is ``a to z, then aa to zz, then aaa to zzz,
etc.''
The conversion requires a bit of thought because it is \emph{not} an
ordinary conversion of integer to string as we usually think of such
things.
The problem is that the meaning of the letters depends on the
position; the letter~a acts like a zero in some positions or a one in
others.

The solution I have implemented uses a variable {\Tt{}bound\nwendquote} which is
always equal to $26^k$ for some~$k$.
If we write the recurrence $B_k = B_{k-1} + 26^k$, with $B_0 = 0$, we
then use a string of~$k$ letters to represent numbers between
$B_{k-1}$~and~$B_k$.
Within that string, a's are 0's, and so on up to z's which are 25's,
and we use standard integer-conversion methods to encode $n-B_{k-1}$.

The following Icon implementation may be more perspicuous than the
{\TeX} code actually used.
Here the variable {\Tt{}bound\nwendquote} is $26^k$, with $k=1$ initially, and
{\Tt{}n\nwendquote}~is $n-B_{k-1}$.
The first loop finds the right~$k$, and the second does the usual
string conversion.
\nwenddocs{}\nwbegincode{147}\sublabel{NW4Nr7fb-41ggXv-1}\nwmargintag{{\nwtagstyle{}\subpageref{NW4Nr7fb-41ggXv-1}}}\moddef{Icon code for subpage numbering~{\nwtagstyle{}\subpageref{NW4Nr7fb-41ggXv-1}}}\endmoddef\nwstartdeflinemarkup\nwenddeflinemarkup
procedure alphastring(n)
  bound := 26

  while n >= bound do \{
    # invariant: bound = 26^(k+1) & n is initial n - B_k
    n -:= bound
    bound *:= 26
  \}
  
  while bound > 1 do \{
    bound /:= 26
    d := integer(n / bound)
    n -:= d * bound
    writes(&lcase[d+1])
  \}
end
\nwnotused{Icon code for subpage numbering}\nwendcode{}\nwbegindocs{148}\nwdocspar
Here's {\TeX} code to achieve the same end.
The entire macro body is enclosed in braces, so that it can be used
with {\Tt{}{\nwbackslash}loop\nwendquote} without picking up the wrong {\Tt{}{\nwbackslash}repeat\nwendquote}.
\nwenddocs{}\nwbegincode{149}\sublabel{NW4Nr7fb-3olr1Q-f}\nwmargintag{{\nwtagstyle{}\subpageref{NW4Nr7fb-3olr1Q-f}}}\moddef{noweb.sty~{\nwtagstyle{}\subpageref{NW4Nr7fb-3olr1Q-1}}}\plusendmoddef\nwstartdeflinemarkup\nwprevnextdefs{NW4Nr7fb-3olr1Q-e}{NW4Nr7fb-3olr1Q-g}\nwenddeflinemarkup
\\newcount\\@nwalph@n
\\let\\@nwalph@d\\@tempcnta
\\let\\@nwalph@bound\\@tempcntb
\\def\\@nwlongalph#1\{\{%
  \\@nwalph@n=#1\\advance\\@nwalph@n by-1
  \\@nwalph@bound=26
  \\loop\\ifnum\\@nwalph@n<\\@nwalph@bound\\else
     \\advance\\@nwalph@n by -\\@nwalph@bound
     \\multiply\\@nwalph@bound by 26
  \\repeat
  \\loop\\ifnum\\@nwalph@bound>1
    \\divide\\@nwalph@bound by 26
    \\@nwalph@d=\\@nwalph@n\\divide\\@nwalph@d by \\@nwalph@bound
    % d := d * bound ; n -:= d; d := d / bound --- saves a temporary
    \\multiply\\@nwalph@d by \\@nwalph@bound
    \\advance\\@nwalph@n by -\\@nwalph@d
    \\divide\\@nwalph@d by \\@nwalph@bound
    \\advance\\@nwalph@d by 1 \\@alph\{\\@nwalph@d\}%
  \\repeat
\}\}
\nwendcode{}\nwbegindocs{150}\nwdocspar
\iffalse
\nwenddocs{}\nwbegincode{151}\sublabel{NW4Nr7fb-3RhSlV-5}\nwmargintag{{\nwtagstyle{}\subpageref{NW4Nr7fb-3RhSlV-5}}}\moddef{man page: \code{}{\nwbackslash}noweboptions\edoc{}~{\nwtagstyle{}\subpageref{NW4Nr7fb-3RhSlV-1}}}\plusendmoddef\nwstartdeflinemarkup\nwprevnextdefs{NW4Nr7fb-3RhSlV-4}{NW4Nr7fb-3RhSlV-6}\nwenddeflinemarkup
.TP
.B alphasubpage, numsubpage, nosubpage
Number chunks by the number of the page on which they appear,
followed by an alphabetic (numeric, not used) ``sub-page'' indicator.
Defaults to 
.B alphasubpage.
.B nosubpage
implies
.B nomargintag.
\nwendcode{}\nwbegindocs{152}\fi
\nwenddocs{}\nwbegindocs{153}\nwdocspar

\subsection{{\tt WEB}-like chunk numbering}
Here's a righteous hack: we get the effect of WEB-like chunk numbers
just by redefining {\Tt{}{\nwbackslash}sublabel\nwendquote} to use a counter instead of the current page number.
Since the numbers are all distinct, no sub-page number is ever used.
\nwenddocs{}\nwbegincode{154}\sublabel{NW4Nr7fb-3olr1Q-g}\nwmargintag{{\nwtagstyle{}\subpageref{NW4Nr7fb-3olr1Q-g}}}\moddef{noweb.sty~{\nwtagstyle{}\subpageref{NW4Nr7fb-3olr1Q-1}}}\plusendmoddef\nwstartdeflinemarkup\nwprevnextdefs{NW4Nr7fb-3olr1Q-f}{NW4Nr7fb-3olr1Q-h}\nwenddeflinemarkup
\\newcount\\nw@chunkcount
\\nw@chunkcount=\\@ne
\\newcommand\{\\weblabel\}[1]\{%
  \\@bsphack
  \\nwblindhyperanchor\{#1\}%
  \\if@filesw \{\\let\\thepage\\relax
   \\def\\protect\{\\noexpand\\noexpand\\noexpand\}%
   \\edef\\@tempa\{\\write\\@auxout\{\\string
      \\newsublabel\{#1\}\{\{\}\{\\number\\nw@chunkcount\}\}\}\}%
   \\expandafter\}\\@tempa
   \\global\\advance\\nw@chunkcount by \\@ne
   \\if@nobreak \\ifvmode\\nobreak\\fi\\fi\\fi\\@esphack\}
\\def\\nwopt@webnumbering\{%
  \\let\\sublabel=\\weblabel
  \\def\\nwpageword\{chunk\}\\def\\nwpagesword\{chunks\}%
  \\def\\nwpageprep\{in\}\}
\nwendcode{}\nwbegindocs{155}\nwdocspar
\iffalse
\nwenddocs{}\nwbegincode{156}\sublabel{NW4Nr7fb-3RhSlV-6}\nwmargintag{{\nwtagstyle{}\subpageref{NW4Nr7fb-3RhSlV-6}}}\moddef{man page: \code{}{\nwbackslash}noweboptions\edoc{}~{\nwtagstyle{}\subpageref{NW4Nr7fb-3RhSlV-1}}}\plusendmoddef\nwstartdeflinemarkup\nwprevnextdefs{NW4Nr7fb-3RhSlV-5}{NW4Nr7fb-3RhSlV-7}\nwenddeflinemarkup
.TP
.B webnumbering
Number chunks consecutively, in 
.I WEB
style, instead of using sub-page numbers.
\nwendcode{}\nwbegindocs{157}\fi
\nwenddocs{}\nwbegindocs{158}\nwdocspar
\subsection{Indexing (identifier cross-reference) support}

\subsubsection{Tracking definitions and uses}
All index definitions and uses are associated with 
a label defined with {\Tt{}{\nwbackslash}sublabel\nwendquote} or {\Tt{}{\nwbackslash}nosublabel\nwendquote}.
Either the label is the {\Tt{}{\nwbackslash}sublabel\nwendquote} of the code chunk in which the definition or use
appears, or it is a {\Tt{}{\nwbackslash}nosublabel\nwendquote} appearing in the middle of a
documentation chunk.
\nwenddocs{}\nwbegincode{159}\sublabel{NW4Nr7fb-3olr1Q-h}\nwmargintag{{\nwtagstyle{}\subpageref{NW4Nr7fb-3olr1Q-h}}}\moddef{noweb.sty~{\nwtagstyle{}\subpageref{NW4Nr7fb-3olr1Q-1}}}\plusendmoddef\nwstartdeflinemarkup\nwprevnextdefs{NW4Nr7fb-3olr1Q-g}{NW4Nr7fb-3olr1Q-i}\nwenddeflinemarkup
% \\nwindexdefn\{printable name\}\{identifying label\}\{label of chunk\}
% \\nwindexuse\{printable name\}\{identifying label\}\{label of chunk\}

\\def\\nwindexdefn#1#2#3\{\\@auxix\{\\protect\\nwixd\}\{#2\}\{#3\}\}
\\def\\nwindexuse#1#2#3\{\\@auxix\{\\protect\\nwixu\}\{#2\}\{#3\}\}

\\def\\@auxix#1#2#3\{% \{marker\}\{id label\}\{subpage label\}
   \\@bsphack\\if@filesw \{\\let\\nwixd\\relax\\let\\nwixu\\relax
   \\def\\protect\{\\noexpand\\noexpand\\noexpand\}%
   \\edef\\@tempa\{\\write\\@auxout\{\\string\\nwixadd\{#1\}\{#2\}\{#3\}\}\}%
   \\expandafter\}\\@tempa
   \\if@nobreak \\ifvmode\\nobreak\\fi\\fi\\fi\\@esphack\}
\nwendcode{}\nwbegincode{160}\sublabel{NW4Nr7fb-3olr1Q-i}\nwmargintag{{\nwtagstyle{}\subpageref{NW4Nr7fb-3olr1Q-i}}}\moddef{noweb.sty~{\nwtagstyle{}\subpageref{NW4Nr7fb-3olr1Q-1}}}\plusendmoddef\nwstartdeflinemarkup\nwprevnextdefs{NW4Nr7fb-3olr1Q-h}{NW4Nr7fb-3olr1Q-j}\nwenddeflinemarkup
% \\nwixadd\{marker\}\{idlabel\}\{subpage label\}
\\def\\nwixadd#1#2#3\{%
  \\@ifundefined\{nwixl@#2\}%
    \{\\global\\@namedef\{nwixl@#2\}\{#1\{#3\}\}\}%
    \{\\expandafter\\nwix@cons\\csname nwixl@#2\\endcsname\{#1\{#3\}\}\}\}
\nwendcode{}\nwbegindocs{161}\nwdocspar
\subsubsection{Subscripted identifiers}
We use either explicit subscripts or hyperlinks to point identifiers
to their definitions.
\nwenddocs{}\nwbegincode{162}\sublabel{NW4Nr7fb-3olr1Q-j}\nwmargintag{{\nwtagstyle{}\subpageref{NW4Nr7fb-3olr1Q-j}}}\moddef{noweb.sty~{\nwtagstyle{}\subpageref{NW4Nr7fb-3olr1Q-1}}}\plusendmoddef\nwstartdeflinemarkup\nwprevnextdefs{NW4Nr7fb-3olr1Q-i}{NW4Nr7fb-3olr1Q-k}\nwenddeflinemarkup
\\def\\@nwsubscriptident#1#2\{\\mbox\{$\\mbox\{#1\}_\{\\mathrm\{\\subpageref\{#2\}\}\}$\}\}
\\def\\@nwnosubscriptident#1#2\{#1\}
\\def\\@nwhyperident#1#2\{\\leavevmode\\nwhyperreference\{#2\}\{#1\}\}
\nwendcode{}\nwbegindocs{163}\nwdocspar
We can use subscripts, hyperlinks, or nothing on all identifiers.
\nwenddocs{}\nwbegincode{164}\sublabel{NW4Nr7fb-3olr1Q-k}\nwmargintag{{\nwtagstyle{}\subpageref{NW4Nr7fb-3olr1Q-k}}}\moddef{noweb.sty~{\nwtagstyle{}\subpageref{NW4Nr7fb-3olr1Q-1}}}\plusendmoddef\nwstartdeflinemarkup\nwprevnextdefs{NW4Nr7fb-3olr1Q-j}{NW4Nr7fb-3olr1Q-l}\nwenddeflinemarkup
\\def\\nwopt@subscriptidents\{%
  \\let\\nwlinkedidentq\\@nwsubscriptident
  \\let\\nwlinkedidentc\\@nwsubscriptident
\}
\\def\\nwopt@nosubscriptidents\{%
  \\let\\nwlinkedidentq\\@nwnosubscriptident
  \\let\\nwlinkedidentc\\@nwnosubscriptident
\}
\\def\\nwopt@hyperidents\{%
  \\let\\nwlinkedidentq\\@nwhyperident
  \\let\\nwlinkedidentc\\@nwhyperident
\}
\\def\\nwopt@nohyperidents\{%
  \\let\\nwlinkedidentq\\@nwnosubscriptident
  \\let\\nwlinkedidentc\\@nwnosubscriptident
\}
\nwendcode{}\nwbegindocs{165}\nwdocspar
We can change only identifiers appearing in quoted code.
\nwenddocs{}\nwbegincode{166}\sublabel{NW4Nr7fb-3olr1Q-l}\nwmargintag{{\nwtagstyle{}\subpageref{NW4Nr7fb-3olr1Q-l}}}\moddef{noweb.sty~{\nwtagstyle{}\subpageref{NW4Nr7fb-3olr1Q-1}}}\plusendmoddef\nwstartdeflinemarkup\nwprevnextdefs{NW4Nr7fb-3olr1Q-k}{NW4Nr7fb-3olr1Q-m}\nwenddeflinemarkup
\\def\\nwopt@subscriptquotedidents\{%
  \\let\\nwlinkedidentq\\@nwsubscriptident
\}
\\def\\nwopt@nosubscriptquotedidents\{%
  \\let\\nwlinkedidentq\\@nwnosubscriptident
\}
\\def\\nwopt@hyperquotedidents\{%
  \\let\\nwlinkedidentq\\@nwhyperident
\}
\\def\\nwopt@nohyperquotedidents\{%
  \\let\\nwlinkedidentq\\@nwnosubscriptident
\}
\nwendcode{}\nwbegindocs{167}\nwdocspar
The default is to hyperlink everything.
\nwenddocs{}\nwbegincode{168}\sublabel{NW4Nr7fb-3olr1Q-m}\nwmargintag{{\nwtagstyle{}\subpageref{NW4Nr7fb-3olr1Q-m}}}\moddef{noweb.sty~{\nwtagstyle{}\subpageref{NW4Nr7fb-3olr1Q-1}}}\plusendmoddef\nwstartdeflinemarkup\nwprevnextdefs{NW4Nr7fb-3olr1Q-l}{NW4Nr7fb-3olr1Q-n}\nwenddeflinemarkup
\\nwopt@hyperidents
\nwendcode{}\nwbegindocs{169}\nwdocspar
\iffalse
\nwenddocs{}\nwbegincode{170}\sublabel{NW4Nr7fb-3RhSlV-7}\nwmargintag{{\nwtagstyle{}\subpageref{NW4Nr7fb-3RhSlV-7}}}\moddef{man page: \code{}{\nwbackslash}noweboptions\edoc{}~{\nwtagstyle{}\subpageref{NW4Nr7fb-3RhSlV-1}}}\plusendmoddef\nwstartdeflinemarkup\nwprevnextdefs{NW4Nr7fb-3RhSlV-6}{NW4Nr7fb-3RhSlV-8}\nwenddeflinemarkup
.TP
.B subscriptidents, nosubscriptidents, hyperidents, nohyperidents
Controls subscripting of identifiers in code, including quoted code.
Selecting
.B subscriptidents
means an identifier appearing in a code chunk (or in quoted code
within a documentation 
chunk) will be subscripted with the chunk number of its definition.
.B hyperidents
means such identifiers will be hyperlinked to their definitions,
provided of course that a hypertext package like
.B hyperref
is loaded.
.B nosubscriptidents
and 
.B nohyperidents
are equivalent, and they turn off such markings.
The default is
.B hyperidents.
.TP
.B subscriptquotedidents, nosubscriptquotedidents, hyperquotedidents, nohyperquotedidents
Controls linking of identifiers as above, but applies only to uses of
identifiers
in quoted code.
\nwendcode{}\nwbegindocs{171}\fi
\nwenddocs{}\nwbegindocs{172}\nwdocspar
\subsubsection{Writing lists with commas and ``and''}
You get one of
\begin{itemize}
\item ``$a$''
\item ``$a$ and $b$''
\item ``$a$, $\ldots$, $b$, and $c$''
\end{itemize}
Plus {\Tt{}{\nwbackslash}{\nwbackslash}\nwendquote} is applied to each element of the list.
\nwenddocs{}\nwbegincode{173}\sublabel{NW4Nr7fb-3olr1Q-n}\nwmargintag{{\nwtagstyle{}\subpageref{NW4Nr7fb-3olr1Q-n}}}\moddef{noweb.sty~{\nwtagstyle{}\subpageref{NW4Nr7fb-3olr1Q-1}}}\plusendmoddef\nwstartdeflinemarkup\nwprevnextdefs{NW4Nr7fb-3olr1Q-m}{NW4Nr7fb-3olr1Q-o}\nwenddeflinemarkup
\\newcount\\@commacount
\\def\\commafy#1\{%
  \{\\nwix@listcount\{#1\}\\@commacount=\\nwix@counter
   \\let\\@comma@each=\\\\%
   \\ifcase\\@commacount\\let\\\\=\\@comma@each\\or\\let\\\\=\\@comma@each\\or
     \\def\\\\\{\\def\\\\\{ \\@nwlangdepand\\ \\@comma@each\}\\@comma@each\}\\else
     \\def\\\\\{\\def\\\\\{, %
                   \\advance\\@commacount by \\m@ne
                   \\ifnum\\@commacount=1 \\@nwlangdepand~\\fi\\@comma@each\}\\@comma@each\}\\fi
   #1\}\}
\nwendcode{}\nwbegindocs{174}\nwdocspar

\subsubsection{New, improved index code}
There are two kinds of lists.
One kind is a generic list in which elements are preceded by {\Tt{}{\nwbackslash}{\nwbackslash}\nwendquote}.
If the elements are index elements, they are {\em{\Tt{}{\nwlbrace}\nwendquote}printable
identifier{\Tt{}{\nwrbrace}{\nwlbrace}\nwendquote}label{\Tt{}{\nwrbrace}\nwendquote}} pairs.
The other kind is a list of sub-page labels, in which each 
element is preceded by either {\Tt{}{\nwbackslash}nwixd\nwendquote} or {\Tt{}{\nwbackslash}nwixu\nwendquote}.
\nwenddocs{}\nwbegincode{175}\sublabel{NW4Nr7fb-3olr1Q-o}\nwmargintag{{\nwtagstyle{}\subpageref{NW4Nr7fb-3olr1Q-o}}}\moddef{noweb.sty~{\nwtagstyle{}\subpageref{NW4Nr7fb-3olr1Q-1}}}\plusendmoddef\nwstartdeflinemarkup\nwprevnextdefs{NW4Nr7fb-3olr1Q-n}{NW4Nr7fb-3olr1Q-p}\nwenddeflinemarkup
\\def\\nwix@cons#1#2\{% \{list\}\{\\marker\{element\}\}
  \{\\toks0=\\expandafter\{#1\}\\def\\@tempa\{#2\}\\toks2=\\expandafter\{\\@tempa\}%
   \\xdef#1\{\\the\\toks0 \\the\\toks2 \}\}\}
\nwendcode{}\nwbegindocs{176}\nwdocspar
The reference list for an identifier labelled {\em id}
is always called {\Tt{}{\nwbackslash}nwixl@\nwendquote}{\em id}.
Most applications will work with reference lists by applying {\Tt{}{\nwbackslash}{\nwbackslash}\nwendquote}
either to the defs or to the uses.
\nwenddocs{}\nwbegincode{177}\sublabel{NW4Nr7fb-3olr1Q-p}\nwmargintag{{\nwtagstyle{}\subpageref{NW4Nr7fb-3olr1Q-p}}}\moddef{noweb.sty~{\nwtagstyle{}\subpageref{NW4Nr7fb-3olr1Q-1}}}\plusendmoddef\nwstartdeflinemarkup\nwprevnextdefs{NW4Nr7fb-3olr1Q-o}{NW4Nr7fb-3olr1Q-q}\nwenddeflinemarkup
\\def\\nwix@uses#1\{% \{label\}
  \\def\\nwixu\{\\\\\}\\let\\nwixd\\@gobble\\@nameuse\{nwixl@#1\}\}
\\def\\nwix@defs#1\{% \{label\}
  \\def\\nwixd\{\\\\\}\\let\\nwixu\\@gobble\\@nameuse\{nwixl@#1\}\}
\nwendcode{}\nwbegindocs{178}\nwdocspar

Some applications count uses to see whether there is any need to
display information.
\nwenddocs{}\nwbegincode{179}\sublabel{NW4Nr7fb-3olr1Q-q}\nwmargintag{{\nwtagstyle{}\subpageref{NW4Nr7fb-3olr1Q-q}}}\moddef{noweb.sty~{\nwtagstyle{}\subpageref{NW4Nr7fb-3olr1Q-1}}}\plusendmoddef\nwstartdeflinemarkup\nwprevnextdefs{NW4Nr7fb-3olr1Q-p}{NW4Nr7fb-3olr1Q-r}\nwenddeflinemarkup
\\newcount\\nwix@counter
\\def\\nwix@listcount#1\{% \{list with \\\\\}
  \{\\count@=0
   \\def\\\\##1\{\\advance\\count@ by \\@ne \}%
   #1\\global\\nwix@counter=\\count@ \}\}
\\def\\nwix@usecount#1\{\\nwix@listcount\{\\nwix@uses\{#1\}\}\}
\\def\\nwix@defcount#1\{\\nwix@listcount\{\\nwix@defs\{#1\}\}\}
\nwendcode{}\nwbegindocs{180}\nwdocspar
\subsubsection{Supporting a mini-index at the end of each chunk}
When displaying identifiers used, show the identifier and its
definitions.
\nwenddocs{}\nwbegincode{181}\sublabel{NW4Nr7fb-3olr1Q-r}\nwmargintag{{\nwtagstyle{}\subpageref{NW4Nr7fb-3olr1Q-r}}}\moddef{noweb.sty~{\nwtagstyle{}\subpageref{NW4Nr7fb-3olr1Q-1}}}\plusendmoddef\nwstartdeflinemarkup\nwprevnextdefs{NW4Nr7fb-3olr1Q-q}{NW4Nr7fb-3olr1Q-s}\nwenddeflinemarkup
\\def\\nwix@id@defs#1\{% index pair
  \{\{\\Tt \\@car#1\\@nil\}%
  \\def\\\\##1\{\\nwix@defs@space\\subpageref\{##1\}\}\\nwix@defs\{\\@cdr#1\\@nil\}\}\}
  % useful above to change ~ into something that can break
% this option is undocumented because I think breakdefs is always right
\\def\\nwopt@breakdefs\{\\def\\nwix@defs@space\{\\penalty200\\ \}\}
\\def\\nwopt@nobreakdefs\{\\def\\nwix@defs@space\{~\}\} % old code
\\nwopt@breakdefs
\\def\\nwidentuses#1\{% list of index pairs
  \\nwcodecomment\{\\@nwlangdepuss\\ \\let\\\\=\\nwix@id@defs\\commafy\{#1\}.\}\}
\nwendcode{}\nwbegindocs{182}\nwdocspar
\iffalse
\nwenddocs{}\nwbegincode{183}\sublabel{NW4Nr7fb-3PwwDi-1}\nwmargintag{{\nwtagstyle{}\subpageref{NW4Nr7fb-3PwwDi-1}}}\moddef{undocumented -- man page: \code{}{\nwbackslash}noweboptions\edoc{}~{\nwtagstyle{}\subpageref{NW4Nr7fb-3PwwDi-1}}}\endmoddef\nwstartdeflinemarkup\nwenddeflinemarkup
.TP
.B breakdefs, nobreakdefs
.BR breakdefs ,
which is the default,
permits long lists of definitions to be broken in identifier cross-reference.
Useful if identifier cross-reference produces lots of overfull hboxes.
.B nobreakdefs
is the old behavior, which should never be needed.
\nwnotused{undocumented -- man page: [[\noweboptions]]}\nwendcode{}\nwbegindocs{184}\fi 
\nwenddocs{}\nwbegindocs{185}\nwdocspar
The definitions section is a bit more complex, because it is omitted
if none of the identifiers defined is ever used.
\nwenddocs{}\nwbegincode{186}\sublabel{NW4Nr7fb-3olr1Q-s}\nwmargintag{{\nwtagstyle{}\subpageref{NW4Nr7fb-3olr1Q-s}}}\moddef{noweb.sty~{\nwtagstyle{}\subpageref{NW4Nr7fb-3olr1Q-1}}}\plusendmoddef\nwstartdeflinemarkup\nwprevnextdefs{NW4Nr7fb-3olr1Q-r}{NW4Nr7fb-3olr1Q-t}\nwenddeflinemarkup
\\def\\nwix@totaluses#1\{% list of index pairs
  \{\\count@=0
   \\def\\\\##1\{\\nwix@usecount\{\\@cdr##1\\@nil\}\\advance\\count@ by\\nwix@counter\}%
   #1\\global\\nwix@counter\\count@ \}\}
\\def\\nwix@id@uses#1#2\{% \{ident\}\{label\}
  \\nwix@usecount\{#2\}\\ifnum\\nwix@counter>0
    \{\\advance\\leftskip by \\codemargin
     \\nwcodecomment\{\{\\Tt #1\}, \\@nwlangdepusd\\ \\nwpageprep\\ \\@pagesl\{\\nwix@uses\{#2\}\}.\}\}%
  \\else
    \\ifnw@hideunuseddefs\\else
      \{\\advance\\leftskip by \\codemargin \\nwcodecomment\{\{\\Tt #1\}, \\@nwlangdepnvu.\}\}%
    \\fi
  \\fi\}
\\def\\nwidentdefs#1\{% list of index pairs
  \\ifnw@hideunuseddefs\\nwix@totaluses\{#1\}\\else\\nwix@listcount\{#1\}\\fi
  \\ifnum\\nwix@counter>0
    \\nwcodecomment\{\\@nwlangdepdfs:\}%
    \{\\def\\\\##1\{\\nwix@id@uses ##1\}#1\}%
  \\fi\}
\nwendcode{}\nwbegincode{187}\sublabel{NW4Nr7fb-3olr1Q-t}\nwmargintag{{\nwtagstyle{}\subpageref{NW4Nr7fb-3olr1Q-t}}}\moddef{noweb.sty~{\nwtagstyle{}\subpageref{NW4Nr7fb-3olr1Q-1}}}\plusendmoddef\nwstartdeflinemarkup\nwprevnextdefs{NW4Nr7fb-3olr1Q-s}{NW4Nr7fb-3olr1Q-u}\nwenddeflinemarkup
\\newif\\ifnw@hideunuseddefs\\nw@hideunuseddefsfalse
\\def\\nwopt@hideunuseddefs\{\\nw@hideunuseddefstrue\}
\nwendcode{}\nwbegindocs{188}\nwdocspar
\iffalse
\nwenddocs{}\nwbegincode{189}\sublabel{NW4Nr7fb-3RhSlV-8}\nwmargintag{{\nwtagstyle{}\subpageref{NW4Nr7fb-3RhSlV-8}}}\moddef{man page: \code{}{\nwbackslash}noweboptions\edoc{}~{\nwtagstyle{}\subpageref{NW4Nr7fb-3RhSlV-1}}}\plusendmoddef\nwstartdeflinemarkup\nwprevnextdefs{NW4Nr7fb-3RhSlV-7}{NW4Nr7fb-3RhSlV-9}\nwenddeflinemarkup
.TP
.B hideunuseddefs
Omit defined but unused identifiers from
the local identifier cross-reference (Preston Briggs).
\nwendcode{}\nwbegindocs{190}\fi
\nwenddocs{}\nwbegincode{191}\sublabel{NW4Nr7fb-3olr1Q-u}\nwmargintag{{\nwtagstyle{}\subpageref{NW4Nr7fb-3olr1Q-u}}}\moddef{noweb.sty~{\nwtagstyle{}\subpageref{NW4Nr7fb-3olr1Q-1}}}\plusendmoddef\nwstartdeflinemarkup\nwprevnextdefs{NW4Nr7fb-3olr1Q-t}{NW4Nr7fb-3olr1Q-v}\nwenddeflinemarkup
\\def\\nwopt@noidentxref\{%
  \\let\\nwidentdefs\\@gobble
  \\let\\nwidentuses\\@gobble\}
\nwendcode{}\nwbegindocs{192}\nwdocspar
\iffalse
\nwenddocs{}\nwbegincode{193}\sublabel{NW4Nr7fb-3RhSlV-9}\nwmargintag{{\nwtagstyle{}\subpageref{NW4Nr7fb-3RhSlV-9}}}\moddef{man page: \code{}{\nwbackslash}noweboptions\edoc{}~{\nwtagstyle{}\subpageref{NW4Nr7fb-3RhSlV-1}}}\plusendmoddef\nwstartdeflinemarkup\nwprevnextdefs{NW4Nr7fb-3RhSlV-8}{NW4Nr7fb-3RhSlV-A}\nwenddeflinemarkup
.TP
.B noidentxref
Omit the local identifier cross-reference which follows each code chunk.
\nwendcode{}\nwbegindocs{194}\fi

\subsubsection{Support for chunk and identifier indices}
The index in the back shows absolutely all the pages.
\nwenddocs{}\nwbegincode{195}\sublabel{NW4Nr7fb-3olr1Q-v}\nwmargintag{{\nwtagstyle{}\subpageref{NW4Nr7fb-3olr1Q-v}}}\moddef{noweb.sty~{\nwtagstyle{}\subpageref{NW4Nr7fb-3olr1Q-1}}}\plusendmoddef\nwstartdeflinemarkup\nwprevnextdefs{NW4Nr7fb-3olr1Q-u}{NW4Nr7fb-3olr1Q-w}\nwenddeflinemarkup
\\def\\nw@underlinedefs\{% \{list with \\nwixd, \\nwixu\}
  \\let\\\\=\\relax\\def\\nw@comma\{, \}
  \\def\\nwixd##1\{\\\\\\underline\{\\subpageref\{##1\}\}\\let\\\\\\nw@comma\}%
  \\def\\nwixu##1\{\\\\\\subpageref\{##1\}\\let\\\\\\nw@comma\}\}

\\def\\nw@indexline#1#2\{%
   \{\\indent \{\\Tt #1\}: \\nw@underlinedefs\\@nameuse\{nwixl@#2\}\\par\}\}

\\newenvironment\{thenowebindex\}\{\\parindent=-10pt \\parskip=\\z@ 
        \\advance\\leftskip by 10pt 
        \\advance\\rightskip by 0pt plus1in\\par\\@afterindenttrue
    \\def\\\\##1\{\\nw@indexline##1\}\}\{\}
\nwendcode{}\nwbegindocs{196}\nwdocspar
The information comes from the list {\Tt{}nwisx@i\nwendquote}.
\nwenddocs{}\nwbegincode{197}\sublabel{NW4Nr7fb-3olr1Q-w}\nwmargintag{{\nwtagstyle{}\subpageref{NW4Nr7fb-3olr1Q-w}}}\moddef{noweb.sty~{\nwtagstyle{}\subpageref{NW4Nr7fb-3olr1Q-1}}}\plusendmoddef\nwstartdeflinemarkup\nwprevnextdefs{NW4Nr7fb-3olr1Q-v}{NW4Nr7fb-3olr1Q-x}\nwenddeflinemarkup
\\def\\nowebindex\{%
  \\@ifundefined\{nwixs@i\}%
     \{\\@warning\{The \\string\\nowebindex\\space is empty\}\}%
     \{\\begin\{thenowebindex\}\\@nameuse\{nwixs@i\}\\end\{thenowebindex\}\}\}
\nwendcode{}\nwbegindocs{198}\nwdocspar
Here's a more efficient version for the external case:
\nwenddocs{}\nwbegincode{199}\sublabel{NW4Nr7fb-3olr1Q-x}\nwmargintag{{\nwtagstyle{}\subpageref{NW4Nr7fb-3olr1Q-x}}}\moddef{noweb.sty~{\nwtagstyle{}\subpageref{NW4Nr7fb-3olr1Q-1}}}\plusendmoddef\nwstartdeflinemarkup\nwprevnextdefs{NW4Nr7fb-3olr1Q-w}{NW4Nr7fb-3olr1Q-y}\nwenddeflinemarkup
\\def\\nowebindex@external\{%
  \{\\let\\nwixadds@c=\\@gobble
   \\def\\nwixadds@i##1\{\\nw@indexline##1\}%
   \\def\\nwixaddsx##1##2\{\\@nameuse\{nwixadds@##1\}\{##2\}\}%
   \\begin\{thenowebindex\}\\@input\{\\jobname.nwi\}\\end\{thenowebindex\}\}\}
\nwendcode{}\nwbegindocs{200}\nwdocspar
That list ({\Tt{}nwisx@i\nwendquote}) is created by calls to {\Tt{}{\nwbackslash}nwixlogsorted{\nwlbrace}i{\nwrbrace}\nwendquote}.
\nwenddocs{}\nwbegincode{201}\sublabel{NW4Nr7fb-3olr1Q-y}\nwmargintag{{\nwtagstyle{}\subpageref{NW4Nr7fb-3olr1Q-y}}}\moddef{noweb.sty~{\nwtagstyle{}\subpageref{NW4Nr7fb-3olr1Q-1}}}\plusendmoddef\nwstartdeflinemarkup\nwprevnextdefs{NW4Nr7fb-3olr1Q-x}{NW4Nr7fb-3olr1Q-z}\nwenddeflinemarkup
\\def\\nwixlogsorted#1#2\{% list data
   \\@bsphack\\if@filesw 
     \\toks0=\{#2\}\\immediate\\write\\@auxout\{\\string\\nwixadds\{#1\}\{\\the\\toks0\}\}
   \\if@nobreak \\ifvmode\\nobreak\\fi\\fi\\fi\\@esphack\}
\nwendcode{}\nwbegindocs{202}\nwdocspar
{\Tt{}nwixs@c\nwendquote} and {\Tt{}nwixs@i\nwendquote} are sorted lists of chunks and identifiers, respectively.
\nwenddocs{}\nwbegincode{203}\sublabel{NW4Nr7fb-3olr1Q-z}\nwmargintag{{\nwtagstyle{}\subpageref{NW4Nr7fb-3olr1Q-z}}}\moddef{noweb.sty~{\nwtagstyle{}\subpageref{NW4Nr7fb-3olr1Q-1}}}\plusendmoddef\nwstartdeflinemarkup\nwprevnextdefs{NW4Nr7fb-3olr1Q-y}{NW4Nr7fb-3olr1Q-10}\nwenddeflinemarkup
\\def\\nwixadds#1#2\{%
  \\@ifundefined\{nwixs@#1\}%
    \{\\global\\@namedef\{nwixs@#1\}\{\\\\\{#2\}\}\}%
    \{\\expandafter\\nwix@cons\\csname nwixs@#1\\endcsname\{\\\\\{#2\}\}\}\}
\\let\\nwixaddsx=\\@gobbletwo
\nwendcode{}\nwbegindocs{204}\nwdocspar
If an external index is used, we need a {\Tt{}.nwi\nwendquote} file,
{\Tt{}{\nwbackslash}nwixadds\nwendquote} is to be ignored, and we use {\Tt{}{\nwbackslash}nwixaddsx\nwendquote}.
\nwenddocs{}\nwbegincode{205}\sublabel{NW4Nr7fb-3olr1Q-10}\nwmargintag{{\nwtagstyle{}\subpageref{NW4Nr7fb-3olr1Q-10}}}\moddef{noweb.sty~{\nwtagstyle{}\subpageref{NW4Nr7fb-3olr1Q-1}}}\plusendmoddef\nwstartdeflinemarkup\nwprevnextdefs{NW4Nr7fb-3olr1Q-z}{NW4Nr7fb-3olr1Q-11}\nwenddeflinemarkup
\\def\\nwopt@externalindex\{%
  \\ifx\\nwixadds\\@gobbletwo % already called
  \\else
    \\let\\nwixaddsx=\\nwixadds \\let\\nwixadds=\\@gobbletwo
    \\let\\nowebindex=\\nowebindex@external
    \\let\\nowebchunks=\\nowebchunks@external
  \\fi\}
\nwendcode{}\nwbegindocs{206}\nwdocspar
\iffalse
\nwenddocs{}\nwbegincode{207}\sublabel{NW4Nr7fb-3RhSlV-A}\nwmargintag{{\nwtagstyle{}\subpageref{NW4Nr7fb-3RhSlV-A}}}\moddef{man page: \code{}{\nwbackslash}noweboptions\edoc{}~{\nwtagstyle{}\subpageref{NW4Nr7fb-3RhSlV-1}}}\plusendmoddef\nwstartdeflinemarkup\nwprevnextdefs{NW4Nr7fb-3RhSlV-9}{NW4Nr7fb-3RhSlV-B}\nwenddeflinemarkup
.TP
.B externalindex
Use an index generated with 
.I noindex(1)
(q.v.).
\nwendcode{}\nwbegindocs{208}\fi
\nwenddocs{}\nwbegindocs{209}\nwdocspar
\nwenddocs{}\nwbegincode{210}\sublabel{NW4Nr7fb-3olr1Q-11}\nwmargintag{{\nwtagstyle{}\subpageref{NW4Nr7fb-3olr1Q-11}}}\moddef{noweb.sty~{\nwtagstyle{}\subpageref{NW4Nr7fb-3olr1Q-1}}}\plusendmoddef\nwstartdeflinemarkup\nwprevnextdefs{NW4Nr7fb-3olr1Q-10}{NW4Nr7fb-3olr1Q-12}\nwenddeflinemarkup
\\def\\nowebchunks\{%
  \\@ifundefined\{nwixs@c\}%
     \{\\@warning\{The are no \\string\\nowebchunks\}\}%
     \{\\begin\{thenowebchunks\}\\@nameuse\{nwixs@c\}\\end\{thenowebchunks\}\}\}
\\def\\nowebchunks@external\{%
  \{\\let\\nwixadds@i=\\@gobble
   \\def\\nwixadds@c##1\{\\nw@onechunk##1\}%
   \\def\\nwixaddsx##1##2\{\\@nameuse\{nwixadds@##1\}\{##2\}\}%
   \\begin\{thenowebchunks\}\\@input\{\\jobname.nwi\}\\end\{thenowebchunks\}\}\}
    \\@namedef\{r@nw@notdef\}\{\{0\}\{(\\@nwlangdepnvd)\}\}
\nwendcode{}\nwbegincode{211}\sublabel{NW4Nr7fb-3olr1Q-12}\nwmargintag{{\nwtagstyle{}\subpageref{NW4Nr7fb-3olr1Q-12}}}\moddef{noweb.sty~{\nwtagstyle{}\subpageref{NW4Nr7fb-3olr1Q-1}}}\plusendmoddef\nwstartdeflinemarkup\nwprevnextdefs{NW4Nr7fb-3olr1Q-11}{NW4Nr7fb-3olr1Q-13}\nwenddeflinemarkup
\\def\\nw@chunkunderlinedefs\{% \{list of labels with \\nwixd, \\nwixu\}
  \\let\\\\=\\relax\\def\\nw@comma\{, \}
  \\def\\nwixd##1\{\\\\\\underline\{\\subpageref\{##1\}\}\\let\\\\\\nw@comma\}%
  \\def\\nwixu##1\{\\\\\\subpageref\{##1\}\\let\\\\\\nw@comma\}\}
\nwendcode{}\nwbegincode{212}\sublabel{NW4Nr7fb-3olr1Q-13}\nwmargintag{{\nwtagstyle{}\subpageref{NW4Nr7fb-3olr1Q-13}}}\moddef{noweb.sty~{\nwtagstyle{}\subpageref{NW4Nr7fb-3olr1Q-1}}}\plusendmoddef\nwstartdeflinemarkup\nwprevnextdefs{NW4Nr7fb-3olr1Q-12}{NW4Nr7fb-3olr1Q-14}\nwenddeflinemarkup
\\def\\nw@onechunk#1#2#3\{% \{name\}\{label of first definition\}\{list with \\nwixd, \\nwixu\}
  \\@ifundefined\{r@#2\}\{\}\{%
    \\indent\\LA #1~\{\\nwtagstyle\\subpageref\{#2\}\}\\RA 
    \\if@nwlongchunks\{~\\nw@chunkunderlinedefs#3\}\\fi\\par\}\}
\nwendcode{}\nwbegincode{213}\sublabel{NW4Nr7fb-3olr1Q-14}\nwmargintag{{\nwtagstyle{}\subpageref{NW4Nr7fb-3olr1Q-14}}}\moddef{noweb.sty~{\nwtagstyle{}\subpageref{NW4Nr7fb-3olr1Q-1}}}\plusendmoddef\nwstartdeflinemarkup\nwprevnextdefs{NW4Nr7fb-3olr1Q-13}{NW4Nr7fb-3olr1Q-15}\nwenddeflinemarkup
\\newenvironment\{thenowebchunks\}\{\\vskip3pt
  \\parskip=\\z@\\parindent=-10pt \\advance\\leftskip by 10pt
  \\advance\\rightskip by 0pt plus10pt \\@afterindenttrue
  \\def\\\\##1\{\\nw@onechunk##1\}\}\{\}
\nwendcode{}\nwbegincode{214}\sublabel{NW4Nr7fb-3olr1Q-15}\nwmargintag{{\nwtagstyle{}\subpageref{NW4Nr7fb-3olr1Q-15}}}\moddef{noweb.sty~{\nwtagstyle{}\subpageref{NW4Nr7fb-3olr1Q-1}}}\plusendmoddef\nwstartdeflinemarkup\nwprevnextdefs{NW4Nr7fb-3olr1Q-14}{NW4Nr7fb-3olr1Q-16}\nwenddeflinemarkup
\\newif\\if@nwlongchunks
\\@nwlongchunksfalse
\\let\\nwopt@longchunks\\@nwlongchunkstrue
\nwendcode{}\nwbegindocs{215}\iffalse
\nwenddocs{}\nwbegincode{216}\sublabel{NW4Nr7fb-3RhSlV-B}\nwmargintag{{\nwtagstyle{}\subpageref{NW4Nr7fb-3RhSlV-B}}}\moddef{man page: \code{}{\nwbackslash}noweboptions\edoc{}~{\nwtagstyle{}\subpageref{NW4Nr7fb-3RhSlV-1}}}\plusendmoddef\nwstartdeflinemarkup\nwprevnextdefs{NW4Nr7fb-3RhSlV-A}{NW4Nr7fb-3RhSlV-C}\nwenddeflinemarkup
.TP
.B longchunks
When expanding 
.B "\\\\\\\\nowebchunks,"
show page numbers of definitions and uses of each chunk.
\nwendcode{}\nwbegindocs{217}\fi

\nwenddocs{}\nwbegindocs{218}\nwdocspar
\subsection{Support for hypertext}
There are two sets of support for hypertext.
 Balasubramanian Narasimhan wrote initial support for \texttt{hyper.sty}.
\nwenddocs{}\nwbegincode{219}\sublabel{NW4Nr7fb-3olr1Q-16}\nwmargintag{{\nwtagstyle{}\subpageref{NW4Nr7fb-3olr1Q-16}}}\moddef{noweb.sty~{\nwtagstyle{}\subpageref{NW4Nr7fb-3olr1Q-1}}}\plusendmoddef\nwstartdeflinemarkup\nwprevnextdefs{NW4Nr7fb-3olr1Q-15}{NW4Nr7fb-3olr1Q-17}\nwenddeflinemarkup
\\newcommand\\@nw@hyper@ref\{\\hyperreference\} % naras
\\newcommand\\@nw@hyper@anc\{\\blindhyperanchor\} % naras
\nwendcode{}\nwbegindocs{220}Norman Ramsey wrote support for the \texttt{hyperrref} package (May
 1998).
\nwenddocs{}\nwbegincode{221}\sublabel{NW4Nr7fb-3olr1Q-17}\nwmargintag{{\nwtagstyle{}\subpageref{NW4Nr7fb-3olr1Q-17}}}\moddef{noweb.sty~{\nwtagstyle{}\subpageref{NW4Nr7fb-3olr1Q-1}}}\plusendmoddef\nwstartdeflinemarkup\nwprevnextdefs{NW4Nr7fb-3olr1Q-16}{NW4Nr7fb-3olr1Q-18}\nwenddeflinemarkup
\\newcommand\\@nw@hyperref@ref[2]\{\\hyperlink\{noweb.#1\}\{#2\}\}  % nr
\\newcommand\\@nw@hyperref@anc[1]\{\\hypertarget\{noweb.#1\}\{\\relax\}\}  % nr
%%\\renewcommand\\@nw@hyperref@ref[2]\{\{#2\}\}  % nr
%%\\renewcommand\\@nw@hyperref@anc[1]\{\}  % nr
\nwendcode{}\nwbegindocs{222}We define the independent macros {\Tt{}{\nwbackslash}nwhyperreference\nwendquote} and
 {\Tt{}{\nwbackslash}nwblindhyperanchor\nwendquote},
which test for the presence of one of these two packages, 
redefine themselve accordingly, and re-invoke themselves.
\nwenddocs{}\nwbegincode{223}\sublabel{NW4Nr7fb-3olr1Q-18}\nwmargintag{{\nwtagstyle{}\subpageref{NW4Nr7fb-3olr1Q-18}}}\moddef{noweb.sty~{\nwtagstyle{}\subpageref{NW4Nr7fb-3olr1Q-1}}}\plusendmoddef\nwstartdeflinemarkup\nwprevnextdefs{NW4Nr7fb-3olr1Q-17}{NW4Nr7fb-3olr1Q-19}\nwenddeflinemarkup
\\newcommand\\nwhyperreference\{%
  \\@ifundefined\{hyperlink\}
    \{\\@ifundefined\{hyperreference\}
       \{\\global\\let\\nwhyperreference\\@gobble\}
       \{\\global\\let\\nwhyperreference\\@nw@hyper@ref\}\}
    \{\\global\\let\\nwhyperreference\\@nw@hyperref@ref\}%
  \\nwhyperreference
\}

\\newcommand\\nwblindhyperanchor\{%
  \\@ifundefined\{hyperlink\}
    \{\\@ifundefined\{hyperreference\}
       \{\\global\\let\\nwblindhyperanchor\\@gobble\}
       \{\\global\\let\\nwblindhyperanchor\\@nw@hyper@anc\}\}
    \{\\global\\let\\nwblindhyperanchor\\@nw@hyperref@anc\}%
  \\nwblindhyperanchor
\}
\nwendcode{}\nwbegindocs{224}\nwdocspar
\subsection{Support for hypertext translation to HTML}
\nwenddocs{}\nwbegincode{225}\sublabel{NW4Nr7fb-3olr1Q-19}\nwmargintag{{\nwtagstyle{}\subpageref{NW4Nr7fb-3olr1Q-19}}}\moddef{noweb.sty~{\nwtagstyle{}\subpageref{NW4Nr7fb-3olr1Q-1}}}\plusendmoddef\nwstartdeflinemarkup\nwprevnextdefs{NW4Nr7fb-3olr1Q-18}{NW4Nr7fb-3olr1Q-1A}\nwenddeflinemarkup
\\newcommand\\nwanchorto\{%
  \\begingroup\\let\\do\\@makeother\\dospecials
     \\catcode`\\\{=1 \\catcode`\\\}=2 \\nw@anchorto\}
\\newcommand\\nw@anchorto[1]\{\\endgroup\\def\\nw@next\{#1\}\\nw@anchortofin\}
\\newcommand\\nw@anchortofin[1]\{#1\\footnote\{See URL \\texttt\{\\nw@next\}.\}\}
\\let\\nwanchorname\\@gobble
\nwendcode{}\nwbegincode{226}\sublabel{NW4Nr7fb-4PphKm-2}\nwmargintag{{\nwtagstyle{}\subpageref{NW4Nr7fb-4PphKm-2}}}\moddef{man page: noweb style control sequences~{\nwtagstyle{}\subpageref{NW4Nr7fb-4PphKm-1}}}\plusendmoddef\nwstartdeflinemarkup\nwprevnextdefs{NW4Nr7fb-4PphKm-1}{NW4Nr7fb-4PphKm-3}\nwenddeflinemarkup
.PP
.B "\\\\\\\\nwanchorto\{URL\}\{anchor text\}"
.RS
Creates a link to the given URL with the given anchor text.
Implemented in 
.I latex(1)
using footnotes, but 
.I sl2h(1)
translates this to
.B "<a href=URL>anchor text</a>"
.RE
.PP
.B "\\\\\\\\nwanchorname\{name\}\{anchor text\}"
.RS
Creates an anchor point for a hyperlink.
Implemented in 
.I latex(1)
using 
.B "\\\\\\\\label",
but
.I sl2h(1)
translates this to
.B "<a name=name>anchor text</a>"
.RE
\nwendcode{}\nwbegindocs{227}\nwdocspar
This lets us hide stuff intended for use only when converting to HTML:
\nwenddocs{}\nwbegincode{228}\sublabel{NW4Nr7fb-3olr1Q-1A}\nwmargintag{{\nwtagstyle{}\subpageref{NW4Nr7fb-3olr1Q-1A}}}\moddef{noweb.sty~{\nwtagstyle{}\subpageref{NW4Nr7fb-3olr1Q-1}}}\plusendmoddef\nwstartdeflinemarkup\nwprevnextdefs{NW4Nr7fb-3olr1Q-19}{NW4Nr7fb-3olr1Q-1B}\nwenddeflinemarkup
\\newif\\ifhtml
\\htmlfalse
\nwendcode{}\nwbegincode{229}\sublabel{NW4Nr7fb-4PphKm-3}\nwmargintag{{\nwtagstyle{}\subpageref{NW4Nr7fb-4PphKm-3}}}\moddef{man page: noweb style control sequences~{\nwtagstyle{}\subpageref{NW4Nr7fb-4PphKm-1}}}\plusendmoddef\nwstartdeflinemarkup\nwprevnextdefs{NW4Nr7fb-4PphKm-2}{\relax}\nwenddeflinemarkup
.PP
.B "\\\\\\\\ifhtml ... \\\\\\\\fi"
.RS
Text between
.B "\\\\\\\\ifhtml"
and
.B "\\\\\\\\fi"
is ignored by 
.I latex(1),
but 
.I sl2h(1)
and the 
.I l2h 
noweb filter translate the text into HTML.
.RE
\nwendcode{}\nwbegindocs{230}\nwdocspar
\subsection{Support for Prettyprinting}
The following macro can be redefined to allow custom typesetting of
identifiers in the index and mini-indices.
\nwenddocs{}\nwbegincode{231}\sublabel{NW4Nr7fb-3olr1Q-1B}\nwmargintag{{\nwtagstyle{}\subpageref{NW4Nr7fb-3olr1Q-1B}}}\moddef{noweb.sty~{\nwtagstyle{}\subpageref{NW4Nr7fb-3olr1Q-1}}}\plusendmoddef\nwstartdeflinemarkup\nwprevnextdefs{NW4Nr7fb-3olr1Q-1A}{NW4Nr7fb-3olr1Q-1C}\nwenddeflinemarkup
\\let\\nwixident=\\relax
\nwendcode{}\nwbegindocs{232}\nwdocspar
\nwenddocs{}\nwbegindocs{233}\nwdocspar
The following macros can be redefined to typeset `{\Tt{}{\nwbackslash}\nwendquote}', `{\Tt{}{\nwlbrace}\nwendquote}' and
`{\Tt{}{\nwrbrace}\nwendquote}' correctly in non-typewriter fonts. 
The problem is that the built-in {\LaTeX} {\Tt{}{\nwbackslash}{\nwlbrace}\nwendquote} tries to produce a
math symbol, which doesn't exist in the typewriter font, so we get a
brace in the wrong font and a warning.  Most unpleasant.
Noweave therefore attempts to emit {\Tt{}{\nwbackslash}nwlbrace\nwendquote} and {\Tt{}{\nwbackslash}nwrbrace\nwendquote}
wherever it believes braces should appear.
The standard noweb style is to set code in typewriter font, and so the
standard definitions just select the proper characters from that font.
 People setting code in
fonts other than typewriter are responsible for redefining those
macros to work in their environment.
\nwenddocs{}\nwbegincode{234}\sublabel{NW4Nr7fb-eb66b-B}\nwmargintag{{\nwtagstyle{}\subpageref{NW4Nr7fb-eb66b-B}}}\moddef{kernel~{\nwtagstyle{}\subpageref{NW4Nr7fb-eb66b-1}}}\plusendmoddef\nwstartdeflinemarkup\nwusesondefline{\\{NW4Nr7fb-3olr1Q-5}\\{NW4Nr7fb-38jgpJ-D}}\nwprevnextdefs{NW4Nr7fb-eb66b-A}{\relax}\nwenddeflinemarkup
\\def\\nwbackslash\{\\char92\}
\\def\\nwlbrace\{\\char123\}
\\def\\nwrbrace\{\\char125\}
\nwused{\\{NW4Nr7fb-3olr1Q-5}\\{NW4Nr7fb-38jgpJ-D}}\nwendcode{}\nwbegindocs{235}\nwdocspar
\subsection{Language-dependent macros} 

Miguel Filgueiras
(DCC-FCUP \& LIACC, Universidade do Porto) provided some changes to
add multilingual support for the words Noweb uses in indexing and
cross-reference.
He inserted macros that are defined by, e.g., {\Tt{}{\nwbackslash}noweboptions{\nwlbrace}english{\nwrbrace}\nwendquote}.
The Noweb package uses the
(apparently standard) {\LaTeX} macro {\Tt{}{\nwbackslash}languagename\nwendquote} to select a
language at load time.
If the \texttt{babel} package is
loaded (with the appropriate language name) before Noweb is loaded, 
the Noweb package will select language appropriately, provided the
language is one of those Noweb supports.
Mr.~Filgueiras provided support for English,
Portuguese, German, and French. 
He notes that the French is
faulty; the translations may be poor, and there are
bugs in the implementation that he could not solve.

\label{subsection:langdeps}

The language-dependent macros are defined here in each supported
language in a different subsubsection.

The choice of language depends on testing the {\Tt{}{\nwbackslash}languagename\nwendquote}
macro. There must be a more elegant way of coding the tests below\ldots
\nwenddocs{}\nwbegincode{236}\sublabel{NW4Nr7fb-3olr1Q-1C}\nwmargintag{{\nwtagstyle{}\subpageref{NW4Nr7fb-3olr1Q-1C}}}\moddef{noweb.sty~{\nwtagstyle{}\subpageref{NW4Nr7fb-3olr1Q-1}}}\plusendmoddef\nwstartdeflinemarkup\nwprevnextdefs{NW4Nr7fb-3olr1Q-1B}{\relax}\nwenddeflinemarkup
\LA{}language support~{\nwtagstyle{}\subpageref{NW4Nr7fb-2c2IOz-1}}\RA{}
\\ifx\\languagename\\undefined % default is English
  \\noweboptions\{english\}
\\else
  \\@ifundefined\{nwopt@\\languagename\}
     \{\\noweboptions\{english\}\}
     \{\\expandafter\\noweboptions\\expandafter\{\\languagename\}\}
\\fi
\nwendcode{}\nwbegindocs{237}\nwdocspar
\subsubsection{Support for English}

This describes the original English text.
\nwenddocs{}\nwbegincode{238}\sublabel{NW4Nr7fb-2c2IOz-1}\nwmargintag{{\nwtagstyle{}\subpageref{NW4Nr7fb-2c2IOz-1}}}\moddef{language support~{\nwtagstyle{}\subpageref{NW4Nr7fb-2c2IOz-1}}}\endmoddef\nwstartdeflinemarkup\nwusesondefline{\\{NW4Nr7fb-3olr1Q-1C}}\nwprevnextdefs{\relax}{NW4Nr7fb-2c2IOz-2}\nwenddeflinemarkup
\\def\\nwopt@english\{%
  \\def\\@nwlangdepdef\{This definition is continued\}%
  \\def\\@nwlangdepcud\{This code is used\}%
  \\def\\@nwlangdeprtc\{Root chunk (not used in this document)\}%
  \\def\\@nwlangdepcwf\{This code is written to file\}%
  \\def\\@nwlangdepchk\{chunk\}%
  \\def\\@nwlangdepchks\{chunks\}%
  \\def\\@nwlangdepin\{in\}%
  \\def\\@nwlangdepand\{and\}%
  \\def\\@nwlangdepuss\{Uses\}%
  \\def\\@nwlangdepusd\{used\}%
  \\def\\@nwlangdepnvu\{never used\}%
  \\def\\@nwlangdepdfs\{Defines\}%
  \\def\\@nwlangdepnvd\{never defined\}%
\}
\\let\\nwopt@american\\nwopt@english
\nwalsodefined{\\{NW4Nr7fb-2c2IOz-2}\\{NW4Nr7fb-2c2IOz-3}\\{NW4Nr7fb-2c2IOz-4}\\{NW4Nr7fb-2c2IOz-5}\\{NW4Nr7fb-2c2IOz-6}}\nwused{\\{NW4Nr7fb-3olr1Q-1C}}\nwendcode{}\nwbegindocs{239}\nwdocspar

\nwenddocs{}\nwbegindocs{240}\nwdocspar
\subsubsection{Partial support for Icelandic}

Partial translation of noweb into Icelandic courtesy of
Johann ``Myrkraverk'' Oskarsson" \verb+<johann@myrkraverk.com>+.
\nwenddocs{}\nwbegincode{241}\sublabel{NW4Nr7fb-2c2IOz-2}\nwmargintag{{\nwtagstyle{}\subpageref{NW4Nr7fb-2c2IOz-2}}}\moddef{language support~{\nwtagstyle{}\subpageref{NW4Nr7fb-2c2IOz-1}}}\plusendmoddef\nwstartdeflinemarkup\nwusesondefline{\\{NW4Nr7fb-3olr1Q-1C}}\nwprevnextdefs{NW4Nr7fb-2c2IOz-1}{NW4Nr7fb-2c2IOz-3}\nwenddeflinemarkup
\\def\\nwopt@icelandic\{%
  \\def\\@nwlangdepdef\{This definition is continued\}%
  \\def\\@nwlangdepcud\{This code is used\}%
  \\def\\@nwlangdeprtc\{Root chunk (not used in this document)\}%
  \\def\\@nwlangdepcwf\{This code is written to file\}%
  \\def\\@nwlangdepchk\{k��a\}%
  \\def\\@nwlangdepchks\{k��um\}%
  \\def\\@nwlangdepin\{�\}%
  \\def\\@nwlangdepand\{og\}%
  \\def\\@nwlangdepuss\{Notar\}%
  \\def\\@nwlangdepusd\{nota�\}%
  \\def\\@nwlangdepnvu\{hvergi nota�\}%
  \\def\\@nwlangdepdfs\{Skilgreinir\}%
  \\def\\@nwlangdepnvd\{hvergi skilgreint\}%
\}
\nwused{\\{NW4Nr7fb-3olr1Q-1C}}\nwendcode{}\nwbegindocs{242}\nwdocspar
\subsubsection{Support for Portuguese}

This contains the text in Portuguese.
\nwenddocs{}\nwbegincode{243}\sublabel{NW4Nr7fb-2c2IOz-3}\nwmargintag{{\nwtagstyle{}\subpageref{NW4Nr7fb-2c2IOz-3}}}\moddef{language support~{\nwtagstyle{}\subpageref{NW4Nr7fb-2c2IOz-1}}}\plusendmoddef\nwstartdeflinemarkup\nwusesondefline{\\{NW4Nr7fb-3olr1Q-1C}}\nwprevnextdefs{NW4Nr7fb-2c2IOz-2}{NW4Nr7fb-2c2IOz-4}\nwenddeflinemarkup
\\def\\nwopt@portuges\{%
  \\def\\@nwlangdepdef\{Defini\\c\{c\}\\~ao continuada em\}%
  % This definition is continued
  \\def\\@nwlangdepcud\{C\\'odigo usado em\}%
  % This code is used
  \\def\\@nwlangdeprtc\{Fragmento de topo (sem uso no documento)\}%
  % Root chunk (not used in this document)
  \\def\\@nwlangdepcwf\{Este c\\'odigo foi escrito no ficheiro\}%
  % This code is written to file
  \\def\\@nwlangdepchk\{fragmento\}%
  % chunk
  \\def\\@nwlangdepchks\{fragmentos\}%
  % chunks
  \\def\\@nwlangdepin\{no(s)\}%
  % in
  \\def\\@nwlangdepand\{e\}%
  % and
  \\def\\@nwlangdepuss\{Usa\}%
  % Uses
  \\def\\@nwlangdepusd\{usado\}%
  % used
  \\def\\@nwlangdepnvu\{nunca usado\}%
  % never used
  \\def\\@nwlangdepdfs\{Define\}%
  % Defines
  \\def\\@nwlangdepnvd\{nunca definido\}%
  % never defined
\}
\nwused{\\{NW4Nr7fb-3olr1Q-1C}}\nwendcode{}\nwbegindocs{244}\nwdocspar
\subsubsection{Support for French}

This is a tentative translation to French.
NR has made some corrections, and it has been reviewed by Fr\'ed\'eric
Lin\'e, but some errors may remain.

There are problems with using accents: on the {\Tt{}{\nwbackslash}@nwlangdepnvd\nwendquote} macro
(which apparently is not used in the context of {\Tt{}{\nwbackslash}nwcodecomment\nwendquote}),
and in some other macros (\LaTeX{} complains about missing
{\Tt{}{\nwbackslash}endcsname\nwendquote}). This should be fixed by someone with experience in
using \TeX\ldots
\nwenddocs{}\nwbegincode{245}\sublabel{NW4Nr7fb-2c2IOz-4}\nwmargintag{{\nwtagstyle{}\subpageref{NW4Nr7fb-2c2IOz-4}}}\moddef{language support~{\nwtagstyle{}\subpageref{NW4Nr7fb-2c2IOz-1}}}\plusendmoddef\nwstartdeflinemarkup\nwusesondefline{\\{NW4Nr7fb-3olr1Q-1C}}\nwprevnextdefs{NW4Nr7fb-2c2IOz-3}{NW4Nr7fb-2c2IOz-5}\nwenddeflinemarkup
\\def\\nwopt@frenchb\{%
  \\def\\@nwlangdepdef\{Suite de la d\\'efinition\}%
  % This definition is continued
  \\def\\@nwlangdepcud\{Ce code est employ\\'e\}%
  % This code is used
  \\def\\@nwlangdeprtc\{Morceau racine (pas employ\\'e dans ce document)\}%
  % Root chunk (not used in this document)
  \\def\\@nwlangdepcwf\{Ce code est \\'ecrit dans le fichier\}%
  % This code is written to file
  \\def\\@nwlangdepchk\{le morceau\}%
  % chunk
  \\def\\@nwlangdepchks\{les morceaux\}%
  % chunks
  \\def\\@nwlangdepin\{dans\}%
  % in
  \\def\\@nwlangdepand\{et\}%
  % and
  \\def\\@nwlangdepuss\{Utilise\}%
  % Uses
  \\def\\@nwlangdepusd\{utilis\\'\{e\}\}%
  % used
  \\def\\@nwlangdepnvu\{jamais employ\\'\{e\}\}%
  % never used
  \\def\\@nwlangdepdfs\{D\\'\{e\}finit\}%
  % Defines
  % Cannot use the accent here: \\def\\@nwlangdepnvd\{jamais d\\'\{e\}fini\}%
  \\def\\@nwlangdepnvd\{jamais defini\}%
  % never defined
\}
\\let\\nwopt@french\\nwopt@frenchb
\nwused{\\{NW4Nr7fb-3olr1Q-1C}}\nwendcode{}\nwbegindocs{246}\nwdocspar
\subsubsection{Support for German}

This is a translation to German by Sabine Broda (DCC-FCUP \& LIACC,
Universidade do Porto) with revisions by Christian Lindig and further
revisions by Pascal Schmitt.
\nwenddocs{}\nwbegincode{247}\sublabel{NW4Nr7fb-2c2IOz-5}\nwmargintag{{\nwtagstyle{}\subpageref{NW4Nr7fb-2c2IOz-5}}}\moddef{language support~{\nwtagstyle{}\subpageref{NW4Nr7fb-2c2IOz-1}}}\plusendmoddef\nwstartdeflinemarkup\nwusesondefline{\\{NW4Nr7fb-3olr1Q-1C}}\nwprevnextdefs{NW4Nr7fb-2c2IOz-4}{NW4Nr7fb-2c2IOz-6}\nwenddeflinemarkup
\\def\\nwopt@german\{%
  \\def\\@nwlangdepdef\{Diese Definition wird fortgesetzt\}%
  % This definition is continued
  \\def\\@nwlangdepcud\{Dieser Code wird benutzt\}%
  % This code is used
  \\def\\@nwlangdeprtc\{Hauptteil (nicht in diesem Dokument benutzt)\}%
  % Root chunk (not used in this document)
  \\def\\@nwlangdepcwf\{Geh\\"ort in die Datei\}%
  % This code is written to file
  \\def\\@nwlangdepchk\{Abschnitt\}%
  % chunk
  \\def\\@nwlangdepchks\{den Abschnitten\}%
  % chunks
  \\def\\@nwlangdepin\{in\}%
  % in
  \\def\\@nwlangdepand\{und\}%
  % and
  \\def\\@nwlangdepuss\{Benutzt\}%
  % Uses
  \\def\\@nwlangdepusd\{benutzt\}%
  % used
  \\def\\@nwlangdepnvu\{nicht benutzt\}%
  % never used
  \\def\\@nwlangdepdfs\{Definiert\}%
  % Defines
  \\def\\@nwlangdepnvd\{nicht definiert\}%
  % never defined
\}
\nwused{\\{NW4Nr7fb-3olr1Q-1C}}\nwendcode{}\nwbegindocs{248}\nwdocspar
Here is a revised version, because the version above did not satisy
(explanations below).
\nwenddocs{}\nwbegincode{249}\sublabel{NW4Nr7fb-2c2IOz-6}\nwmargintag{{\nwtagstyle{}\subpageref{NW4Nr7fb-2c2IOz-6}}}\moddef{language support~{\nwtagstyle{}\subpageref{NW4Nr7fb-2c2IOz-1}}}\plusendmoddef\nwstartdeflinemarkup\nwusesondefline{\\{NW4Nr7fb-3olr1Q-1C}}\nwprevnextdefs{NW4Nr7fb-2c2IOz-5}{\relax}\nwenddeflinemarkup
\\def\\nwopt@german\{%
  \\def\\@nwlangdepdef\{Diese Definition wird fortgesetzt\}%
  % This definition is continued
  \\def\\@nwlangdepcud\{Dieser Code wird benutzt\}%
  % This code is used
  \\def\\@nwlangdeprtc\{Hauptteil (nicht in diesem Dokument benutzt)\}%
  % Root chunk (not used in this document)
  \\def\\@nwlangdepcwf\{Dieser Code erzeugt die Datei\}
  % This code generates the file
  \\def\\@nwlangdepchk\{Teil\}%
  % chunk
  \\def\\@nwlangdepchks\{Teils\}%
  % chunks
  \\def\\@nwlangdepin\{im\}%
  % in
  \\def\\@nwlangdepand\{und\}%
  % and
  \\def\\@nwlangdepuss\{Benutzt\}%
  % Uses
  \\def\\@nwlangdepusd\{benutzt\}%
  % used
  \\def\\@nwlangdepnvu\{nicht benutzt\}%
  % never used
  \\def\\@nwlangdepdfs\{Definiert\}%
  % Defines
  \\def\\@nwlangdepnvd\{nicht definiert\}%
  % never defined
\}
\\let\\nwopt@ngerman\\nwopt@german
\nwused{\\{NW4Nr7fb-3olr1Q-1C}}\nwendcode{}\nwbegindocs{250}Explanations:
\begin{quote}
Changed the {\Tt{}{\nwbackslash}@nwlangdepcwf\nwendquote} yet again.
Johannes Wiedersich \url{<johannes@physik.blm.tu-muenchen.de>} writes:
\begin{quote}
The German latex files contain
\begin{quote} 
Dieser Code schreibt man zum File
\end{quote}
this is not correct (or understandable) German and not a correct
translation of the English ``This code is written to file.''
A better translation would be
\begin{quote} 
Dieser Code geht in Datei
\end{quote}
I would suggest to use either
\begin{itemize} 
\item Dieser Code schreibt den File 
\item Dieser Code erzeugt die Datei 
\end{itemize}
The first is closer to a literal translation of the English text; the
second option is closer to the \emph{meaning} of the English text and to 
what
actually happens. (This code generates the file)
\end{quote}
The old explanation was:
\begin{quote} 
``Diese Code geht in Datei (x)'' isn't perfect either; the best
translation would be ``Dieser Code wird in Datei (x) geschrieben'' but
this would require to put the file name into the middle of the sentence.
\begin{verbatim} 
    \def\@nwlangdepcwf{Dieser Code geht in Datei}%
\end{verbatim}
The {\Tt{}{\nwbackslash}@nwlangdepchks\nwendquote} macro is problematic because it has to work
together with {\Tt{}{\nwbackslash}@nwlangdepin\nwendquote}. It would be best to emit:
\begin{verbatim} 
    benutzt in den Teilen 1b und 2.
    benutzt im Teil 1.
\end{verbatim}
This would require to switch between ``in'' and ``im''. Since ``im'' is a
compund of ``in dem'', I suggest to use an article for the singular, too,
and to emit:
\begin{verbatim} 
    benutzt in den Teilen 1b und 2.
    benutzt in dem Teil 1.
\end{verbatim}
Hence:
\begin{verbatim} 
    \def\@nwlangdepchks{den Teilen}
    \def\@nwlangdepchk{dem Teil}
\end{verbatim}
\end{quote}
\end{quote}
\nwenddocs{}\nwbegindocs{251}\nwdocspar
\nwenddocs{}\nwbegindocs{252}\iffalse
\nwenddocs{}\nwbegincode{253}\sublabel{NW4Nr7fb-3RhSlV-C}\nwmargintag{{\nwtagstyle{}\subpageref{NW4Nr7fb-3RhSlV-C}}}\moddef{man page: \code{}{\nwbackslash}noweboptions\edoc{}~{\nwtagstyle{}\subpageref{NW4Nr7fb-3RhSlV-1}}}\plusendmoddef\nwstartdeflinemarkup\nwprevnextdefs{NW4Nr7fb-3RhSlV-B}{\relax}\nwenddeflinemarkup
.TP
.B english, french, german, portuges, icelandic
Write cross-reference information in the language specified.
Defaults to 
.B english.
\nwendcode{}\nwbegindocs{254}\fi
\nwenddocs{}\nwbegindocs{255}\nwdocspar
\clearpage
\section{The {\tt nwmac} macros for use with plain {\TeX}}

First we make {\Tt{}@\nwendquote} a letter so that we can use `private' macro names.
\nwenddocs{}\nwbegincode{256}\sublabel{NW4Nr7fb-38jgpJ-5}\nwmargintag{{\nwtagstyle{}\subpageref{NW4Nr7fb-38jgpJ-5}}}\moddef{nwmac.tex~{\nwtagstyle{}\subpageref{NW4Nr7fb-38jgpJ-1}}}\plusendmoddef\nwstartdeflinemarkup\nwprevnextdefs{NW4Nr7fb-38jgpJ-4}{NW4Nr7fb-38jgpJ-6}\nwenddeflinemarkup
\\catcode`\\@=11
\nwendcode{}\nwbegincode{257}\sublabel{NW4Nr7fb-38jgpJ-6}\nwmargintag{{\nwtagstyle{}\subpageref{NW4Nr7fb-38jgpJ-6}}}\moddef{nwmac.tex~{\nwtagstyle{}\subpageref{NW4Nr7fb-38jgpJ-1}}}\plusendmoddef\nwstartdeflinemarkup\nwprevnextdefs{NW4Nr7fb-38jgpJ-5}{NW4Nr7fb-38jgpJ-7}\nwenddeflinemarkup
% scale cmbx10 instead of using cmbx12 because \{\\LaTeX\} does, so fonts exist
\\font\\twlbf=cmbx10 scaled \\magstep1
\\font\\frtbf=cmbx10 scaled \\magstep2
% These fonts don't work with xdvi!

\\advance\\hoffset 0.5 true in
\\advance\\hsize -1.5 true in
\\newdimen\\textsize
\\textsize=\\hsize
\\def\\today\{\\ifcase\\month\\or
  January\\or February\\or March\\or April\\or May\\or June\\or
  July\\or August\\or September\\or October\\or November\\or December\\fi
  \\space\\number\\day, \\number\\year\}
\nwendcode{}\nwbegincode{258}\sublabel{NW4Nr7fb-38jgpJ-7}\nwmargintag{{\nwtagstyle{}\subpageref{NW4Nr7fb-38jgpJ-7}}}\moddef{nwmac.tex~{\nwtagstyle{}\subpageref{NW4Nr7fb-38jgpJ-1}}}\plusendmoddef\nwstartdeflinemarkup\nwprevnextdefs{NW4Nr7fb-38jgpJ-6}{NW4Nr7fb-38jgpJ-8}\nwenddeflinemarkup
\\long\\def\\ifundefined#1#2#3\{%
   \\expandafter\\ifx\\csname#1\\endcsname\\relax
       #2%
   \\else#3%
   \\fi\}

\\ifundefined\{myheadline\}
    \{\\headline=\{\\hbox to \\textsize\{\\tentt\\firstmark\\hfil\\tenrm\\today\\hbox
                to 4em\{\\hss\\folio\}\}\\hss\}\}
    \{\\expandafter\\headline\\expandafter\{\\myheadline\}\}

\\ifundefined\{myfootline\}
    \{\\footline=\{\\hfil\}\}
    \{\\expandafter\\footline\\expandafter\{\\myfootline\}\}
\nwendcode{}\nwbegincode{259}\sublabel{NW4Nr7fb-38jgpJ-8}\nwmargintag{{\nwtagstyle{}\subpageref{NW4Nr7fb-38jgpJ-8}}}\moddef{nwmac.tex~{\nwtagstyle{}\subpageref{NW4Nr7fb-38jgpJ-1}}}\plusendmoddef\nwstartdeflinemarkup\nwprevnextdefs{NW4Nr7fb-38jgpJ-7}{NW4Nr7fb-38jgpJ-9}\nwenddeflinemarkup
\\def\\semifilbreak\{\\vskip0pt plus1.5in\\penalty-200\\vskip0pt plus -1.5in\}
\\raggedbottom
\nwendcode{}\nwbegincode{260}\sublabel{NW4Nr7fb-38jgpJ-9}\nwmargintag{{\nwtagstyle{}\subpageref{NW4Nr7fb-38jgpJ-9}}}\moddef{nwmac.tex~{\nwtagstyle{}\subpageref{NW4Nr7fb-38jgpJ-1}}}\plusendmoddef\nwstartdeflinemarkup\nwprevnextdefs{NW4Nr7fb-38jgpJ-8}{NW4Nr7fb-38jgpJ-A}\nwenddeflinemarkup
%
% \\chapcenter macro to produce nice centered chapter titles
%
\\def\\chapcenter\{\\leftskip=0.5 true in plus 4em minus 0.5 true in
    \\rightskip=\\leftskip
    \\parfillskip=0pt \\spaceskip=.3333em \\xspaceskip=.5em
    \\pretolerance=9999 \\tolerance=9999
    \\hyphenpenalty=9999 \\exhyphenpenalty=9999\}
\nwendcode{}\nwbegincode{261}\sublabel{NW4Nr7fb-38jgpJ-A}\nwmargintag{{\nwtagstyle{}\subpageref{NW4Nr7fb-38jgpJ-A}}}\moddef{nwmac.tex~{\nwtagstyle{}\subpageref{NW4Nr7fb-38jgpJ-1}}}\plusendmoddef\nwstartdeflinemarkup\nwprevnextdefs{NW4Nr7fb-38jgpJ-9}{NW4Nr7fb-38jgpJ-B}\nwenddeflinemarkup
% \\startsection\{LEVEL\}\{INDENT\}\{BEFORESKIP\}\{AFTERSKIP\}\{STYLE\}\{HEADING\}
%               #1     #2      #3          #4         #5     #6
%
%       LEVEL:          depth; e.g. part=0 chapter=1 sectino=2...
%       INDENT:         indentation of heading from left margin
%       BEFORESKIP:     skip before header
%       AFTERSKIP:      skip after header
%       STYLE:          style of heading; e.g.\\bf
%       HEADING:        heading of the sectino
%
\\def\\startsection#1#2#3#4#5#6\{\\par\\vskip#3 plus 2in
        \\penalty-200\\vskip 0pt plus -2in
    \\noindent\{\\leftskip=#2 \\rightskip=0.5true in plus 4em minus 0.5 true in
              \\hyphenpenalty=9999 \\exhyphenpenalty=9999
              #5#6\\par\}\\vskip#4%
    \{\\def\\code##1\{[[\}\\def\\edoc##1\{]]\}\\message\{[#6]\}\}
    \\settocparms\{#1\}
    \\def\\themodtitle\{#6\}
%%%%    \{\\def\\code\{\\string\\code\}\\def\\edoc\{\\string\\edoc\}%
    \\edef\\next\{\\noexpand\\write\\cont\{\\tocskip
        \\tocline\{\\hskip\\tocindent\\tocstyle\\relax\\themodtitle\}
                \{\\noexpand\\the\\pageno\}\}\}\\next % write to toc
    %\}
\}
\nwendcode{}\nwbegincode{262}\sublabel{NW4Nr7fb-38jgpJ-B}\nwmargintag{{\nwtagstyle{}\subpageref{NW4Nr7fb-38jgpJ-B}}}\moddef{nwmac.tex~{\nwtagstyle{}\subpageref{NW4Nr7fb-38jgpJ-1}}}\plusendmoddef\nwstartdeflinemarkup\nwprevnextdefs{NW4Nr7fb-38jgpJ-A}{NW4Nr7fb-38jgpJ-C}\nwenddeflinemarkup
\\def\\settocparms#1\{
        \\count@=#1
        \\ifnum\\count@<1
            \\def\\tocskip\{\\vskip3ptplus1in\\penalty-100
                        \\vskip0ptplus-1in\}%
            \\def\\tocstyle\{\\bf\}
            \\def\\tocindent\{0pt\}
        \\else
            \\def\\tocskip\{\}
            \\def\\tocstyle\{\\rm\}
            \\dimen@=2em \\advance\\count@ by \\m@ne \\dimen@=\\count@\\dimen@
            \\edef\\tocindent\{\\the\\dimen@\}
        \\fi
\}
\nwendcode{}\nwbegincode{263}\sublabel{NW4Nr7fb-38jgpJ-C}\nwmargintag{{\nwtagstyle{}\subpageref{NW4Nr7fb-38jgpJ-C}}}\moddef{nwmac.tex~{\nwtagstyle{}\subpageref{NW4Nr7fb-38jgpJ-1}}}\plusendmoddef\nwstartdeflinemarkup\nwprevnextdefs{NW4Nr7fb-38jgpJ-B}{NW4Nr7fb-38jgpJ-D}\nwenddeflinemarkup
\\def\\tocline#1#2\{\\line\{\{\\ignorespaces#1\}\\leaders\\hbox to .5em\{.\\hfil\}\\hfil
     \\hbox to1.5em\{\\hss#2\}\}\}
\nwendcode{}\nwbegincode{264}\sublabel{NW4Nr7fb-38jgpJ-D}\nwmargintag{{\nwtagstyle{}\subpageref{NW4Nr7fb-38jgpJ-D}}}\moddef{nwmac.tex~{\nwtagstyle{}\subpageref{NW4Nr7fb-38jgpJ-1}}}\plusendmoddef\nwstartdeflinemarkup\nwprevnextdefs{NW4Nr7fb-38jgpJ-C}{NW4Nr7fb-38jgpJ-E}\nwenddeflinemarkup
\\def\\section#1\{\\par \\vskip3ex\\noindent \{\\bf #1\}\\par\\nobreak\\vskip1ex\\nobreak\}
\\def\\chapter#1\{\\vfil\\eject\\startsection\{0\}\{0pt\}\{6ex\}\{3ex\}\{\\frtbf\\chapcenter\}\{#1\}\}
\\def\\section#1\{\\startsection\{1\}\{0pt\}\{4ex\}\{2ex\}\{\\twlbf\}\{#1\}\}
\\def\\subsection#1\{\\startsection\{2\}\{0pt\}\{2ex\}\{1ex\}\{\\bf\}\{#1\}\}
\\def\\subsubsection#1\{\\startsection\{3\}\{0pt\}\{1ex\}\{0.5ex\}\{\\it\}\{#1\}\}
\\def\\paragraph#1\{\\startsection\{4\}\{0pt\}\{1.5ex\}\{0ex\}\{\\it\}\{#1\}\}

\LA{}kernel~{\nwtagstyle{}\subpageref{NW4Nr7fb-eb66b-1}}\RA{}

\\def\\nwfilename#1\{\\vfil\\eject\\mark\{#1\}\}

\\def\\nwbegindocs#1\{\\filbreak\}
\\def\\nwenddocs\{\\par\}
\\def\\nwbegincode#1\{\\par\\nobreak
  \\begingroup\\setupcode\\newlines\\parindent=0pt\\parskip=0pt
  %\\let\\oendmoddef=\\endmoddef \\let\\oplusendmoddef=\\plusendmoddef
  %\\def\\endmoddef\{\\oendmoddef\\par\}\\def\\plusendmoddef\{\\oplusendmoddef\\par\}%
  \\let\\onwenddeflinemarkup=\\nwenddeflinemarkup
  \\def\\nwenddeflinemarkup\{\\onwenddeflinemarkup\\par\}%
  \\hsize=\\codehsize\\noindent\\bchack\}
\\def\\nwendcode\{\\endgroup\}
\{\\catcode`\\^^M=\\active % make CR an active character
  \\gdef\\bchack#1^^M\{\\relax#1\}%
\}
\nwendcode{}\nwbegincode{265}\sublabel{NW4Nr7fb-38jgpJ-E}\nwmargintag{{\nwtagstyle{}\subpageref{NW4Nr7fb-38jgpJ-E}}}\moddef{nwmac.tex~{\nwtagstyle{}\subpageref{NW4Nr7fb-38jgpJ-1}}}\plusendmoddef\nwstartdeflinemarkup\nwprevnextdefs{NW4Nr7fb-38jgpJ-D}{NW4Nr7fb-38jgpJ-F}\nwenddeflinemarkup
\\edef\\contentsfile\{\\jobname.toc \} % file that gets table of contents info
\\def\\readcontents\{\\expandafter\\input \\contentsfile\}

\\newwrite\\cont
\\openout\\cont=\\contentsfile
\\write\\cont\{\\string\\catcode`\\string\\@=11\}% a hack to make contents
\nwendcode{}\nwbegincode{266}\sublabel{NW4Nr7fb-38jgpJ-F}\nwmargintag{{\nwtagstyle{}\subpageref{NW4Nr7fb-38jgpJ-F}}}\moddef{nwmac.tex~{\nwtagstyle{}\subpageref{NW4Nr7fb-38jgpJ-1}}}\plusendmoddef\nwstartdeflinemarkup\nwprevnextdefs{NW4Nr7fb-38jgpJ-E}{NW4Nr7fb-38jgpJ-G}\nwenddeflinemarkup
                                 % take stuff in plain.tex
\\def\\bye\{%
    \\write\\cont\{\}% ensure that the contents file isn't empty
    \\closeout\\cont
    \\vfil\\eject\\pageno=-1 % new page causes contents to be really closed
    \\topofcontents\\readcontents\\botofcontents
    \\vfil\\eject\\end\}
\\def\\topofcontents\{\\vfil\\mark\{\{\\bf Contents\}\}\}
\\def\\botofcontents\{\}
\nwendcode{}\nwbegincode{267}\sublabel{NW4Nr7fb-38jgpJ-G}\nwmargintag{{\nwtagstyle{}\subpageref{NW4Nr7fb-38jgpJ-G}}}\moddef{nwmac.tex~{\nwtagstyle{}\subpageref{NW4Nr7fb-38jgpJ-1}}}\plusendmoddef\nwstartdeflinemarkup\nwprevnextdefs{NW4Nr7fb-38jgpJ-F}{NW4Nr7fb-38jgpJ-H}\nwenddeflinemarkup
\\let\\em=\\it
% used to produce an itemized (bulleted) list in plain \{\\TeX\}
% such lists can be nested
% mostly useful with WEB

% Usage:
% \\itemize
% \\item First thing
% \\item second thing
% \\enditemize

\\newcount\\listlevel
\\listlevel=0
\\newdimen\\itemwidth
\\itemwidth=3em

\\def\\itemize\{\\begingroup\\advance\\listlevel by1
    \\def\\item\{\\par\\noindent
         \\raise2pt\\llap\{$\\scriptstyle\\bullet$\\ \}\\ignorespaces\}%
    \\def\\nameditem##1\{\\par\\noindent
         \\llap\{\\rlap\{##1\}\\hskip\\itemwidth\}\\ignorespaces\}%
    \\par\\advance\\leftskip by\\itemwidth\\advance\\rightskip by0.5\\itemwidth\}
\\def\\enditemize\{\\par\\endgroup\\noindent\\ignorespaces\}

\\let\\begindocument=\\relax
\nwendcode{}\nwbegindocs{268}\nwdocspar
Finally we make {\Tt{}@\nwendquote} `other' again.
\nwenddocs{}\nwbegincode{269}\sublabel{NW4Nr7fb-38jgpJ-H}\nwmargintag{{\nwtagstyle{}\subpageref{NW4Nr7fb-38jgpJ-H}}}\moddef{nwmac.tex~{\nwtagstyle{}\subpageref{NW4Nr7fb-38jgpJ-1}}}\plusendmoddef\nwstartdeflinemarkup\nwprevnextdefs{NW4Nr7fb-38jgpJ-G}{\relax}\nwenddeflinemarkup
\\catcode`\\@=12
\nwendcode{}\nwbegindocs{270}\nwdocspar
\section{Chunks} \nowebchunks
\twocolumn[\section{Index}]
\nowebindex*
\nwenddocs{}

\nwixlogsorted{c}{{$\mbox{\code{}{\nwbackslash}@nwhipage\edoc{}} := \mbox{\code{}{\nwbackslash}@nwlopage\edoc{}}+1$}{NW4Nr7fb-XOf88-1}{\nwixu{NW4Nr7fb-3olr1Q-O}\nwixd{NW4Nr7fb-XOf88-1}}}%
\nwixlogsorted{c}{{\code{}{\nwbackslash}nwbegincode\edoc{} separation and penalties}{NW4Nr7fb-23V9Kn-1}{\nwixu{NW4Nr7fb-3olr1Q-9}\nwixd{NW4Nr7fb-23V9Kn-1}}}%
\nwixlogsorted{c}{{\code{}{\nwbackslash}obeylines\edoc{} setup}{NW4Nr7fb-2oTcJP-1}{\nwixu{NW4Nr7fb-3olr1Q-A}\nwixd{NW4Nr7fb-2oTcJP-1}}}%
\nwixlogsorted{c}{{\code{}{\nwbackslash}trivlist\edoc{} clich\'e (\`a la {\Tt verbatim})}{NW4Nr7fb-2VyA2L-1}{\nwixu{NW4Nr7fb-3olr1Q-A}\nwixd{NW4Nr7fb-2VyA2L-1}}}%
\nwixlogsorted{c}{{add \code{}{\nwbackslash}nwcodecommentsep\edoc{} if this is the first \code{}{\nwbackslash}nwcodecomment\edoc{}}{NW4Nr7fb-7q7UY-1}{\nwixu{NW4Nr7fb-3olr1Q-J}\nwixd{NW4Nr7fb-7q7UY-1}}}%
\nwixlogsorted{c}{{add range to range list}{NW4Nr7fb-1VTyPN-1}{\nwixu{NW4Nr7fb-rhQSd-1}\nwixd{NW4Nr7fb-1VTyPN-1}\nwixu{NW4Nr7fb-3olr1Q-R}\nwixu{NW4Nr7fb-3olr1Q-S}}}%
\nwixlogsorted{c}{{definition of \code{}{\nwbackslash}newsublabel\edoc{}}{NW4Nr7fb-48i3K7-1}{\nwixu{NW4Nr7fb-3olr1Q-a}\nwixd{NW4Nr7fb-48i3K7-1}\nwixd{NW4Nr7fb-48i3K7-2}\nwixd{NW4Nr7fb-48i3K7-3}\nwixd{NW4Nr7fb-48i3K7-4}\nwixd{NW4Nr7fb-48i3K7-5}}}%
\nwixlogsorted{c}{{Icon code for subpage numbering}{NW4Nr7fb-41ggXv-1}{\nwixd{NW4Nr7fb-41ggXv-1}}}%
\nwixlogsorted{c}{{kernel}{NW4Nr7fb-eb66b-1}{\nwixd{NW4Nr7fb-eb66b-1}\nwixd{NW4Nr7fb-eb66b-2}\nwixd{NW4Nr7fb-eb66b-3}\nwixd{NW4Nr7fb-eb66b-4}\nwixd{NW4Nr7fb-eb66b-5}\nwixd{NW4Nr7fb-eb66b-6}\nwixd{NW4Nr7fb-eb66b-7}\nwixd{NW4Nr7fb-eb66b-8}\nwixd{NW4Nr7fb-eb66b-9}\nwixd{NW4Nr7fb-eb66b-A}\nwixu{NW4Nr7fb-3olr1Q-5}\nwixd{NW4Nr7fb-eb66b-B}\nwixu{NW4Nr7fb-38jgpJ-D}}}%
\nwixlogsorted{c}{{language support}{NW4Nr7fb-2c2IOz-1}{\nwixu{NW4Nr7fb-3olr1Q-1C}\nwixd{NW4Nr7fb-2c2IOz-1}\nwixd{NW4Nr7fb-2c2IOz-2}\nwixd{NW4Nr7fb-2c2IOz-3}\nwixd{NW4Nr7fb-2c2IOz-4}\nwixd{NW4Nr7fb-2c2IOz-5}\nwixd{NW4Nr7fb-2c2IOz-6}}}%
\nwixlogsorted{c}{{make all those damn active characters ``other''}{NW4Nr7fb-2GjufW-1}{\nwixu{NW4Nr7fb-1YqcEl-1}\nwixd{NW4Nr7fb-2GjufW-1}}}%
\nwixlogsorted{c}{{man page: \code{}{\nwbackslash}noweboptions\edoc{}}{NW4Nr7fb-3RhSlV-1}{\nwixd{NW4Nr7fb-3RhSlV-1}\nwixd{NW4Nr7fb-3RhSlV-2}\nwixd{NW4Nr7fb-3RhSlV-3}\nwixd{NW4Nr7fb-3RhSlV-4}\nwixd{NW4Nr7fb-3RhSlV-5}\nwixd{NW4Nr7fb-3RhSlV-6}\nwixd{NW4Nr7fb-3RhSlV-7}\nwixd{NW4Nr7fb-3RhSlV-8}\nwixd{NW4Nr7fb-3RhSlV-9}\nwixd{NW4Nr7fb-3RhSlV-A}\nwixd{NW4Nr7fb-3RhSlV-B}\nwixd{NW4Nr7fb-3RhSlV-C}}}%
\nwixlogsorted{c}{{man page: noweb style control sequences}{NW4Nr7fb-4PphKm-1}{\nwixd{NW4Nr7fb-4PphKm-1}\nwixd{NW4Nr7fb-4PphKm-2}\nwixd{NW4Nr7fb-4PphKm-3}}}%
\nwixlogsorted{c}{{new range starting with \code{}{\#}2\edoc{}}{NW4Nr7fb-rhQSd-1}{\nwixu{NW4Nr7fb-3olr1Q-O}\nwixd{NW4Nr7fb-rhQSd-1}}}%
\nwixlogsorted{c}{{normal range}{NW4Nr7fb-3rTQ1n-1}{\nwixu{NW4Nr7fb-8i24d-1}\nwixd{NW4Nr7fb-3rTQ1n-1}}}%
\nwixlogsorted{c}{{noweb.sty}{NW4Nr7fb-3olr1Q-1}{\nwixd{NW4Nr7fb-3olr1Q-1}\nwixd{NW4Nr7fb-3olr1Q-2}\nwixd{NW4Nr7fb-3olr1Q-3}\nwixd{NW4Nr7fb-3olr1Q-4}\nwixd{NW4Nr7fb-3olr1Q-5}\nwixd{NW4Nr7fb-3olr1Q-6}\nwixd{NW4Nr7fb-3olr1Q-7}\nwixd{NW4Nr7fb-3olr1Q-8}\nwixd{NW4Nr7fb-3olr1Q-9}\nwixd{NW4Nr7fb-3olr1Q-A}\nwixd{NW4Nr7fb-3olr1Q-B}\nwixd{NW4Nr7fb-3olr1Q-C}\nwixd{NW4Nr7fb-3olr1Q-D}\nwixd{NW4Nr7fb-3olr1Q-E}\nwixd{NW4Nr7fb-3olr1Q-F}\nwixd{NW4Nr7fb-3olr1Q-G}\nwixd{NW4Nr7fb-3olr1Q-H}\nwixd{NW4Nr7fb-3olr1Q-I}\nwixd{NW4Nr7fb-3olr1Q-J}\nwixd{NW4Nr7fb-3olr1Q-K}\nwixd{NW4Nr7fb-3olr1Q-L}\nwixd{NW4Nr7fb-3olr1Q-M}\nwixd{NW4Nr7fb-3olr1Q-N}\nwixd{NW4Nr7fb-3olr1Q-O}\nwixd{NW4Nr7fb-3olr1Q-P}\nwixd{NW4Nr7fb-3olr1Q-Q}\nwixd{NW4Nr7fb-3olr1Q-R}\nwixd{NW4Nr7fb-3olr1Q-S}\nwixd{NW4Nr7fb-3olr1Q-T}\nwixd{NW4Nr7fb-3olr1Q-U}\nwixd{NW4Nr7fb-3olr1Q-V}\nwixd{NW4Nr7fb-3olr1Q-W}\nwixd{NW4Nr7fb-3olr1Q-X}\nwixd{NW4Nr7fb-3olr1Q-Y}\nwixd{NW4Nr7fb-3olr1Q-Z}\nwixd{NW4Nr7fb-3olr1Q-a}\nwixd{NW4Nr7fb-3olr1Q-b}\nwixd{NW4Nr7fb-3olr1Q-c}\nwixd{NW4Nr7fb-3olr1Q-d}\nwixd{NW4Nr7fb-3olr1Q-e}\nwixd{NW4Nr7fb-3olr1Q-f}\nwixd{NW4Nr7fb-3olr1Q-g}\nwixd{NW4Nr7fb-3olr1Q-h}\nwixd{NW4Nr7fb-3olr1Q-i}\nwixd{NW4Nr7fb-3olr1Q-j}\nwixd{NW4Nr7fb-3olr1Q-k}\nwixd{NW4Nr7fb-3olr1Q-l}\nwixd{NW4Nr7fb-3olr1Q-m}\nwixd{NW4Nr7fb-3olr1Q-n}\nwixd{NW4Nr7fb-3olr1Q-o}\nwixd{NW4Nr7fb-3olr1Q-p}\nwixd{NW4Nr7fb-3olr1Q-q}\nwixd{NW4Nr7fb-3olr1Q-r}\nwixd{NW4Nr7fb-3olr1Q-s}\nwixd{NW4Nr7fb-3olr1Q-t}\nwixd{NW4Nr7fb-3olr1Q-u}\nwixd{NW4Nr7fb-3olr1Q-v}\nwixd{NW4Nr7fb-3olr1Q-w}\nwixd{NW4Nr7fb-3olr1Q-x}\nwixd{NW4Nr7fb-3olr1Q-y}\nwixd{NW4Nr7fb-3olr1Q-z}\nwixd{NW4Nr7fb-3olr1Q-10}\nwixd{NW4Nr7fb-3olr1Q-11}\nwixd{NW4Nr7fb-3olr1Q-12}\nwixd{NW4Nr7fb-3olr1Q-13}\nwixd{NW4Nr7fb-3olr1Q-14}\nwixd{NW4Nr7fb-3olr1Q-15}\nwixd{NW4Nr7fb-3olr1Q-16}\nwixd{NW4Nr7fb-3olr1Q-17}\nwixd{NW4Nr7fb-3olr1Q-18}\nwixd{NW4Nr7fb-3olr1Q-19}\nwixd{NW4Nr7fb-3olr1Q-1A}\nwixd{NW4Nr7fb-3olr1Q-1B}\nwixd{NW4Nr7fb-3olr1Q-1C}}}%
\nwixlogsorted{c}{{nwmac.tex}{NW4Nr7fb-38jgpJ-1}{\nwixd{NW4Nr7fb-38jgpJ-1}\nwixd{NW4Nr7fb-38jgpJ-2}\nwixd{NW4Nr7fb-38jgpJ-3}\nwixd{NW4Nr7fb-38jgpJ-4}\nwixd{NW4Nr7fb-38jgpJ-5}\nwixd{NW4Nr7fb-38jgpJ-6}\nwixd{NW4Nr7fb-38jgpJ-7}\nwixd{NW4Nr7fb-38jgpJ-8}\nwixd{NW4Nr7fb-38jgpJ-9}\nwixd{NW4Nr7fb-38jgpJ-A}\nwixd{NW4Nr7fb-38jgpJ-B}\nwixd{NW4Nr7fb-38jgpJ-C}\nwixd{NW4Nr7fb-38jgpJ-D}\nwixd{NW4Nr7fb-38jgpJ-E}\nwixd{NW4Nr7fb-38jgpJ-F}\nwixd{NW4Nr7fb-38jgpJ-G}\nwixd{NW4Nr7fb-38jgpJ-H}}}%
\nwixlogsorted{c}{{old noweb.sty}{NW4Nr7fb-1YYI5t-1}{\nwixd{NW4Nr7fb-1YYI5t-1}}}%
\nwixlogsorted{c}{{set \code{}{\nwbackslash}@tempa\edoc{} to page range(s), marked with \code{}{\nwbackslash}{\nwbackslash}\edoc{}}{NW4Nr7fb-PvOgT-1}{\nwixu{NW4Nr7fb-1VTyPN-1}\nwixd{NW4Nr7fb-PvOgT-1}}}%
\nwixlogsorted{c}{{undocumented -- man page: \code{}{\nwbackslash}noweboptions\edoc{}}{NW4Nr7fb-3PwwDi-1}{\nwixd{NW4Nr7fb-3PwwDi-1}}}%
\nwixlogsorted{c}{{use rules from Chicago style manual}{NW4Nr7fb-8i24d-1}{\nwixu{NW4Nr7fb-PvOgT-1}\nwixd{NW4Nr7fb-8i24d-1}}}%
\nwixlogsorted{c}{{warn of undefined reference to \code{}{\#}1\edoc{}}{NW4Nr7fb-4CAbV-1}{\nwixu{NW4Nr7fb-1dYqBd-1}\nwixu{NW4Nr7fb-1YYI5t-1}\nwixd{NW4Nr7fb-4CAbV-1}}}%
\nwixlogsorted{c}{{warn of undefined reference to \code{}{\#}1\edoc{} and add page ??}{NW4Nr7fb-1dYqBd-1}{\nwixu{NW4Nr7fb-3olr1Q-Q}\nwixd{NW4Nr7fb-1dYqBd-1}}}%
\nwixlogsorted{c}{{zap ligatures, fix spaces}{NW4Nr7fb-1YqcEl-1}{\nwixu{NW4Nr7fb-3olr1Q-A}\nwixd{NW4Nr7fb-1YqcEl-1}}}%
\nwbegindocs{271}\nwdocspar
\end{document}


\nwenddocs{}
